%%
%% Author: Alexander
%% 25.12.2017
%%

\documentclass[a4paper,10pt]{article}
\usepackage{mathtext}
\usepackage[T2A]{fontenc}
\usepackage[utf8]{inputenc}
\usepackage[russian]{babel}
\usepackage{amsmath}
\usepackage{amsfonts}
\usepackage{longtable}
\usepackage{amssymb}
\usepackage{graphicx}
\usepackage[left=2cm,right=2cm,
    top=2cm,bottom=2cm,bindingoffset=0cm]{geometry}
\usepackage{color}
\usepackage{gensymb}

\usepackage{enumitem}
\setlist[enumerate]{label*=\arabic*.}

\usepackage{indentfirst}

\usepackage{titlesec}

% Document
\begin{document}

%    </ macro rad_dist_alg() />

    При расчете параметров потока по высоте лопаточного венца будем пользоваться описанной ниже методикой.

    \begin{enumerate}

        \item Допущения:

        \begin{enumerate}

            \item Постоянство температуры торможения на входе в СА: $T_0^*(r) = const$.
            \item Постояноство скорости на входе: $c_0(r) = const$.
            \item Постоянство угла потока на входе: $\alpha_0(r) = const$.
            \item Постоянство полного давления на входе: $p_0^*(r) = const$.
            \item Постоянство работы на окружности колеса: $L_u(r) = const$.
            \item Потерь в лопаточных венцах нет.
            \item Ступень цилиндрическая.

        \end{enumerate}

%        </ if type == 'const_angle' />

        \item Тип профилирования: постоянный угол на выходе из СА.

%        </ elif type == 'const_circ' />

        \item Тип профилирования: постоянная циркуляция.

%        </ endif />

        \item Статическая температура на входе в СА:
        \[
            T_0(r) = T_0^* - \frac{c_0 ^ 2}{2 c_p}
        \]

        \item Окружная скорость на выходе из СА:
%        </ if type == 'const angle' />
        \[
            c_{1u}(r) = c_{1u\ ср} \cdot \left( \frac{r_{ср}}{r} \right) ^
                {\cos{\alpha_{1ср}} ^ 2}
        \]
%        </ elif type == 'const circ' />
        \[
            c_{1u}(r) = \frac{r_{ср}}{r} c_{1u\ ср}
        \]
%        </ endif />

        \item Осевая скорость на выходе из СА:
%        </ if type == 'const angle' />
        \[
            c_{1u}(r) = c_{1a\ ср} \cdot \left( \frac{r_{ср}}{r} \right) ^
                {\cos{\alpha_{1ср}} ^ 2}
        \]
%        </ elif type == 'const circ' />
        \[
            c_{1u}(r) = c_{1a\ ср}
        \]
%        </ endif />

        \item Скорость в абсолютном движении на выходе из СА:
        \[
            c_1(r) = \sqrt{c_{1a}(r)^2 + c_{1u}(r)^2}
        \]

        \item Угол потока в абсолютном движении на выходе из СА:
        \[
            \alpha_1 (r) = \arcsin{\frac{c_{1a}(r)}{c_1(r)}}
        \]

        \item Теплоперепад на СА:
        \[
            H_с (r) = \frac{c_1(r) ^ 2}{2}
        \]

        \item Статическое давление на выходе из СА:
        \[
            p_1 = p_0^*(r) \left( 1 - \frac{H_с (r)}{T_0^* c_p} \right) ^ {\frac{k}{k - 1}}
        \]

        \item Статическая температура на выходе из СА:
        \[
            T_1 (r) = T_0^*(r) - \frac{H_с (r)}{c_p}
        \]

        \item Окружная скорость:
        \[
            u(r) = \frac{2 \pi n r}{60}
        \]

        \item Температура торможения на выходе из РК:
        \[
            T_2^* = T_0^*(r) - \frac{L_u (r)}{c_p}
        \]

        \item Окружная скорость на выходе из РК:
        \[
            c_{2u}(r) = \frac{L_u(r) - c_{1u} u(r)}{u(r)}
        \]

        \item Осевая скорость на выходе из РК:
        \[
            c_{2a}(r) = \sqrt{
                    c_{2a\ ср}^2 + с_{2u\ ср}^2 - c_{2u}(r)^2 -
                    2 \cdot \int_{r_{ср}}^{r} \frac{c_{2u}(r)^2}{r} dr
            }
        \]

        \item Скорость в абсолютном движении на выходе из РК:
        \[
            c_2(r) = \sqrt{c_{2a}(r)^2 + c_{2u}(r)^2}
        \]

        \item Угол потока в абсолютном движении на выходе и РК:
        \[
            \alpha_2 (r) = \arctan{\frac{c_{2a}(r)}{c_{2u}(r)}}
        \]

        \item Окружная составляющая относительной скорости на выходе из РК:
        \[
            w_{2u}(r) = c_{2u}(r) + u(r)
        \]

        \item Относитеьная скорость на выходе из РК:
        \[
            w_2 (r) = \sqrt{w_{2u}(r)^2 + c_{2a}(r)^2}
        \]

        \item Относительная скорость на выходе из СА:
        \[
            w_1 (r) = \sqrt{c_1(r)^2 + u(r)^2 - 2 \cdot u(r) c_1(r) \cos{\alpha_1(r)}}
        \]

        \item Температура торможения в относительном движении на выходе из РК:
        \[
            T_{1w}^* = T_1(r) + \frac{w_1(r) ^ 2}{2 c_p}
        \]

        \item Теплоперепад в РК:
        \[
            H_л (r) = 0.5 \cdot (w_2(r)^2 - w_1(r)^2)
        \]

        \item Статическое давление на выходе из РК:
        \[
            p_2 (r) = p_1(r) \cdot \left( 1 - \frac{H_л}{c_p T_1(r)} \right) ^ {\frac{k}{k - 1}}
        \]

        \item Статическая температура на выходе из РК:
        \[
            T_2 (r) = T_1 (r) - \frac{w_2 (r)^2 - w_1(r)^2}{2 c_p}
        \]

        \item Статический теплоперепад на ступени:
        \[
            H_0 (r) = c_p \cdot T_0^*(r) \cdot \left( 1 - \frac{p_0^*(r)}{p_2(r)} \right) ^ {\frac{1 - k}{k}}
        \]

        \item Степень реактивности:
        \[
            \rho (r) = \frac{H_л (r)}{H_0 (r)}
        \]

    \end{enumerate}
%        </ endmacro />

%    </ macro rad_dist_res(params, caption='Параметры на различных радиусах.', type='const angle') />
    \begin{longtable}{
    |
%    </ for i in range(params['Value'][0].__len__() + 1) />
    c|
%    </ endfor />
    }
        \caption{<< caption >>}\\
        \hline

%        </ for i in range(params.__len__())/>
        << params['Name'][i] >>
%        </ for val in params['Value'][i]/>
        & << val >>
%        </ endfor />
        \\
        \hline
%        </ endfor />

    \end{longtable}

%        </ endmacro />


\end{document}