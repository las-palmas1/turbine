%%
%% Author: User1
%% 13.12.2017
%%

% Preamble
\documentclass[a4paper,10pt]{article}

% Packages
\usepackage{mathtext}
\usepackage[T2A]{fontenc}
\usepackage[utf8]{inputenc}
\usepackage{longtable}
\usepackage[russian]{babel}
\usepackage{amsmath}
\usepackage{amsfonts}
\usepackage{amssymb}
\usepackage{graphicx}
\usepackage[left=2cm,right=2cm,
    top=2cm,bottom=2cm,bindingoffset=0cm]{geometry}
\usepackage{color}
\usepackage{gensymb}

\usepackage{enumitem}
\setlist[enumerate]{label*=\arabic*.}

\usepackage{indentfirst}

\usepackage{titlesec}


% Document
\begin{document}

    \section{Расчет турбины по средней линии тока}

    </ import 'turb_average_streamline.tex' as avline />
    </ import 'turb_profiling.tex' as prof />

    \subsection{Расчет первой ступени}

    \subsubsection{Исходные данные}

    << avline.stage_init_data(turb[0], 'heat_drop') >>

    \subsubsection{Расчет}

    << avline.heat_drop_stage(turb[0]) >>

    \subsection{Расчет второй ступени}

    \subsubsection{Исходные данные}

    << avline.stage_init_data(turb[1], 'work') >>

    \subsubsection{Расчет}

    << avline.work_stage(turb[1]) >>

    \subsection{Расчет интегральных параметров турбины}

    </ set L_t_sum_exp = ((turb[0].L_t_prime / 10**6) | round(4)).__str__() + '\cdot 10^6' + '+' +
        ((turb[1].L_t_prime / 10**6) | round(4)).__str__() + '\cdot 10^6' />

    << avline.integrate_turbine_param(turb, L_t_sum_exp) >>

    \subsection{Параметры ступеней турбины}
    << avline.stage_tables(turb) >>

    \section{Расчет параметров потока по высоте.}

    << prof.rad_dist_alg() >>

    << prof.rad_dist_res(st1_params, 'Параметры первой ступени на различных радиусах.', st1_prof_type) >>




\end{document}