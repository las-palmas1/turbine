%%
%% Author: User1
%% 13.12.2017
%%

% Preamble
\documentclass[a4paper,10pt]{article}

% Packages
\usepackage{mathtext}
\usepackage[T2A]{fontenc}
\usepackage[utf8]{inputenc}
\usepackage{longtable}
\usepackage[russian]{babel}
\usepackage{amsmath}
\usepackage{amsfonts}
\usepackage{amssymb}
\usepackage{graphicx}
\usepackage[left=2cm,right=2cm,
    top=2cm,bottom=2cm,bindingoffset=0cm]{geometry}
\usepackage{color}
\usepackage{gensymb}

\usepackage{enumitem}
\setlist[enumerate]{label*=\arabic*.}

\usepackage{indentfirst}

\usepackage{titlesec}


% Document
\begin{document}

    \section{Расчет турбины по средней линии тока}

    
    
    

    \subsection{Расчет первой ступени}

    \subsubsection{Исходные данные}

    
    \begin{enumerate}

        \item Температура торможения на входе в ступень: $T_0^* = 1400\ К $.
        \item Давление торможения на входе в ступень: $p_0^* = 0.55 \cdot 10^6 \ Па$.
        \item Расход газа на входе в ступень: $G_{вх} = 25\ кг/с$.
        \item Расход газа на входе в СА первой ступени: $ G_т = 25\ кг/с $.
        \item Расход топлива на входе в турбину: $ G_{топл} = 1\ кг/с $.
        \item Степень реактивности: $ \rho = 0.4 $.
        \item Коэффициент скорости в СА: $ \phi = 0.97 $.
        \item Коэффициент скорости в РК: $ \psi = 0.97 $.
        \item Длина лопатки на входе в РК: $ l_1 = 0.1204\ м $.
        \item Длина лопатки на выходе из РК: $ l_2 = 0.1334\ м $.
        \item Средний диаметр на входе в РК: $ D_1 = 0.4816\ м $.
        \item Средний диаметр на выходе в РК: $ D_2 = 0.4919\ м $.
        \item Радиальный зазор: $ \delta_r = 0.00133\ м $.
        \item Частота вращения ротора: $ n = 15000.0\ об/мин $
        \item Степень парциальности: $ \varepsilon = 1 $.
        \item Расход охлаждающего воздуха, отнесенный к расходу на входе в турбину: $ g_{охл} = 0.004 $.
        \item Температура торможения охлаждающего воздуха: $ T_{охл} = 700\ К $.

        
        \item Статический теплоперепад на ступени: $ H_0 = 0.1636 \cdot 10^6 \ Дж/кг $.

        

    \end{enumerate}
    

    \subsubsection{Расчет}

    
    \begin{enumerate}

        \item Относительный расход топлива на входе в ступень:
        \[
            g_{топл.вх} = \frac{ G_{топл} }{ G_{вх} - G_{топл} } =
                \frac{ 1 }{ 25 - 1 } =
            0.0417
        \]

        \item Коэффициент избытка воздуха на входе:
        \[
            \alpha_{вх} = \frac{ 1 }{ l_0 g_{топл.вх} } =
                \frac{ 1 }{ 14.61 \cdot 0.0417 } =
            1.643
        \]

        \item Средний в ступени коэффициент адиабаты из предпоследней итерации:
        \[
            k_г = 1.2937
        \]

        \item Средняя в ступени теплоемкость газа из предпоследней итерации:
        \[
            c_{pг} = 1265.84 \ Дж/(кг \cdot К)
        \]

        
        

        

        \item Определим теплоперепад на сопловом аппарате:

        \[
            H_с = \left( 1 - \rho \right) H_0 =
	        \left( 1 - 0.4 \right) \cdot 0.1636 \cdot 10^6 =
            0.0982 \cdot 10^6 \/\ Дж/кг
        \]

        \item Окружная скорость на диаметре $ D_1 $:

        \[
            u_1 = \frac{\pi D_1 n }{60} =
                \frac{\pi \cdot 0.4816 \cdot 15000.0}{60} =
            378.22\ м/с
        \]

        \item Определим действительную скорость истечения из СА:

	    \[
            c_1 = \phi \sqrt{2 H_с} =
	        0.97 \cdot\sqrt{2 \cdot 0.1636 \cdot 10^6}  =
            429.82 \/\ м/с
        \]

        \item Определим температуру на выходе из СА:

	    \[
            T_1 = T_0^* - \frac{ H_с \phi^2 }{ c_{pг} } =
	        1400 -
            \frac{
                0.0982 \cdot 10^6 \cdot {0.97}^2
            }{
                2 \cdot 1265.84
            } = 1327.03 \/\ К
        \]

	    \item Определим температуру конца адиабатного расширения:

	    \[
            T_1^\prime = T_0^* - \frac{ H_c }{ c_{pг} } =
	        1400 -
            \frac{
                0.0982 \cdot 10^6
            }{
                1265.84
            }
            = 1322.44  \/\ К
        \]

        \item Определим давление на выходе из СА:

	    \[
            p_1 = p_0^* \left(
                                \frac{ T_1^\prime }{ T_0^* }
                        \right)^
                    \frac{ k_г }{ k_г - 1 } =
            0.55 \cdot 10^6 \cdot
                \left(
                        \frac{ 1322.44 }{ 1400 }
                \right)^
                \frac{ 1.2937 }{ 1.2937 - 1 } =
            0.4279 \cdot 10^6 \/\ МПа
        \]

        \item Определим площадь на выходе из СА:

	    \[
            A_{1a} = \pi l_1 D_1 =
	        \pi \cdot 0.1204 \cdot 0.4816 =
            0.18214 \/\ м^2
        \]

        \item Определим плотность газа на выходе из СА:

	    \[
            \rho_1 = \frac{p_1}{R_г T_1} =
	        \frac{
                0.4279 \cdot 10^6
            }{
                287.4 \cdot 1327.03
            } =
            1.122 \/\ кг/м^3
        \]

        \item Осевая составляющая абсолютной скорости на выходе из СА:

        \[
            c_{1a} = \frac{G_{вх} }{ \rho_1 A_{1a} } =
                \frac{
                    25
                }{
                    1.122 \cdot 0.18214
                } =
            122.34\ м/с
        \]

        \item Угол потока в абсолютном движении после СА:

        \[
            \alpha_1 = \arcsin{ \frac{ c_{1a} }{ c_1 } } =
            \arcsin{ \frac{ 122.34 }{ 429.82 } } =
            = 16.536 \degree
        \]

        \item Окружная составляющая абсолютной скорости на входе:

        \[
            c_{1u} = c_1 \cos{\alpha_1} = 429.82 \cdot \cos{16.536 \degree} =
            412.04\ м/с
        \]

        \item Определим относительную скорость на входе в РК:

	    \begin{gather*}
	        w_1 = \sqrt{c_1^2 + u_1^2 - 2 c_1 u_1 \cos \alpha_1} =\\
	        = \sqrt{
            429.82 ^ 2 +
            378.22 ^ 2 -
            2 \cdot 429.82 \cdot 378.22 \cdot \cos 16.16.536 \degree
            }
            = 126.93 \/\ м/с\\
	    \end{gather*}

        \item Угол потока в относительном движении:

        
        \[
            \beta_1 = \arctan{ \frac{c_{1a}}{c_{1u} - u_1} } =
                    \arctan{ \frac{ 122.34 }{412.04 - 378.22} } =
            74.545 \degree
        \]
        

        \item Осевая составляющая относительной скорости:

        \[
            w_{1a} = w_1 \sin{\beta_1} = 126.93 \cdot  \sin{74.545 \degree} =
            122.34\ м/с
        \]

        \item Окружная составляющая относительной скорости:

        \[
            w_{1u} = w_1 \cos{\beta_1} = 126.93 \cdot  \cos{74.545 \degree} =
            33.82\ м/с
        \]

         \item Определим теплоперепад на РК:

	    \[
            H_л = H_0 \rho \frac{T_1}{T_1^\prime} =
	        0.1636 \cdot 10^6 \cdot 0.4 \cdot
            \frac{ 1327.03 }{ 1322.44 } =
            0.0657 \cdot 10^6 \/\ Дж/кг
        \]

        \item Окружная скорость на диаметре:

        \[
            u_2 = \frac{ \pi D_2 n }{ 60 } =
                    \frac{ \pi \cdot 0.4919 \cdot 15000.0 }{ 60 } =
            386.36\ м/с
        \]

        \item Температура торможения в относительном движении после СА:

        \[
            T_{1w}^* = T_1 + \frac{ w_1^2 }{ 2 \cdot c_{pг}} =
                1327.03 + \frac{ 126.93 ^ 2 }{ 2 \cdot 1265.84}
        \]

        \item Определим относительную скорость истечения газа из РК:

	    \begin{gather*}
	        w_2 = \psi \sqrt{w_1^2 + 2H_л +\left( u_2^2 - u_1^2 \right)} =\\
	        = 0.97 \cdot
            \sqrt{
                126.93 ^ 2 +
                2 \cdot 0.0657 \cdot 10^6 +
                \left( 386.36 ^ 2 - 378.22 ^ 2 \right)
            } =
            380.27 \/\ м/с\\
	    \end{gather*}

        \item Определим статическую температуру на выходе из РК:

	    \begin{gather*}
	        T_2 = T_1 + \frac{
	 	        \left( w_1^2  - w_2^2 \right) + \left( u_2^2 - u_1^2 \right)
            }{
                2 c_{pг}
            } =\\
	        = 1327.03 + \frac{
	 	        \left( 126.93 ^ 2  - 380.27 ^ 2 \right) +
                \left( 386.36 ^ 2 - 378.22 ^ 2 \right)
	        }{
            2 \cdot 1265.84
            }
            = 1278.73 \/\ К\\
	    \end{gather*}

        \item Определим статическую температуру при адиабатическом процессе в РК:

	    \[
            T_2^\prime = T_1 - \frac{
	 	        H_л
	        }{ c_{p г}} =
	        1327.03 - \frac{
	 	        0.0657 \cdot 10^6
	        }{
                1265.84
            }
            = 1275.14 \/\ К
        \]

        \item Определим давление на выходе из РК:

	    \[
            p_2 = p_1 \left( \frac{T_2^\prime}{T_1} \right)^{\frac{k_г}{k_г - 1}} =
               0.4279 \cdot 10^6 \cdot
               \left(
               \frac{ 1275.14 }{ 1327.03 }
               \right) ^
               {\frac{
               1.294
               }{
               1.294 - 1
               }}
            = 0.359 \cdot 10^6 \/\ Па
        \]

        \item Определим плотность газа на выходе из РК:
	    \[
            \rho_2 = \frac{p_2}{R T_2} =
                \frac{
                    0.359 \cdot 10^6
                }{
                    287.4 \cdot 1278.73
                }
            = 0.977\ кг/м^3
        \]

        \item Определим площадь на выходе из РК:
        \[
            A_{2a} = \pi D_2 l_2 = \pi \cdot 0.4919 \cdot 0.1334 =
            0.2062\ м^2
        \]

        \item Осевая составляющая абсолютной скорости на выходе из РК:
        \[
            c_{2a} = \frac{ G_{вх} }{ A_{2a} \rho_2 } =
            \frac{ 25 }{ 0.2062 \cdot 0.977 }
            = 124.13\ м/с
        \]

        \item Угол потока в относительном движении на выходе из РК:
        \[
            \beta_2 = \arcsin{ \frac{ c_{2a} }{ w_2 } } =
                    \arcsin{ \frac{ 124.13 }{ 380.27 } }
            = 19.051 \degree
        \]

        \item Осевая составляющая относительной скорости потока на выходе из РК:
        \[
            w_{2a} = w_2 \cdot \sin{\beta_2} =
                    380.27 \cdot \sin{19.051 \degree}
            = 124.13\ м/с
        \]

        \item Окружная составляющая относительной скорости потока на выходе из РК:
        \[
            w_{2u} = w_2 \cdot \cos{\beta_2} =
                    380.27 \cdot \cos{19.051 \degree}
            = 359.44\ м/с
        \]

        \item Определим окружную составляющую скорости на выходе из РК:
	    \[
            c_{2u} = w_{2u} - u_2 =
	        359.44 - 386.36 = -26.92 \/\ м/с
        \]

        \item Опеределим угол потока на выходе из РК:
        
        \[
            \alpha_2 = \pi + \arctan{ \frac{ c_{2a} }{ c_{2u} } } =
                    \pi + \arctan{ \frac{ 124.13 }{ -26.92 } } =
            102.235 \degree
        \]
        

        \item Определим скорость потока на выходе из РК:
	    \[
            c_2 = \sqrt{c_{2u}^2 + c_{2a}^2} =
                \sqrt{-26.92 ^ 2 + 124.13 ^ 2} =
            127.01 \/\ м/с
        \]

        \item Определим работу на окружности колеса:
	    \[
            L_u = c_{1u} u_1 + c_{2u} u_2 =
                    412.04 \cdot 378.22 +
                    -26.92 \cdot 386.36 =
            0.1454 \cdot 10^6 \/\ Дж/кг
        \]

        \item Определим КПД на окружности колеса:
	    \[
            \eta_u = \frac{L_u}{H_0} =
                \frac{ 0.1454 \cdot 10^6 }{ 0.1636 \cdot 10^6 }
            = 0.8889
        \]

        \item Определим удельные потери в СА:
	    \[
            h_с = \left(
                        \frac{ 1 }{ \phi^2 } - 1
                \right)
                \frac{ c_1^2 }{ 2 } =
	        \left(
                \frac{ 1 }{ 0.97 ^ 2} - 1
            \right) \cdot
            \frac{ 429.82 ^ 2 }{ 2 } = 5.8022 \cdot 10^3 \/\ Дж/кг
        \]

        \item Удельные потери в СА с учетом их использования в рабочих лопатках:
        \[
            h_с^\prime = h_с \frac{ T_2^\prime }{ T_1 } =
                5.8022 \cdot 10^3 \cdot
                \frac{ 1275.14 }{ 1327.03 } =
            5.5753 \cdot 10^3 \/\ Дж/кг
        \]

        \item Относительные потери в СА:
        \[
            \zeta_с = \frac{ h_с }{ H_0 } =
                \frac{ 5.8022 \cdot 10^3 }{ 0.1636 \cdot 10^6 } =
            0.0355
        \]

        \item Относительные потери в СА с учетом их использования в рабочих лопатках:
        \[
            \zeta_с^\prime = \frac{ h_с^\prime }{ H_0 } =
                \frac{ 5.5753 \cdot 10^3 }{ 0.1636 \cdot 10^6 } =
            0.0341
        \]

        \item Удельные потери в рабочих лопатках:
        \[
            h_л = \left(
                    \frac{ 1 }{ \psi^2 } - 1
                \right)) \cdot
                \frac{ w_2^2 }{ 2 } =
            \left(
                \frac{ 1 }{ 0.97 ^ 2 } - 1
            \right) \cdot
            \frac{ 380.27 ^ 2} {2}
            = 4.5416 \cdot 10^3 \/\ Дж/кг
        \]

        \item Относительные потери в рабочих лопатках:
        \[
            \zeta_л = \frac{ h_л }{ H_0 } =
                \frac{ 4.5416 \cdot 10^3 }{ 0.1636 \cdot 10^6 } =
            0.0278
        \]

        \item Определим удельные потери с выходной скоростью:
        \[
            h_{вых} = \frac{ c_2 ^ 2 }{ 2} =
                    \frac{ 127.01 ^ 2 }{ 2 } =  8.0658 \cdot 10^3 \/\ Дж/кг
        \]

        \item Относительные потери с выходной скоростью:
        \[
            \zeta_{вых} = \frac{ h_{вых} }{ H_0 } =
                \frac{ 8.0658 \cdot 10^3 }{ 0.1636 \cdot 10^6 } =
            0.0493
        \]

        \item Проверка КПД на окружности колеса:
        \[
            \eta_u = 1 - \zeta_с^\prime - \zeta_л - \zeta_{вых} = 1 - 0.0341 -
                    0.0278 - 0.0493 = 0.8889
        \]

        \item Средний диаметр:
        \[
            D_{ср} = 0.5 \cdot (D_1 + D_2) =
                    0.5 \cdot (0.4816 + 0.4919) =
            0.4867\ м
        \]

        \item Определим удельные потери в радиальном зазоре:

	    \begin{gather*}
	        h_з = 1.37 \cdot
                \left(
                    1 + 1.6 \rho
                \right)
                \left(
                    1 + \frac{l_2}{D_{ср}}
                \right)
            \frac{ \delta_r }{ l_2 } \cdot L_u =\\
	        = 1.37 \cdot
            \left(
                1 + 1.6 \cdot 0.4
            \right)
            \left(
                1 + \frac{ 0.1334 }{ 0.4867 }
            \right)
            \frac{ 0.00133 }{ 0.1334 } \cdot
            0.1454 \cdot 10^6 =
	        4.1636 \cdot 10^3 \/\ Дж/кг\\
	    \end{gather*}

        \item Относительные удельные потери в радиальном зазоре:
        \[
            \zeta_з = \frac{ h_з }{ H_0 } =
                \frac{ 4.1636 \cdot 10^3 }{ 0.1636 \cdot 10^6 } =
            0.0254
        \]

        \item Удельная работа ступени с учетом потери в радиальном зазоре:
        \[
            L_{uз} = L_u - h_з = 0.1454 \cdot 10^6 -
                4.1636 \cdot 10^3 =
            0.1413 \cdot 10^6 \ Дж/кг
        \]

        \item Мощностной КПД ступени:
        \[
            \eta_т^\prime = \eta_u - \zeta_з =
                0.8889 - 0.0254 = 0.8634
        \]

        \item Лопаточный КПД ступени:
        \[
            \eta_л^\prime = \eta_т^\prime + \zeta_{вых} =
                 0.8634 +  0.0493 =
            0.9127
        \]

        \item Средняя длина лопатки:
        \[
            l_{ср} = 0.5 \cdot (l_1 + l_2) =
                0.5 \cdot (0.1204 + 0.1334) =
            0.1269\ м
        \]

        \item Средняя окружная скоротсь:
        \[
            u_{ср} = 0.5 \cdot (u_1 + u_2) =
                0.5 \cdot (378.22 + 386.36) =
            382.29\ м/с
        \]

        \item Затраты мощности на трение и вентиляцию:
        \begin{gather*}
            N_{т.в} = \left[
                    1.07 \cdot D_{av}^2 + 61 \cdot (1 - \varepsilon) \cdot D_{av} l_{av}
            \right] \cdot
            \left(
                \frac{ u_{av} }{ 100 }
            \right) ^ 3 \cdot
            \rho =\\
            = \left[
                1.07 \cdot 0.4867^2 +
                61 \cdot (1 - 1) \cdot
                0.4867 \cdot 0.1269
            \right] \cdot
            \left(
                \frac{ 382.29 }{ 100 }
            \right) ^ 3 \cdot
            0.4=\\
            = 0.0057 \cdot 10^3 \ Вт \\
        \end{gather*}

        \item Удельные потери на трение и вентиляцию:
        \[
            h_{т.в} = \frac{ N_{т.в} }{ G_{вх} } =
                \frac{
                    0.0057 \cdot 10^3
                }{
                    25
                }
            = 0.0002 \cdot 10^3 \ Дж/кг
        \]

        \item Относительные потери на трение и вентиляцию:
        \[
            \zeta_{т.в} = \frac{ h_{т.в} }{ H_0 } =
                \frac{ 0.0002 \cdot 10^3 }{ 0.1636 \cdot 10^6 } =
            0.0
        \]

        \item Мощностной КПД с учетом потерь на трению и вентиляцию:
        \[
            \eta_т = \eta_т^\prime - \zeta_{т.в} =
                0.8634 - 0.0 =
            0.8634
        \]

        \item Лопаточный КПД с учетом потерь на трению и вентиляцию:
        \[
            \eta_л = \eta_л^\prime - \zeta_{т.в} =
                0.9127 - 0.0 =
            0.9127
        \]

        
        \item Определим удельную работу ступени:
        \[
            L_т = H_0 \eta_т = 0.1636 \cdot 10^6 \cdot \eta_т =
            = 0.1413 \cdot 10^6 \ Дж/кг
        \]
        

        \item Удельная работа ступени, отнесенная к расходу на в СА первой ступени:
        \[
            L_т^\prime = L_т \frac{ G_{вх} }{ G_т }  =
                0.1413 \cdot 10^6 \cdot
                \frac{ 25 }{ 25 } =
            0.1413 \cdot 10^6 \ Дж/кг
        \]

        \item Статическая температура за ступенью:
        \[
            T_{ст} = T_2 + \frac{ h_з }{ c_{pг} } + \frac{ h_{т.в} }{ c_{pг} } =
                1278.73 +
                \frac{4.1636 \cdot 10^3 }{ 1265.84 } +
                \frac{ 0.0002 \cdot 10^3 }{ 1265.84 } =
            1282.02 \ К
        \]

        \item Температура торможения за турбиной:
        \[
            T_{ст}^* = T_{ст} + \frac{ h_{вых} }{ c_{pг} } =
                1282.02 +
                \frac{ 8.0658 \cdot 10^3 }{ 1265.84 } =
            1288.39 \ К
        \]

        \item Средняя теплоемкость газа в интервале температур от 273 К до $T_0^*$:
        \[
            c_{pг\ ср} (T_0^*, \alpha_{вх}) =
            1142.12 \ Дж/(кг \cdot К)
        \]

        \item Средняя теплоемкость газа в интервале температур от 273 К до $T_{ст}$:
        \[
            c_{pг\ ср} (T_{ст}, \alpha_{вх}) =
            1127.49 \ Дж/(кг \cdot К)
        \]

        \item Средняя теплоемкость газа в интервале температур от $T_0^*$ до $T_{ст}$:
        \begin{gather*}
            c_{pг}^\prime = \frac{
		        c_{pг\ ср} (T_0^*, \alpha_{вх}) (T_0^* - T_0) - c_{pг\ ср} (T_{ст}, \alpha_{вх})(T_{ст} - T_0)
		    }{
		        T_0^* - T_{ст}} =\\
            =\frac{
		        1142.12 \cdot
                (1400 - 273) -
		        1127.49 \cdot
                (1282.02 - 273)
		    }{
		        1400 - 1282.02} =
		    1267.21 \ Дж / (кг \cdot К)\\
        \end{gather*}

        \item Новое значение показателя адиабаты:
        \[
            k_г^\prime = \frac{c_{pг}^\prime}{c_{pг}^\prime - R_г} =
                \frac{
                    1267.21
                }{
                    1267.21 - 287.4
                }
            = 1.2933
        \]

        \item Невязка по коэффициенту адиабаты:
        \[
            \delta = \frac{ \left| k_г - k_г^\prime \right| }{ k_г } \cdot 100 \%=
                \frac{
                    \left| 1.2937 - 1.2933 \right|
                }{
                    1.2937
                } \cdot 100 \% =
            0.0319 \%
        \]

        \item Давление торможения на выходе из ступени:
        \[
            p_2^* = p_2 \left(
                            \frac{ T_{ст}^* }{ T_{ст} }
                    \right) ^ \frac{ k_г }{ k_г - 1 } =
                 0.359 \cdot 10^6 \cdot \left(
                            \frac{ 1288.39 }{ 1282.02 }
                    \right) ^
                \frac{ 1.2937 }{ 1.2937 - 1 } =
            0.3669 \cdot 10^6 \ Па
        \]

        \item Теплоперепад по параметрам торможения:
        \begin{gather*}
            H_0^* = c_{pг} T_0^* \left[
                        1 - \left(
                                \frac{p_2^*}{p_0^*}
                            \right) ^
                        \frac{k_г - 1}{k_г}
                    \right] =\\
            1265.84 \cdot 1400 \cdot
                    \left[
                        1 - \left(
                                \frac{
                                    0.3669 \cdot 10^6
                                }{
                                    0.55 \cdot 10^6
                                }
                            \right) ^
                        \frac{1.2937 - 1}{1.2937}
                    \right]
            = 0.1556 \cdot 10^6 \ Дж/кг\\
        \end{gather*}

        \item КПД по параметрам торможения:
        \[
            \eta_т^* = \frac{ L_т }{ H_0^* } =
                \frac{
                    0.1413 \cdot 10^6
                }{
                    0.1556 \cdot 10^6 } =
            0.9078
        \]

        \item Расход на выходе из ступени:
        \[
            G_{вых} = G_{вх} + G_т g_{охл} =
                25 + 25 \cdot
                0.004 =
            25.1 \ кг/с
        \]

        \item Относительный расход топлива на выходе из ступени:
        \[
            g_{топл.вых} = \frac{ G_{топл} }{ G_{вых} - G_{топл} } =
                 \frac{ 1 }{ 25.1 - 1 } =
            0.0415
        \]

        \item Коэффициент избытка воздуха на выходе из ступени:
        \[
            \alpha_{вых} = \frac{ 1 }{ l_0 g_{топл.вых} } =
                \frac{ 1 }{ 14.61 \cdot 0.0415 } =
            1.65
        \]

        \item Абсолютный расход охлаждающего воздуха:
        \[
            G_{охл} = G_т g_{охл} = 25 \cdot 0.004 =
            0.1
        \]

        \item Определим температуру торможения на выходе из турбины после подмешивания охлаждающего воздуха.
        \begin{enumerate}

            \item Истинная теплоемкость охлаждающего воздуха при температуре $T_{охл} = 700\ К $:
            \[
                c_{pв} (T_{охл}) = 1074.93\ Дж/ (кг \cdot К)
            \]

            \item Истинная теплоемкость газа при температуре $T_{ст}^* = 1288.39 \ К $:
            \[
                c_{pг} (T_{ст}^*, \alpha_{вх}) =
                1280.34\ Дж/ (кг \cdot К)
            \]

            \item Значение температуры смеси с предпоследней итерации $T_{см}^{*} = 1285.98\ К$.

            \item Истинная теплоемкость смеси:
            \[
                c_{pг} (T_{см}^{*}, \alpha_{вых}) =
                1279.54\ Дж/ (кг \cdot К)
            \]

            \item Новое значение температуры смеси:
            \begin{gather*}
                T_{см}^*\prime = \frac{
                        c_{pг} (T_{ст}^*, \alpha_{вх}) T_{ст}^* G_{вх} + c_{pв} (T_{охл}) T_{охл} G_{охл}
                    }{
                        c_{pг} (T_{см}^{*}, \alpha_{вых}) G_{вых}
                    } =\\
                = \frac{
                    1280.34
                    \cdot 1288.39 \cdot 25 +
                    1074.93
                    \cdot 700 \cdot 0.1
                }{
                    1279.54
                    \cdot  25.1
                } =
                1286.4\ К\\
            \end{gather*}

            \item Значение невязки:
            \[
                \delta = \frac{ \left| T_{см}^{*} - T_{см}^*\prime \right| }{T_{см}^{*}} \cdot 100 \% =
                    \frac{
                        \left| 1285.98 - 1286.4 \right|
                    }{
                        1285.98
                    } \cdot 100 \% =
                0.033 \%
            \]

        \end{enumerate}

        

    \end{enumerate}
    

    \subsection{Расчет второй ступени}

    \subsubsection{Исходные данные}

    
    \begin{enumerate}

        \item Температура торможения на входе в ступень: $T_0^* = 1285.98\ К $.
        \item Давление торможения на входе в ступень: $p_0^* = 0.3669 \cdot 10^6 \ Па$.
        \item Расход газа на входе в ступень: $G_{вх} = 25.1\ кг/с$.
        \item Расход газа на входе в СА первой ступени: $ G_т = 25\ кг/с $.
        \item Расход топлива на входе в турбину: $ G_{топл} = 1\ кг/с $.
        \item Степень реактивности: $ \rho = 0.496 $.
        \item Коэффициент скорости в СА: $ \phi = 0.97 $.
        \item Коэффициент скорости в РК: $ \psi = 0.97 $.
        \item Длина лопатки на входе в РК: $ l_1 = 0.1691\ м $.
        \item Длина лопатки на выходе из РК: $ l_2 = 0.1874\ м $.
        \item Средний диаметр на входе в РК: $ D_1 = 0.5122\ м $.
        \item Средний диаметр на выходе в РК: $ D_2 = 0.5268\ м $.
        \item Радиальный зазор: $ \delta_r = 0.00187\ м $.
        \item Частота вращения ротора: $ n = 15000.0\ об/мин $
        \item Степень парциальности: $ \varepsilon = 1 $.
        \item Расход охлаждающего воздуха, отнесенный к расходу на входе в турбину: $ g_{охл} = 0.003 $.
        \item Температура торможения охлаждающего воздуха: $ T_{охл} = 700\ К $.

        
        \item Удельная работы турбины: $ L_т = 0.1094 \cdot 10^6 \ Дж/кг $.

        

    \end{enumerate}
    

    \subsubsection{Расчет}

    

    \begin{enumerate}

        \item Относительный расход топлива на входе в ступень:
        \[
            g_{топл.вх} = \frac{ G_{топл} }{ G_{вх} - G_{топл} } =
                \frac{ 1 }{ 25.1 - 1 } =
            0.0415
        \]

        \item Коэффициент избытка воздуха на входе:
        \[
            \alpha_{вх} = \frac{ 1 }{ l_0 g_{топл.вх} } =
                \frac{ 1 }{ 14.61 \cdot 0.0415 } =
            1.65
        \]

        \item Мощностной КПД из предпоследней итерации:
        \[
            \eta_{т0} = 0.852
        \]

        \item Статический теплоперепад на ступени:
        \[
            H_0 = \frac{L_т}{\eta_{т0}} =
                \frac{ 0.1094 \cdot 10^6 }{ 0.852 } =
            0.1284 \cdot 10^6 \ Дж/кг
        \]

        \item Средний в ступени коэффициент адиабаты из предпоследней итерации:
        \[
            k_г = 1.3015
        \]

        \item Средняя в ступени теплоемкость газа из предпоследней итерации:
        \[
            c_{pг} = 1240.59 \ Дж/(кг \cdot К)
        \]

        
        

        

        \item Определим теплоперепад на сопловом аппарате:

        \[
            H_с = \left( 1 - \rho \right) H_0 =
	        \left( 1 - 0.496 \right) \cdot 0.1284 \cdot 10^6 =
            0.0647 \cdot 10^6 \/\ Дж/кг
        \]

        \item Окружная скорость на диаметре $ D_1 $:

        \[
            u_1 = \frac{\pi D_1 n }{60} =
                \frac{\pi \cdot 0.5122 \cdot 15000.0}{60} =
            402.32\ м/с
        \]

        \item Определим действительную скорость истечения из СА:

	    \[
            c_1 = \phi \sqrt{2 H_с} =
	        0.97 \cdot\sqrt{2 \cdot 0.1284 \cdot 10^6}  =
            348.85 \/\ м/с
        \]

        \item Определим температуру на выходе из СА:

	    \[
            T_1 = T_0^* - \frac{ H_с \phi^2 }{ c_{pг} } =
	        1285.98 -
            \frac{
                0.0647 \cdot 10^6 \cdot {0.97}^2
            }{
                2 \cdot 1240.59
            } = 1236.93 \/\ К
        \]

	    \item Определим температуру конца адиабатного расширения:

	    \[
            T_1^\prime = T_0^* - \frac{ H_c }{ c_{pг} } =
	        1285.98 -
            \frac{
                0.0647 \cdot 10^6
            }{
                1240.59
            }
            = 1233.85  \/\ К
        \]

        \item Определим давление на выходе из СА:

	    \[
            p_1 = p_0^* \left(
                                \frac{ T_1^\prime }{ T_0^* }
                        \right)^
                    \frac{ k_г }{ k_г - 1 } =
            0.3669 \cdot 10^6 \cdot
                \left(
                        \frac{ 1233.85 }{ 1285.98 }
                \right)^
                \frac{ 1.3015 }{ 1.3015 - 1 } =
            0.3069 \cdot 10^6 \/\ МПа
        \]

        \item Определим площадь на выходе из СА:

	    \[
            A_{1a} = \pi l_1 D_1 =
	        \pi \cdot 0.1691 \cdot 0.5122 =
            0.27214 \/\ м^2
        \]

        \item Определим плотность газа на выходе из СА:

	    \[
            \rho_1 = \frac{p_1}{R_г T_1} =
	        \frac{
                0.3069 \cdot 10^6
            }{
                287.4 \cdot 1236.93
            } =
            0.863 \/\ кг/м^3
        \]

        \item Осевая составляющая абсолютной скорости на выходе из СА:

        \[
            c_{1a} = \frac{G_{вх} }{ \rho_1 A_{1a} } =
                \frac{
                    25.1
                }{
                    0.863 \cdot 0.27214
                } =
            106.84\ м/с
        \]

        \item Угол потока в абсолютном движении после СА:

        \[
            \alpha_1 = \arcsin{ \frac{ c_{1a} }{ c_1 } } =
            \arcsin{ \frac{ 106.84 }{ 348.85 } } =
            = 17.834 \degree
        \]

        \item Окружная составляющая абсолютной скорости на входе:

        \[
            c_{1u} = c_1 \cos{\alpha_1} = 348.85 \cdot \cos{17.834 \degree} =
            332.09\ м/с
        \]

        \item Определим относительную скорость на входе в РК:

	    \begin{gather*}
	        w_1 = \sqrt{c_1^2 + u_1^2 - 2 c_1 u_1 \cos \alpha_1} =\\
	        = \sqrt{
            348.85 ^ 2 +
            402.32 ^ 2 -
            2 \cdot 348.85 \cdot 402.32 \cdot \cos 16.17.834 \degree
            }
            = 127.86 \/\ м/с\\
	    \end{gather*}

        \item Угол потока в относительном движении:

        
        \[
            \beta_1 = \pi + \arctan{ \frac{c_{1a}}{c_{1u} - u_1} } =
                    \pi + \arctan{ \frac{ 106.84 }{332.09 - 402.32} } =
            123.316 \degree
        \]
        

        \item Осевая составляющая относительной скорости:

        \[
            w_{1a} = w_1 \sin{\beta_1} = 127.86 \cdot  \sin{123.316 \degree} =
            106.84\ м/с
        \]

        \item Окружная составляющая относительной скорости:

        \[
            w_{1u} = w_1 \cos{\beta_1} = 127.86 \cdot  \cos{123.316 \degree} =
            -70.23\ м/с
        \]

         \item Определим теплоперепад на РК:

	    \[
            H_л = H_0 \rho \frac{T_1}{T_1^\prime} =
	        0.1284 \cdot 10^6 \cdot 0.4962 \cdot
            \frac{ 1236.93 }{ 1233.85 } =
            0.0638 \cdot 10^6 \/\ Дж/кг
        \]

        \item Окружная скорость на диаметре:

        \[
            u_2 = \frac{ \pi D_2 n }{ 60 } =
                    \frac{ \pi \cdot 0.5268 \cdot 15000.0 }{ 60 } =
            413.75\ м/с
        \]

        \item Температура торможения в относительном движении после СА:

        \[
            T_{1w}^* = T_1 + \frac{ w_1^2 }{ 2 \cdot c_{pг}} =
                1236.93 + \frac{ 127.86 ^ 2 }{ 2 \cdot 1240.59}
        \]

        \item Определим относительную скорость истечения газа из РК:

	    \begin{gather*}
	        w_2 = \psi \sqrt{w_1^2 + 2H_л +\left( u_2^2 - u_1^2 \right)} =\\
	        = 0.97 \cdot
            \sqrt{
                127.86 ^ 2 +
                2 \cdot 0.0638 \cdot 10^6 +
                \left( 413.75 ^ 2 - 402.32 ^ 2 \right)
            } =
            379.88 \/\ м/с\\
	    \end{gather*}

        \item Определим статическую температуру на выходе из РК:

	    \begin{gather*}
	        T_2 = T_1 + \frac{
	 	        \left( w_1^2  - w_2^2 \right) + \left( u_2^2 - u_1^2 \right)
            }{
                2 c_{pг}
            } =\\
	        = 1236.93 + \frac{
	 	        \left( 127.86 ^ 2  - 379.88 ^ 2 \right) +
                \left( 413.75 ^ 2 - 402.32 ^ 2 \right)
	        }{
            2 \cdot 1240.59
            }
            = 1189.12 \/\ К\\
	    \end{gather*}

        \item Определим статическую температуру при адиабатическом процессе в РК:

	    \[
            T_2^\prime = T_1 - \frac{
	 	        H_л
	        }{ c_{p г}} =
	        1236.93 - \frac{
	 	        0.0638 \cdot 10^6
	        }{
                1240.59
            }
            = 1185.46 \/\ К
        \]

        \item Определим давление на выходе из РК:

	    \[
            p_2 = p_1 \left( \frac{T_2^\prime}{T_1} \right)^{\frac{k_г}{k_г - 1}} =
               0.3069 \cdot 10^6 \cdot
               \left(
               \frac{ 1185.46 }{ 1236.93 }
               \right) ^
               {\frac{
               1.302
               }{
               1.302 - 1
               }}
            = 0.2554 \cdot 10^6 \/\ Па
        \]

        \item Определим плотность газа на выходе из РК:
	    \[
            \rho_2 = \frac{p_2}{R T_2} =
                \frac{
                    0.2554 \cdot 10^6
                }{
                    287.4 \cdot 1189.12
                }
            = 0.747\ кг/м^3
        \]

        \item Определим площадь на выходе из РК:
        \[
            A_{2a} = \pi D_2 l_2 = \pi \cdot 0.5268 \cdot 0.1874 =
            0.3102\ м^2
        \]

        \item Осевая составляющая абсолютной скорости на выходе из РК:
        \[
            c_{2a} = \frac{ G_{вх} }{ A_{2a} \rho_2 } =
            \frac{ 25.1 }{ 0.3102 \cdot 0.747 }
            = 108.26\ м/с
        \]

        \item Угол потока в относительном движении на выходе из РК:
        \[
            \beta_2 = \arcsin{ \frac{ c_{2a} }{ w_2 } } =
                    \arcsin{ \frac{ 108.26 }{ 379.88 } }
            = 16.558 \degree
        \]

        \item Осевая составляющая относительной скорости потока на выходе из РК:
        \[
            w_{2a} = w_2 \cdot \sin{\beta_2} =
                    379.88 \cdot \sin{16.558 \degree}
            = 108.26\ м/с
        \]

        \item Окружная составляющая относительной скорости потока на выходе из РК:
        \[
            w_{2u} = w_2 \cdot \cos{\beta_2} =
                    379.88 \cdot \cos{16.558 \degree}
            = 364.12\ м/с
        \]

        \item Определим окружную составляющую скорости на выходе из РК:
	    \[
            c_{2u} = w_{2u} - u_2 =
	        364.12 - 413.75 = -49.63 \/\ м/с
        \]

        \item Опеределим угол потока на выходе из РК:
        
        \[
            \alpha_2 = \pi + \arctan{ \frac{ c_{2a} }{ c_{2u} } } =
                    \pi + \arctan{ \frac{ 108.26 }{ -49.63 } } =
            114.627 \degree
        \]
        

        \item Определим скорость потока на выходе из РК:
	    \[
            c_2 = \sqrt{c_{2u}^2 + c_{2a}^2} =
                \sqrt{-49.63 ^ 2 + 108.26 ^ 2} =
            119.1 \/\ м/с
        \]

        \item Определим работу на окружности колеса:
	    \[
            L_u = c_{1u} u_1 + c_{2u} u_2 =
                    332.09 \cdot 402.32 +
                    -49.63 \cdot 413.75 =
            0.1131 \cdot 10^6 \/\ Дж/кг
        \]

        \item Определим КПД на окружности колеса:
	    \[
            \eta_u = \frac{L_u}{H_0} =
                \frac{ 0.1131 \cdot 10^6 }{ 0.1284 \cdot 10^6 }
            = 0.8809
        \]

        \item Определим удельные потери в СА:
	    \[
            h_с = \left(
                        \frac{ 1 }{ \phi^2 } - 1
                \right)
                \frac{ c_1^2 }{ 2 } =
	        \left(
                \frac{ 1 }{ 0.97 ^ 2} - 1
            \right) \cdot
            \frac{ 348.85 ^ 2 }{ 2 } = 3.8221 \cdot 10^3 \/\ Дж/кг
        \]

        \item Удельные потери в СА с учетом их использования в рабочих лопатках:
        \[
            h_с^\prime = h_с \frac{ T_2^\prime }{ T_1 } =
                3.8221 \cdot 10^3 \cdot
                \frac{ 1185.46 }{ 1236.93 } =
            3.6631 \cdot 10^3 \/\ Дж/кг
        \]

        \item Относительные потери в СА:
        \[
            \zeta_с = \frac{ h_с }{ H_0 } =
                \frac{ 3.8221 \cdot 10^3 }{ 0.1284 \cdot 10^6 } =
            0.0298
        \]

        \item Относительные потери в СА с учетом их использования в рабочих лопатках:
        \[
            \zeta_с^\prime = \frac{ h_с^\prime }{ H_0 } =
                \frac{ 3.6631 \cdot 10^3 }{ 0.1284 \cdot 10^6 } =
            0.0285
        \]

        \item Удельные потери в рабочих лопатках:
        \[
            h_л = \left(
                    \frac{ 1 }{ \psi^2 } - 1
                \right)) \cdot
                \frac{ w_2^2 }{ 2 } =
            \left(
                \frac{ 1 }{ 0.97 ^ 2 } - 1
            \right) \cdot
            \frac{ 379.88 ^ 2} {2}
            = 4.5321 \cdot 10^3 \/\ Дж/кг
        \]

        \item Относительные потери в рабочих лопатках:
        \[
            \zeta_л = \frac{ h_л }{ H_0 } =
                \frac{ 4.5321 \cdot 10^3 }{ 0.1284 \cdot 10^6 } =
            0.0353
        \]

        \item Определим удельные потери с выходной скоростью:
        \[
            h_{вых} = \frac{ c_2 ^ 2 }{ 2} =
                    \frac{ 119.1 ^ 2 }{ 2 } =  7.0919 \cdot 10^3 \/\ Дж/кг
        \]

        \item Относительные потери с выходной скоростью:
        \[
            \zeta_{вых} = \frac{ h_{вых} }{ H_0 } =
                \frac{ 7.0919 \cdot 10^3 }{ 0.1284 \cdot 10^6 } =
            0.0553
        \]

        \item Проверка КПД на окружности колеса:
        \[
            \eta_u = 1 - \zeta_с^\prime - \zeta_л - \zeta_{вых} = 1 - 0.0285 -
                    0.0353 - 0.0553 = 0.8809
        \]

        \item Средний диаметр:
        \[
            D_{ср} = 0.5 \cdot (D_1 + D_2) =
                    0.5 \cdot (0.5122 + 0.5268) =
            0.5195\ м
        \]

        \item Определим удельные потери в радиальном зазоре:

	    \begin{gather*}
	        h_з = 1.37 \cdot
                \left(
                    1 + 1.6 \rho
                \right)
                \left(
                    1 + \frac{l_2}{D_{ср}}
                \right)
            \frac{ \delta_r }{ l_2 } \cdot L_u =\\
	        = 1.37 \cdot
            \left(
                1 + 1.6 \cdot 0.5
            \right)
            \left(
                1 + \frac{ 0.1874 }{ 0.5195 }
            \right)
            \frac{ 0.00187 }{ 0.1874 } \cdot
            0.1131 \cdot 10^6 =
	        3.7813 \cdot 10^3 \/\ Дж/кг\\
	    \end{gather*}

        \item Относительные удельные потери в радиальном зазоре:
        \[
            \zeta_з = \frac{ h_з }{ H_0 } =
                \frac{ 3.7813 \cdot 10^3 }{ 0.1284 \cdot 10^6 } =
            0.0295
        \]

        \item Удельная работа ступени с учетом потери в радиальном зазоре:
        \[
            L_{uз} = L_u - h_з = 0.1131 \cdot 10^6 -
                3.7813 \cdot 10^3 =
            0.1093 \cdot 10^6 \ Дж/кг
        \]

        \item Мощностной КПД ступени:
        \[
            \eta_т^\prime = \eta_u - \zeta_з =
                0.8809 - 0.0295 = 0.8514
        \]

        \item Лопаточный КПД ступени:
        \[
            \eta_л^\prime = \eta_т^\prime + \zeta_{вых} =
                 0.8514 +  0.0553 =
            0.9067
        \]

        \item Средняя длина лопатки:
        \[
            l_{ср} = 0.5 \cdot (l_1 + l_2) =
                0.5 \cdot (0.1691 + 0.1874) =
            0.1783\ м
        \]

        \item Средняя окружная скоротсь:
        \[
            u_{ср} = 0.5 \cdot (u_1 + u_2) =
                0.5 \cdot (402.32 + 413.75) =
            408.03\ м/с
        \]

        \item Затраты мощности на трение и вентиляцию:
        \begin{gather*}
            N_{т.в} = \left[
                    1.07 \cdot D_{av}^2 + 61 \cdot (1 - \varepsilon) \cdot D_{av} l_{av}
            \right] \cdot
            \left(
                \frac{ u_{av} }{ 100 }
            \right) ^ 3 \cdot
            \rho =\\
            = \left[
                1.07 \cdot 0.5195^2 +
                61 \cdot (1 - 1) \cdot
                0.5195 \cdot 0.1783
            \right] \cdot
            \left(
                \frac{ 408.03 }{ 100 }
            \right) ^ 3 \cdot
            0.4962=\\
            = 0.0097 \cdot 10^3 \ Вт \\
        \end{gather*}

        \item Удельные потери на трение и вентиляцию:
        \[
            h_{т.в} = \frac{ N_{т.в} }{ G_{вх} } =
                \frac{
                    0.0097 \cdot 10^3
                }{
                    25.1
                }
            = 0.0004 \cdot 10^3 \ Дж/кг
        \]

        \item Относительные потери на трение и вентиляцию:
        \[
            \zeta_{т.в} = \frac{ h_{т.в} }{ H_0 } =
                \frac{ 0.0004 \cdot 10^3 }{ 0.1284 \cdot 10^6 } =
            0.0
        \]

        \item Мощностной КПД с учетом потерь на трению и вентиляцию:
        \[
            \eta_т = \eta_т^\prime - \zeta_{т.в} =
                0.8514 - 0.0 =
            0.8514
        \]

        \item Лопаточный КПД с учетом потерь на трению и вентиляцию:
        \[
            \eta_л = \eta_л^\prime - \zeta_{т.в} =
                0.9067 - 0.0 =
            0.9067
        \]

        

        \item Удельная работа ступени, отнесенная к расходу на в СА первой ступени:
        \[
            L_т^\prime = L_т \frac{ G_{вх} }{ G_т }  =
                0.1094 \cdot 10^6 \cdot
                \frac{ 25.1 }{ 25 } =
            0.1098 \cdot 10^6 \ Дж/кг
        \]

        \item Статическая температура за ступенью:
        \[
            T_{ст} = T_2 + \frac{ h_з }{ c_{pг} } + \frac{ h_{т.в} }{ c_{pг} } =
                1189.12 +
                \frac{3.7813 \cdot 10^3 }{ 1240.59 } +
                \frac{ 0.0004 \cdot 10^3 }{ 1240.59 } =
            1192.17 \ К
        \]

        \item Температура торможения за турбиной:
        \[
            T_{ст}^* = T_{ст} + \frac{ h_{вых} }{ c_{pг} } =
                1192.17 +
                \frac{ 7.0919 \cdot 10^3 }{ 1240.59 } =
            1197.88 \ К
        \]

        \item Средняя теплоемкость газа в интервале температур от 273 К до $T_0^*$:
        \[
            c_{pг\ ср} (T_0^*, \alpha_{вх}) =
            1127.75 \ Дж/(кг \cdot К)
        \]

        \item Средняя теплоемкость газа в интервале температур от 273 К до $T_{ст}$:
        \[
            c_{pг\ ср} (T_{ст}, \alpha_{вх}) =
            1116.14 \ Дж/(кг \cdot К)
        \]

        \item Средняя теплоемкость газа в интервале температур от $T_0^*$ до $T_{ст}$:
        \begin{gather*}
            c_{pг}^\prime = \frac{
		        c_{pг\ ср} (T_0^*, \alpha_{вх}) (T_0^* - T_0) - c_{pг\ ср} (T_{ст}, \alpha_{вх})(T_{ст} - T_0)
		    }{
		        T_0^* - T_{ст}} =\\
            =\frac{
		        1127.75 \cdot
                (1285.98 - 273) -
		        1116.14 \cdot
                (1192.17 - 273)
		    }{
		        1285.98 - 1192.17} =
		    1241.58 \ Дж / (кг \cdot К)\\
        \end{gather*}

        \item Новое значение показателя адиабаты:
        \[
            k_г^\prime = \frac{c_{pг}^\prime}{c_{pг}^\prime - R_г} =
                \frac{
                    1241.58
                }{
                    1241.58 - 287.4
                }
            = 1.3012
        \]

        \item Невязка по коэффициенту адиабаты:
        \[
            \delta = \frac{ \left| k_г - k_г^\prime \right| }{ k_г } \cdot 100 \%=
                \frac{
                    \left| 1.3015 - 1.3012 \right|
                }{
                    1.3015
                } \cdot 100 \% =
            0.0242 \%
        \]

        \item Давление торможения на выходе из ступени:
        \[
            p_2^* = p_2 \left(
                            \frac{ T_{ст}^* }{ T_{ст} }
                    \right) ^ \frac{ k_г }{ k_г - 1 } =
                 0.2554 \cdot 10^6 \cdot \left(
                            \frac{ 1197.88 }{ 1192.17 }
                    \right) ^
                \frac{ 1.3015 }{ 1.3015 - 1 } =
            0.2608 \cdot 10^6 \ Па
        \]

        \item Теплоперепад по параметрам торможения:
        \begin{gather*}
            H_0^* = c_{pг} T_0^* \left[
                        1 - \left(
                                \frac{p_2^*}{p_0^*}
                            \right) ^
                        \frac{k_г - 1}{k_г}
                    \right] =\\
            1240.59 \cdot 1285.98 \cdot
                    \left[
                        1 - \left(
                                \frac{
                                    0.2608 \cdot 10^6
                                }{
                                    0.3669 \cdot 10^6
                                }
                            \right) ^
                        \frac{1.3015 - 1}{1.3015}
                    \right]
            = 0.1213 \cdot 10^6 \ Дж/кг\\
        \end{gather*}

        \item КПД по параметрам торможения:
        \[
            \eta_т^* = \frac{ L_т }{ H_0^* } =
                \frac{
                    0.1094 \cdot 10^6
                }{
                    0.1213 \cdot 10^6 } =
            0.9014
        \]

        \item Расход на выходе из ступени:
        \[
            G_{вых} = G_{вх} + G_т g_{охл} =
                25.1 + 25 \cdot
                0.003 =
            25.18 \ кг/с
        \]

        \item Относительный расход топлива на выходе из ступени:
        \[
            g_{топл.вых} = \frac{ G_{топл} }{ G_{вых} - G_{топл} } =
                 \frac{ 1 }{ 25.18 - 1 } =
            0.0414
        \]

        \item Коэффициент избытка воздуха на выходе из ступени:
        \[
            \alpha_{вых} = \frac{ 1 }{ l_0 g_{топл.вых} } =
                \frac{ 1 }{ 14.61 \cdot 0.0414 } =
            1.655
        \]

        \item Абсолютный расход охлаждающего воздуха:
        \[
            G_{охл} = G_т g_{охл} = 25 \cdot 0.003 =
            0.075
        \]

        \item Определим температуру торможения на выходе из турбины после подмешивания охлаждающего воздуха.
        \begin{enumerate}

            \item Истинная теплоемкость охлаждающего воздуха при температуре $T_{охл} = 700\ К $:
            \[
                c_{pв} (T_{охл}) = 1074.93\ Дж/ (кг \cdot К)
            \]

            \item Истинная теплоемкость газа при температуре $T_{ст}^* = 1197.88 \ К $:
            \[
                c_{pг} (T_{ст}^*, \alpha_{вх}) =
                1263.39\ Дж/ (кг \cdot К)
            \]

            \item Значение температуры смеси с предпоследней итерации $T_{см}^{*} = 1196.34\ К$.

            \item Истинная теплоемкость смеси:
            \[
                c_{pг} (T_{см}^{*}, \alpha_{вых}) =
                1262.83\ Дж/ (кг \cdot К)
            \]

            \item Новое значение температуры смеси:
            \begin{gather*}
                T_{см}^*\prime = \frac{
                        c_{pг} (T_{ст}^*, \alpha_{вх}) T_{ст}^* G_{вх} + c_{pв} (T_{охл}) T_{охл} G_{охл}
                    }{
                        c_{pг} (T_{см}^{*}, \alpha_{вых}) G_{вых}
                    } =\\
                = \frac{
                    1263.39
                    \cdot 1197.88 \cdot 25.1 +
                    1074.93
                    \cdot 700 \cdot 0.075
                }{
                    1262.83
                    \cdot  25.18
                } =
                1196.62\ К\\
            \end{gather*}

            \item Значение невязки:
            \[
                \delta = \frac{ \left| T_{см}^{*} - T_{см}^*\prime \right| }{T_{см}^{*}} \cdot 100 \% =
                    \frac{
                        \left| 1196.34 - 1196.62 \right|
                    }{
                        1196.34
                    } \cdot 100 \% =
                0.023 \%
            \]

        \end{enumerate}

        

        \item Невязка по мощностному КПД:
        \[
            \delta_\eta = \frac{ \left| \eta_т - \eta_{т0} \right| }{ \eta_{т0} } \cdot 100 \% =
                \frac{
                    \left| 0.8514 - 0.852 \right|
                }{
                    0.852 } \cdot 100 \% =
            0.068 \%
        \]

    \end{enumerate}
     

    \subsection{Расчет интегральных параметров турбины}

    

    
    \begin{enumerate}

        \item Суммарная работа всех ступеней:
        \[
            L_{т\Sigma} = 0.1413\cdot 10^6+0.1098\cdot 10^6 = 0.2511 \cdot 10^6 \ Дж/кг
        \]

        \item Средняя теплоемкость газа в интервале температур от 273 К до $T_г^*$:
        \[
            c_{pг\ ср} (T_г^*, \alpha_{вх}) =
            1142.12 \ Дж/(кг \cdot К)
        \]

        \item Средняя теплоемкость газа в интервале температур от 273 К до $T_т$:
        \[
            c_{pг\ ср} (T_т, \alpha_{вх}) =
            1116.35 \ Дж/(кг \cdot К)
        \]

        \item Средняя теплоемкость газа в интервале температур от $T_0^*$ до $T_т$:
        \begin{gather*}
            c_{pг} = \frac{
		         c_{pг\ ср} (T_г^*, \alpha_{вх}) (T_г^* - T_0) - c_{pг\ ср} (T_{т}, \alpha_{вх})(T_т - T_0)
		    }{
		        T_г^* - T_т} =\\
            =\frac{
                1142.12 \cdot
                (1400 - 273) -
		        1116.35 \cdot
                (1192.17 - 273)
		    }{
		        1400 - 1192.17} =
		    1256.07 \ Дж / (кг \cdot К)\\
        \end{gather*}

        \item Средний показателя адиабаты:
        \[
            k_г = \frac{c_{pг}}{c_{pг} - R_г} =
                \frac{
                    1256.07
                }{
                    1256.07 - 287.4
                }
            = 1.2967
        \]

        \item Статический теплоперепад на турбине:
        \begin{gather*}
            H_т = c_{pг} T_г^* \left[
                        1 - \left(
                                \frac{p_г^*}{p_т} ^
                                \frac{1 - k_г}{k_г}
                    \right)
                \right] =\\
            = 1256.07 \cdot 1400
                \left[
                    1 - \left(
                            \frac{
                                0.55 \cdot 10^6
                            }{
                                0.2554 \cdot 10^6 } ^
                            \frac{ 1 - 1.2967 }{ 1.2967 }
                    \right)
            \right] =
            0.283 \cdot 10^6 \ Дж/кг\\
        \end{gather*}

        \item Средняя теплоемкость газа в интервале температур от 273 К до $T_т^*$:
        \[
            c_{pг\ ср} (T_т^*, \alpha_{вх}) =
            1116.87 \ Дж/(кг \cdot К)
        \]

        \item Средняя теплоемкость газа в интервале температур от $T_0^*$ до $T_т^*$:
        \begin{gather*}
            c_{pг}^* = \frac{
		         c_{pг\ ср} (T_г^*, \alpha_{вх}) (T_г^* - T_0) - c_{pг\ ср} (T_т^*, \alpha_{вх})(T_т^* - T_0)
		    }{
		        T_г^* - T_т^*} =\\
            =\frac{
                1142.12 \cdot
                (1400 - 273) -
		        1116.87 \cdot
                (1196.34 - 273)
		    }{
		        1400 - 1196.34} =\\
		     = 1256.59 \ Дж / (кг \cdot К)\\
        \end{gather*}

        \item Средний показателя адиабаты по параметрам торможения:
        \[
            k_г^* = \frac{ c_{pг}^* }{ c_{pг}^* - R_г } =
                \frac{
                    1256.59
                }{
                    1256.59 - 287.4
                }
            = 1.2965
        \]

        \item Теплоерепад на турбине оп параметрам торможения:
        \begin{gather*}
            H_т^* = c_{pг}^* T_г^* \left[
                        1 - \left(
                                \frac{p_г^*}{p_т} ^
                                \frac{1 - k_г^*}{k_г^*}
                    \right)
                \right] =\\
            =1256.59 \cdot 1400
                \left[
                    1 - \left(
                            \frac{
                                0.55 \cdot 10^6
                            }{
                                0.2554 \cdot 10^6 } ^
                            \frac{ 1 - 1.2965 }{ 1.2965 }
                    \right)
            \right] =
            0.276 \cdot 10^6 \ Дж/кг\\
        \end{gather*}

        \item Мощностной КПД турбины:
        \[
            \eta_т = \frac{ L_{т\Sigma} }{ H_т } =
                \frac{ 0.2511 \cdot 10^6 }{ 0.283 \cdot 10^6 } =
            0.8872
        \]

        \item Лопаточный КПД турбины:
        \[
            \eta_л = \frac{
                        L_{т\Sigma} + 0.5 \cdot c_{вых}^2
                    }{ H_т } =
            \frac{
                0.2511 \cdot 10^6 + 0.5 \cdot 119.1 ^ 2
            }{ 0.283 \cdot 10^6 } =
            0.9122
        \]

        \item КПД турбины по параметрам торможения:
        \[
            \eta_т^* = \frac{ L_{т\Sigma} }{ H_т^* } =
                \frac{ 0.2511 \cdot 10^6 }{ 0.276 \cdot 10^6 } =
            0.9096
        \]

    \end{enumerate}
    

    \section{Расчет параметров потока по высоте.}

    

    При расчете параметров потока по высоте лопаточного венца будем пользоваться описанной ниже методикой.

    \begin{enumerate}

        \item Допущения:

        \begin{enumerate}

            \item Постоянство температуры торможения на входе в СА: $T_0^*(r) = const$.
            \item Постояноство скорости на входе: $c_0(r) = const$.
            \item Постоянство угла потока на входе: $\alpha_0(r) = const$.
            \item Постоянство полного давления на входе: $p_0^*(r) = const$.
            \item Постоянство работы на окружности колеса: $L_u(r) = const$.
            \item Потерь в лопаточных венцах нет.
            \item Ступень цилиндрическая.

        \end{enumerate}

%        

        \item Статическая температура на входе в СА:
        \[
            T_0(r) = T_0^* - \frac{c_0 ^ 2}{2 c_p}
        \]

        \item Окружная скорость на выходе из СА:
%        

        \item Осевая скорость на выходе из СА:
%        

        \item Скорость в абсолютном движении на выходе из СА:
        \[
            c_1(r) = \sqrt{c_{1a}(r)^2 + c_{1u}(r)^2}
        \]

        \item Угол потока в абсолютном движении на выходе из СА:
        \[
            \alpha_1 (r) = \arcsin{\frac{c_{1a}(r)}{c_1(r)}}
        \]

        \item Теплоперепад на СА:
        \[
            H_с (r) = \frac{c_1(r) ^ 2}{2}
        \]

        \item Статическое давление на выходе из СА:
        \[
            p_1 = p_0^*(r) \left( 1 - \frac{H_с (r)}{T_0^* c_p} \right) ^ {\frac{k}{k - 1}}
        \]

        \item Статическая температура на выходе из СА:
        \[
            T_1 (r) = T_0^*(r) - \frac{H_с (r)}{c_p}
        \]

        \item Окружная скорость:
        \[
            u(r) = \frac{2 \pi n r}{60}
        \]

        \item Температура торможения на выходе из РК:
        \[
            T_2^* = T_0^*(r) - \frac{L_u (r)}{c_p}
        \]

        \item Окружная скорость на выходе из РК:
        \[
            c_{2u}(r) = \frac{L_u(r) - c_{1u} u(r)}{u(r)}
        \]

        \item Осевая скорость на выходе из РК:
        \[
            c_{2a}(r) = \sqrt{
                    c_{2a\ ср}^2 + с_{2u\ ср}^2 - c_{2u}(r)^2 -
                    2 \cdot \int_{r_{ср}}^{r} \frac{c_{2u}(r)^2}{r} dr
            }
        \]

        \item Скорость в абсолютном движении на выходе из РК:
        \[
            c_2(r) = \sqrt{c_{2a}(r)^2 + c_{2u}(r)^2}
        \]

        \item Угол потока в абсолютном движении на выходе и РК:
        \[
            \alpha_2 (r) = \arctan{\frac{c_{2a}(r)}{c_{2u}(r)}}
        \]

        \item Окружная составляющая относительной скорости на выходе из РК:
        \[
            w_{2u}(r) = c_{2u}(r) + u(r)
        \]

        \item Относитеьная скорость на выходе из РК:
        \[
            w_2 (r) = \sqrt{w_{2u}(r)^2 + c_{2a}(r)^2}
        \]

        \item Относительная скорость на выходе из СА:
        \[
            w_1 (r) = \sqrt{c_1(r)^2 + u(r)^2 - 2 \cdot u(r) c_1(r) \cos{\alpha_1(r)}}
        \]

        \item Температура торможения в относительном движении на выходе из РК:
        \[
            T_{1w}^* = T_1(r) + \frac{w_1(r) ^ 2}{2 c_p}
        \]

        \item Теплоперепад в РК:
        \[
            H_л (r) = 0.5 \cdot (w_2(r)^2 - w_1(r)^2)
        \]

        \item Статическое давление на выходе из РК:
        \[
            p_2 (r) = p_1(r) \cdot \left( 1 - \frac{H_л}{c_p T_1(r)} \right) ^ {\frac{k}{k - 1}}
        \]

        \item Статическая температура на выходе из РК:
        \[
            T_2 (r) = T_1 (r) - \frac{w_2 (r)^2 - w_1(r)^2}{2 c_p}
        \]

        \item Статический теплоперепад на ступени:
        \[
            H_0 (r) = c_p \cdot T_0^*(r) \cdot \left( 1 - \frac{p_0^*(r)}{p_2(r)} \right) ^ {\frac{1 - k}{k}}
        \]

        \item Степень реактивности:
        \[
            \rho (r) = \frac{H_л (r)}{H_0 (r)}
        \]

    \end{enumerate}
%        

    
    \begin{longtable}{
    |
%    
    c|
%    
    c|
%    
    c|
%    
    c|
%    
    }
        \caption{Параметры первой ступени на различных радиусах.}\\
        \hline

%        
        $\frac{r - r_{вт}}{r_{п} - r_{вт}}$
%        
        & 0.0
%        
        & 0.5
%        
        & 1.0
%        
        \\
        \hline
%        
        $r,\ мм$
%        
        & 180.6
%        
        & 240.8
%        
        & 301.0
%        
        \\
        \hline
%        
        $\rho$
%        
        & -0.02
%        
        & 0.398
%        
        & 0.6
%        
        \\
        \hline
%        
        $c_1,\ м/с$
%        
        & 559.9
%        
        & 429.8
%        
        & 350.1
%        
        \\
        \hline
%        
        $c_{1a},\ м/с$
%        
        & 159.4
%        
        & 122.3
%        
        & 99.7
%        
        \\
        \hline
%        
        $c_{1u},\ м/с$
%        
        & 536.7
%        
        & 412.0
%        
        & 335.6
%        
        \\
        \hline
%        
        $\alpha_1,\ ^\circ$
%        
        & 16.5
%        
        & 16.5
%        
        & 16.5
%        
        \\
        \hline
%        
        $w_1,\ м/с$
%        
        & 299.1
%        
        & 126.9
%        
        & 169.5
%        
        \\
        \hline
%        
        $w_{1a},\ м/с$
%        
        & 159.4
%        
        & 122.3
%        
        & 99.7
%        
        \\
        \hline
%        
        $w_{1u},\ м/с$
%        
        & 253.1
%        
        & 33.8
%        
        & -137.1
%        
        \\
        \hline
%        
        $\beta_1,\ ^\circ$
%        
        & 32.2
%        
        & 74.5
%        
        & 144.0
%        
        \\
        \hline
%        
        $u,\ м/с$
%        
        & 283.7
%        
        & 378.2
%        
        & 472.8
%        
        \\
        \hline
%        
        $M_{c0}$
%        
        & 0.139
%        
        & 0.139
%        
        & 0.139
%        
        \\
        \hline
%        
        $M_{c1}$
%        
        & 0.813
%        
        & 0.612
%        
        & 0.494
%        
        \\
        \hline
%        
        $M_{w1}$
%        
        & 0.434
%        
        & 0.181
%        
        & 0.239
%        
        \\
        \hline
%        
        $T_1,\ К$
%        
        & 1276.2
%        
        & 1327.0
%        
        & 1351.6
%        
        \\
        \hline
%        
        $p_1,\ МПа$
%        
        & 0.3658
%        
        & 0.4345
%        
        & 0.471
%        
        \\
        \hline
%        
        $T_{1w}^*,\ К$
%        
        & 1311.5
%        
        & 1333.4
%        
        & 1362.9
%        
        \\
        \hline
%        
        $c_2,\ м/с$
%        
        & 128.5
%        
        & 127.0
%        
        & 125.6
%        
        \\
        \hline
%        
        $c_{2a},\ м/с$
%        
        & 126.3
%        
        & 124.0
%        
        & 122.5
%        
        \\
        \hline
%        
        $c_{2u},\ м/с$
%        
        & -24.0
%        
        & -27.5
%        
        & -28.0
%        
        \\
        \hline
%        
        $\alpha_2,\ ^\circ$
%        
        & 100.8
%        
        & 102.5
%        
        & 102.9
%        
        \\
        \hline
%        
        $w_2,\ м/с$
%        
        & 288.7
%        
        & 372.0
%        
        & 461.3
%        
        \\
        \hline
%        
        $w_{2a},\ м/с$
%        
        & 126.3
%        
        & 124.0
%        
        & 122.5
%        
        \\
        \hline
%        
        $w_{2u},\ м/с$
%        
        & 259.7
%        
        & 350.7
%        
        & 444.8
%        
        \\
        \hline
%        
        $\beta_2,\ ^\circ$
%        
        & 25.9
%        
        & 19.5
%        
        & 15.4
%        
        \\
        \hline
%        
        $M_{w2}$
%        
        & 0.419
%        
        & 0.539
%        
        & 0.669
%        
        \\
        \hline
%        
        $T_2,\ К$
%        
        & 1278.6
%        
        & 1278.7
%        
        & 1278.9
%        
        \\
        \hline
%        
        $p_2,\ МПа$
%        
        & 0.3688
%        
        & 0.369
%        
        & 0.3692
%        
        \\
        \hline
%        
        $p_2^*,\ МПа$
%        
        & 0.3772
%        
        & 0.3772
%        
        & 0.3772
%        
        \\
        \hline
%        
        $\pi_т$
%        
        & 1.491
%        
        & 1.49
%        
        & 1.49
%        
        \\
        \hline
%        
        $\pi_т^*$
%        
        & 1.458
%        
        & 1.458
%        
        & 1.458
%        
        \\
        \hline
%        
        $H_л,\ \frac{кДж}{кг}$
%        
        & -3.0
%        
        & 61.1
%        
        & 92.0
%        
        \\
        \hline
%        
        $H_0,\ \frac{кДж}{кг}$
%        
        & 153.7
%        
        & 153.5
%        
        & 153.3
%        
        \\
        \hline
%        

    \end{longtable}

%        

    \section{Расчет охлаждения СА первой ступени.}

    \subsection{Алгоритм расчета пленки.}

    

    \begin{enumerate}

        \item Исходные данные и допущения для алгоритма расчета пленки по одной стороне лопатки.

        \begin{enumerate}

            \item Параметры газа по высоте постоянны: $T_г^* = const$ и $p_г^* = const$.
            \item Координаты отверстий: $x_{отв\ i}$.
            \item Диаметры отверстий: $d_{отв\ i}$.
            \item Числа отверстий в рядах: $N_{отв\ i}$.
            \item Коэффициенты скорости в отверстиях: $\phi_{отв\ i}$.
            \item Коэффициенты расхода в отверстиях: $\mu_{отв\ i}$.
            \item Давление торможения воздуха на входе в канал: $p_{в0}^*$.
            \item Расход охраждающего воздуха на входе в канал: $G_{в0}$.
            \item Распределение коэффициента теплоотдачи со стороны газа вдоль профиля: $\alpha_г (x)$.
            \item Распределение температуры охлаждающего воздуха вдоль профиля: $T_в^* (x)$.
            \item Высота участка лопатки: $l$.

        \end{enumerate}

        \item Зададим распределение приведенной скорости по корыту $\lambda_к \left( \overline{x} \right)$ и
        спинке $\lambda_с \left( \overline{x} \right)$:

		\begin{gather*}
		    \lambda_к \left( \overline{x} \right) =
			\left\{
				1 +
				\left[
					\left(
						\frac{\lambda_1}{\lambda_0}
					\right)^{0.5}
				\right]\overline{x}
			\right\}^{2} \lambda_0, \/\ \overline{x} = \frac{x}{l_к}\\
		    \lambda_с \left( \overline{x} \right) =
			\left\{
				1 +
				\left[
					\left(
						\frac{\lambda_1}{\lambda_0}
					\right)^{4}
				\right]\overline{x}
			\right\}^{0.25}\lambda_0, \/\ \overline{x} = \frac{x}{l_с}
		,\\
		\end{gather*}
		где $l_к$ - длина профиля со стороны корыта, $l_с$ - длина профиля со стороны спинки,
        $\lambda_0$ - приведенная скорость на входе в лопаточный венец, $\lambda_1$ - приведенная
        скорость на выходе из лопаточного венца.

        \item Определим критическую скорость звука $a_{кр}$:
		\[
			a_{кр} = \sqrt{
				\frac{2k_г}{k_г + 1} R_г T_г^*
			}
		\]

        \item Определим скорость газа на корыте $v_к$ и на спинке $v_с$:
            \begin{gather*}
                v_к\left( x \right) = \lambda_к \left( \frac{x}{l_к} \right)\\
                v_c\left( x \right) = \lambda_к \left( \frac{x}{l_c} \right)\\
            \end{gather*}

        Дальнейший расчет идентичен для спинки и корыта, поэтому скорость газа будем обозначать как $v_г$.
        \item Определим эквивалентную ширину щели:
            \[
                s = N_{отв} \frac{\pi d_{отв}^2}{4} \cdot \frac{1}{l},
            \]
            где $N_{отв}$ - количество отверстий, $d_{отв}$ - диаметр отверстия, $l$ - высота профильной части лопатки.

        \item Определим скорость газа в точке выдува воздуха:
            \[
                v_{г \/\ отв} = v_г\left( x_{отв} \right),
            \]
            где $x_{отв}$ - криволинейная координата отверстия.

        \item Определим статическую температуру газа в точке выдува воздуха:
            \[
                T_{г \/\ отв} = T_г^* - \frac{v_{г \/\ отв}}{2 c_{p \/\ г}}
            \]

        \item Определим статическое давление газа в точке выдува воздуха:
            \[
                p_{г \/\ отв} = \frac{p_г^*}{
                    \left(
                        \frac{
                            T_г^*
                        }{
                            T_{г \/\ отв}
                        }
                    \right)^\frac{k_г}{k_г - 1}
                }
            \]

        \item Определим статическую плотность газа в точке выдува воздуха:
            \[
                \rho_{г \/\ отв} = \frac{
                    p_{г \/\ отв}
                }{
                    R_г \cdot T_{г \/\ отв}
                }
            \]

        \item Определим скорость истечения воздуха из отверстия:
            \[
                v_{в \/\ отв} = \phi_{отв} \sqrt{
                    \frac{2k_в}{k_в - 1}
                } R_в T_в^* \left( x_{отв} \right)
                \left[
                    1 -
                    \left(
                        \frac{
                            p_{г \/\ отв}
                        }{
                            p_{в0}^*
                        }
                    \right)^\frac{k_в - 1}{k_в}
                \right],
            \]
            где $\phi_{отв}$ - коэффициент скорости, $T_в^* \left( x_{отв} \right)$ -
            температура воздуха в точке выдува, $p_{в0}^*$ - давление воздуха.

        \item Определим статическую плотность воздуха на выходе из отверстия:
		\[
			\rho_{в \/\ отв} = \frac{
				p_{г \/\ отв}
			}{
				R_в
				\left[
					T_в^* \left( x_{отв} \right) - \frac{v_{в \/\ отв}^2}{2c_{p \/\ в}}
				\right]
			}
		\]

        \item Определим плотность торможения воздуха на входе в отверстия:
            \[
                \rho_{в \/\ отв}^* = \frac{p_{в0}^*}{R_в T_в^* \left( x_{отв} \right) }
            \]

        \item Определим параметр вдува:
            \[
                m = \frac{\rho_{в \/\ отв} v_{в \/\ отв}}{\rho_{г \/\ отв} v_{г \/\ отв}}
            \]

        \item Определим число Рейнольдса по ширине щели:
            \[
                Re_s = \frac{
                    \rho_{г \/\ отв} v_{г \/\ отв} s
                }{\mu_г\left( T_{г \/\ отв} \right)}
            \]

        \item Определим температурный фактор:
            \[
                \phi = T_в^* \left( x_{отв} \right) / T_г^*
            \]

        \item Определим эффективность пленки $\theta_{пл}\left( x \right)$:
            \[
                A\left( x \right) = Re_s^{-0.25} m^{-1.3} \phi^{-1.25}
                \left(
                    \frac{
                        x - x_{отв}
                    }{
                        s
                    }
                \right)
            \]
            \[
                \theta_{пл}\left( x \right) = \left\{
                    \begin{array}{@{}ll@{}}
                        1.0, & \text{если }\ 0 < A \leq 3 \\
                        \left( \frac{A}{3} \right)^{-0.285}, & \text{если } 3 \leq A < 11 \\
                        \left( \frac{A}{7.43} \right)^{-0.95}, & \text{если } A \geq 11 \\
                    \end{array}\right.
            \]

        \item Определим темперутуру пленки в случае нескольких рядов отверстий:
		\[
			T_{пл}^*\left( x \right) = \left\{
                \begin{array}{lc}
                    T_г^*, & \text{если}\ x < x_{отв\ 1} \\
                    T_г^* \cdot \prod_{i = 1}^{x_i \leq x}
                    \left[
                        \left(
                            1 - \theta_{пл \/\ i}
                        \right)
                    \right] + &\\
                    +\sum_{i = 1}^{x_i \leq x} \left[
                        \theta_{пл \/\ i}T_в^*\left( x_{отв \/\ j} \right)
                        \prod_{j = i + 1}^{x_j \leq x}
                        \left(
                            1 - \theta_{пл \/\ j}
                        \right)
                    \right], & \text{если}\ x_{отв\ 1} \leq x \\
                \end{array} \right.
		\]

        \item Определим коэффициент теплоотдачи пленки в случае нескольких рядов отверстий:
		\[
			\alpha_{пл}\left( x \right) = \left\{
                \begin{array}{lc}
                    \alpha_{г}(x), & \text{если }  0 \leq x < x_{отв\ 1}  \\
                    \alpha_{г}(x) \left(
                        1 + \frac{
                            2m_1
                        }{
                        \frac{
                            x - x_{отв \/\ 1}
                        }{s_1}
                    }
                    \right), & \text{если }  x_{отв\ 1} \leq x < x_{отв\ 2}  \\
                    .........\\
                    \alpha_{г}(x) \left(
                        1 + \frac{
                            2m_n
                        }{
                        \frac{
                            x - x_{отв \/\ n}
                        }{s_n}
                    }
                    \right), & \text{если }  x_{отв\ n} \leq x   \\
                \end{array}\right.
		\]

        \item По формуле истечения из сопла определим расход через ряд отверстий:
		\[
			G_{отв} = s \cdot l \cdot  \mu_{отв} \sqrt{
				\frac{2k_в}{k_в - 1} p_{в0}^*\rho_{в \/\ отв}^*
				\left(
					\frac{
						p_{г \/\ отв}
					}{
						p_{в0}^*
					}
				\right)^\frac{2}{k_в}
				\left[
					1 -
					\left(
						\frac{
							p_{г \/\ отв}
						}{
							p_{в0}^*
						}
					\right)^\frac{k_в - 1}{k_в}
				\right]
			}
		\]

        \item В общем случае зависимость расхода воздуха в зазоре от криволинейной координаты имеет вид:
		\[
			G_в \left( x \right) = G_{в0} - \sum_{i = 1}^{x_i \leq x} G_{отв \/\ i}
		\]

        \item Выходные данные расчета пленки:

        \begin{enumerate}

            \item Распределение вдоль профиля коэффициента теплоотдачи со стороны пленки: $\alpha_{пл} (x)$.
            \item Распределение температуры пленки вдоль профиля: $T_{пл} (x)$.
            \item Распределение расхода охлаждающего воздуха вдоль профиля: $G_в (x)$.

        \end{enumerate}

    \end{enumerate}

  %  

    \subsection{Алгоритм расчета локальных параметров.}

    

    \begin{enumerate}

        \item Исходные данные и допущения для расчета локальных параметров с одной стороны лопатки:

        \begin{enumerate}
            \item Теплопроводность материала лопатки: $\lambda_м$.
            \item Толщина стенки лопатки: $\Delta$.
            \item Теплопроводность защитного покрытия: $\lambda_п$.
            \item Толщина покрытия: $\Delta_п$.
            \item Распределение вдоль профиля коэффициента теплоотдачи со стороны пленки: $\alpha_{пл} (x)$.
            \item Распределение температуры пленки вдоль профиля: $T_{пл} (x)$.
            \item Распределение расхода охлаждающего воздуха вдоль профиля: $G_в (x)$.
            \item Ширина канала для охлаждения лопатки: $\delta$.
            \item Высота участка лопатки: $l$.
            \item Теплопроводность воздуха в зависимости от температуры: $\lambda_в (T)$.
            \item Вязкость воздуха в зависимости от температуры: $\mu_в (T)$.
        \end{enumerate}

        \item Определим зависимость коэффициента теплотдачи со строны воздуха от температуры и координаты вдоль профиля:
        \[
            \alpha_в(T_в^*, x) = 0.02 \cdot \lambda_в \left( T_в^* \right) \cdot \frac{1}{2 \delta} \cdot
                \left(
                \frac{G_в }{l \cdot \mu \left( T_в^* \right)}
            \right)
        \]

        \item Определим зависимость коэффициента теплопередачи от температуры и координаты вдоль профиля:
        \[
            k (T_в^*, x) = \frac{1
                    }{
                \frac{1}{\alpha_в(T_в^*, x)} + \frac{1}{\alpha_{пл} (x)} + \frac{\Delta}{\lambda_м} +
                \frac{\Delta_п}{\lambda_п}
            }
        \]

        \item Получаем зависимость производной температуры воздуха от координаты от температуры и координаты
        вдоль профиля:
        \[
            \frac{dT_в^*}{x} = \frac{
                    k (T_в^*, x) (T_пл(x) - T_в^*) l
            }{
                    G_в(x) c_{pв}
            }
        \]

        \item В итоге получаем дифференциальное уравнение вида $\frac{dy}{dx} = f(y, x)$, которое решаем любым из
        известных методов численного решения ДУ, например, методом Эйлера.

        \item Выходные результаты расчета локальных параметров:

        \begin{enumerate}
            \item Рапределение температуры воздуха вдоль профиля: $T_в^*(x)$.
            \item Распределение коэффициента теплоотдачи со стороны воздуха вдоль профиля: $\alpha_в (x)$.
        \end{enumerate}

    \end{enumerate}

%    

    \subsection{Алгоритм расчета коэффициент теплоотдачи со стороны газа.}

    
        \begin{enumerate}

            \item Исходные данные для расчета распределения коэффициента теплоотдачи со стороны газа:
            \begin{enumerate}
                \item Расход газа через участок решетки: $G_г$.
                \item Средний диаметр лопаточной решетки: $D_{ср}$.
                \item Высота участка лопатки: $l$.
                \item Длина хорды профиля: $b$.
                \item Длины спинки и корыта: $l_к$ и $l_с$.
                \item Длины участков входной кромки на спинке и корыте: $l_{вх\ c}$ и $l_{вх\ к}$.
                \item Углы профиля на входе и на выходе: $\alpha_{1л}$ и $\alpha_{2л}$.
                \item Радиус закругления входной кромки: $r_1$.
            \end{enumerate}

            \item Число Рейнольдса по газу:
            \[
                Re_г = \frac{G_г b }{\pi D_{ср} l \mu_г \sin{\alpha_{2л}}}
            \]

            \item Число Нуссельта:
            \[
                Nu_г = \left[ 0.07 + 100 \cdot ( \alpha_{1л} + \alpha_{2л} )^{-2} \right] \cdot Re_г ^{0.68}
            \]

            \item Средний коэффициент теплоотдачи со стороны газа:
            \[
                \alpha_{г\ ср} = \frac{Nu_г \lambda_г}{b}
            \]

            \item Коээфициент теплоотдачи со стороны газа на входной кромке:
            \[
                \alpha_{г\ вх} = \frac{0.74 \lambda_г}{2 r_1} \cdot
                    \left(
                        \frac{2 G_г r_1}{\pi D_{ср} l \sin{\alpha_{л}} \mu_г}
                \right) ^ {0.5}
            \]

            \item Распределение коэффициента теплоотдачи от газа по профилю:
            \[
                \alpha_г(x) = \left\{
                    \begin{array}{lc}
                        1.5 \alpha_{г\ ср}, & \text{если}\ -l_c \leq x < -l_с + \frac{b}{3} \\
                        0.6 \alpha_{г\ ср}, & \text{если}\ -l_с + \frac{b}{3} \leq x < -l_{вх\ c} \\
                        \alpha_{г\ вх}, & \text{если}\ -l_{вх\ c} \leq x < l_{вх\ к} \\
                        \alpha_{г\ ср}, & \text{если}\ l_{вх\ к} \leq x < l_к \\
                    \end{array}
                \right.
            \]

        \end{enumerate}
%    

    \subsection{Результаты расчета.}

    
    \begin{longtable}{
     |
%    
    c|
%    
    c|
%    
    c|
%    
    c|
%    
    c|
%    
    c|
%    
    c|
%    
    c|
%    
    c|
%    
    }
        \caption{Значения параметров в отверстиях.} \\
        \hline
%        

%        
        $x,\ мм$
%        

%        

%        
        & $s,\ 10^{-3}\ мм$
%        

%        

%        
        & $\phi_{отв}$
%        

%        

%        
        & $\mu_{отв}$
%        

%        

%        
        & $m$
%        

%        

%        
        & $\phi$
%        

%        

%        
        & $G_{отв},\ г/с$
%        

%        

%        
        & $G_{отв}/G_{в0}$
%        

%        
        \\
        \hline

%        

%        

%        
        -80.9
%        

%        

%        
        & 57.1
%        

%        

%        
        & 0.98
%        

%        

%        
        & 0.95
%        

%        

%        
        & 1.373
%        

%        

%        
        & 0.48
%        

%        

%        
        & 1.999
%        

%        

%        
        & 0.02
%        

%        
        \\
        \hline

        

%        

%        
        -40.4
%        

%        

%        
        & 57.1
%        

%        

%        
        & 0.98
%        

%        

%        
        & 0.95
%        

%        

%        
        & 1.363
%        

%        

%        
        & 0.473
%        

%        

%        
        & 1.736
%        

%        

%        
        & 0.017
%        

%        
        \\
        \hline

        

%        

%        
        0.0
%        

%        

%        
        & 28.5
%        

%        

%        
        & 0.98
%        

%        

%        
        & 0.95
%        

%        

%        
        & 0.63
%        

%        

%        
        & 0.464
%        

%        

%        
        & 0.142
%        

%        

%        
        & 0.001
%        

%        
        \\
        \hline

        

%        

%        
        0.0
%        

%        

%        
        & 28.5
%        

%        

%        
        & 0.98
%        

%        

%        
        & 0.95
%        

%        

%        
        & 0.63
%        

%        

%        
        & 0.464
%        

%        

%        
        & 0.142
%        

%        

%        
        & 0.001
%        

%        
        \\
        \hline

        

%        

%        
        30.2
%        

%        

%        
        & 57.1
%        

%        

%        
        & 0.98
%        

%        

%        
        & 0.95
%        

%        

%        
        & 1.129
%        

%        

%        
        & 0.474
%        

%        

%        
        & 0.744
%        

%        

%        
        & 0.007
%        

%        
        \\
        \hline

        

%        

%        
        90.7
%        

%        

%        
        & 57.1
%        

%        

%        
        & 0.98
%        

%        

%        
        & 0.95
%        

%        

%        
        & 1.321
%        

%        

%        
        & 0.49
%        

%        

%        
        & 1.521
%        

%        

%        
        & 0.015
%        

%        
        \\
        \hline

        
    \end{longtable}
%    

    
    \begin{longtable}{
     |
%    
    c|
%    
    c|
%    
    c|
%    
    c|
%    
    c|
%    
    c|
%    
    c|
%    
    c|
%    
    c|
%    
    c|
%    
    }
        \caption{Значения локальных параметров.} \\
        \hline
%        

%        
        $x,\ мм$
%        

%        

%        
        & $\alpha_{пл},\ \frac{Вт}{м^2 \cdot К}$
%        

%        

%        
        & $\alpha_{в},\ \frac{Вт}{м^2 \cdot К}$
%        

%        

%        
        & $\alpha_{г},\ \frac{Вт}{м^2 \cdot К}$
%        

%        

%        
        & $T_{в}^*,\ К$
%        

%        

%        
        & $T_{пл}^*,\ К$
%        

%        

%        
        & $T_{ст},\ К$
%        

%        

%        
        & $\theta_{пл}$
%        

%        

%        
        & $\theta_{охл}$
%        

%        
        \\
        \hline

%        

%        

%        
        -161.8
%        

%        

%        
        & 1079.5
%        

%        

%        
        & 1630.5
%        

%        

%        
        & 1077.4
%        

%        

%        
        & 706
%        

%        

%        
        & 1379.0
%        

%        

%        
        & 984
%        

%        

%        
        & 0.031
%        

%        

%        
        & 0.599
%        

%        
        \\
        \hline

        

%        

%        
        -153.7
%        

%        

%        
        & 1079.7
%        

%        

%        
        & 1627.8
%        

%        

%        
        & 1077.4
%        

%        

%        
        & 702
%        

%        

%        
        & 1377.0
%        

%        

%        
        & 981
%        

%        

%        
        & 0.034
%        

%        

%        
        & 0.599
%        

%        
        \\
        \hline

        

%        

%        
        -145.7
%        

%        

%        
        & 1080.0
%        

%        

%        
        & 1625.2
%        

%        

%        
        & 1077.4
%        

%        

%        
        & 698
%        

%        

%        
        & 1374.0
%        

%        

%        
        & 978
%        

%        

%        
        & 0.037
%        

%        

%        
        & 0.6
%        

%        
        \\
        \hline

        

%        

%        
        -137.7
%        

%        

%        
        & 1080.4
%        

%        

%        
        & 1622.5
%        

%        

%        
        & 1077.4
%        

%        

%        
        & 693
%        

%        

%        
        & 1371.0
%        

%        

%        
        & 975
%        

%        

%        
        & 0.041
%        

%        

%        
        & 0.601
%        

%        
        \\
        \hline

        

%        

%        
        -129.7
%        

%        

%        
        & 1080.9
%        

%        

%        
        & 1619.8
%        

%        

%        
        & 1077.4
%        

%        

%        
        & 689
%        

%        

%        
        & 1367.0
%        

%        

%        
        & 971
%        

%        

%        
        & 0.046
%        

%        

%        
        & 0.603
%        

%        
        \\
        \hline

        

%        

%        
        -121.6
%        

%        

%        
        & 1081.6
%        

%        

%        
        & 1617.1
%        

%        

%        
        & 1077.4
%        

%        

%        
        & 685
%        

%        

%        
        & 1362.0
%        

%        

%        
        & 967
%        

%        

%        
        & 0.053
%        

%        

%        
        & 0.606
%        

%        
        \\
        \hline

        

%        

%        
        -113.6
%        

%        

%        
        & 1082.6
%        

%        

%        
        & 1614.3
%        

%        

%        
        & 1077.4
%        

%        

%        
        & 681
%        

%        

%        
        & 1355.0
%        

%        

%        
        & 962
%        

%        

%        
        & 0.063
%        

%        

%        
        & 0.609
%        

%        
        \\
        \hline

        

%        

%        
        -105.6
%        

%        

%        
        & 433.7
%        

%        

%        
        & 1612.7
%        

%        

%        
        & 431.0
%        

%        

%        
        & 678
%        

%        

%        
        & 1344.0
%        

%        

%        
        & 827
%        

%        

%        
        & 0.078
%        

%        

%        
        & 0.794
%        

%        
        \\
        \hline

        

%        

%        
        -97.6
%        

%        

%        
        & 435.0
%        

%        

%        
        & 1611.3
%        

%        

%        
        & 431.0
%        

%        

%        
        & 676
%        

%        

%        
        & 1324.0
%        

%        

%        
        & 821
%        

%        

%        
        & 0.105
%        

%        

%        
        & 0.8
%        

%        
        \\
        \hline

        

%        

%        
        -89.6
%        

%        

%        
        & 438.8
%        

%        

%        
        & 1609.9
%        

%        

%        
        & 431.0
%        

%        

%        
        & 674
%        

%        

%        
        & 1271.0
%        

%        

%        
        & 809
%        

%        

%        
        & 0.178
%        

%        

%        
        & 0.815
%        

%        
        \\
        \hline

        

%        

%        
        -81.5
%        

%        

%        
        & 535.4
%        

%        

%        
        & 1608.9
%        

%        

%        
        & 431.0
%        

%        

%        
        & 673
%        

%        

%        
        & 728.0
%        

%        

%        
        & 685
%        

%        

%        
        & 0.923
%        

%        

%        
        & 0.983
%        

%        
        \\
        \hline

        

%        

%        
        -73.5
%        

%        

%        
        & 433.0
%        

%        

%        
        & 1663.1
%        

%        

%        
        & 431.0
%        

%        

%        
        & 671
%        

%        

%        
        & 1369.0
%        

%        

%        
        & 823
%        

%        

%        
        & 0.043
%        

%        

%        
        & 0.792
%        

%        
        \\
        \hline

        

%        

%        
        -65.5
%        

%        

%        
        & 433.6
%        

%        

%        
        & 1661.6
%        

%        

%        
        & 431.0
%        

%        

%        
        & 668
%        

%        

%        
        & 1360.0
%        

%        

%        
        & 819
%        

%        

%        
        & 0.055
%        

%        

%        
        & 0.794
%        

%        
        \\
        \hline

        

%        

%        
        -57.5
%        

%        

%        
        & 434.9
%        

%        

%        
        & 1660.1
%        

%        

%        
        & 431.0
%        

%        

%        
        & 666
%        

%        

%        
        & 1342.0
%        

%        

%        
        & 814
%        

%        

%        
        & 0.078
%        

%        

%        
        & 0.798
%        

%        
        \\
        \hline

        

%        

%        
        -49.4
%        

%        

%        
        & 438.4
%        

%        

%        
        & 1658.6
%        

%        

%        
        & 431.0
%        

%        

%        
        & 664
%        

%        

%        
        & 1296.0
%        

%        

%        
        & 803
%        

%        

%        
        & 0.142
%        

%        

%        
        & 0.811
%        

%        
        \\
        \hline

        

%        

%        
        -41.4
%        

%        

%        
        & 500.1
%        

%        

%        
        & 1657.5
%        

%        

%        
        & 431.0
%        

%        

%        
        & 662
%        

%        

%        
        & 805.0
%        

%        

%        
        & 697
%        

%        

%        
        & 0.806
%        

%        

%        
        & 0.954
%        

%        
        \\
        \hline

        

%        

%        
        -33.4
%        

%        

%        
        & 431.4
%        

%        

%        
        & 1703.7
%        

%        

%        
        & 431.0
%        

%        

%        
        & 660
%        

%        

%        
        & 1396.0
%        

%        

%        
        & 817
%        

%        

%        
        & 0.005
%        

%        

%        
        & 0.788
%        

%        
        \\
        \hline

        

%        

%        
        -25.4
%        

%        

%        
        & 431.6
%        

%        

%        
        & 1702.1
%        

%        

%        
        & 431.0
%        

%        

%        
        & 658
%        

%        

%        
        & 1395.0
%        

%        

%        
        & 815
%        

%        

%        
        & 0.007
%        

%        

%        
        & 0.788
%        

%        
        \\
        \hline

        

%        

%        
        -17.3
%        

%        

%        
        & 431.9
%        

%        

%        
        & 1700.5
%        

%        

%        
        & 431.0
%        

%        

%        
        & 656
%        

%        

%        
        & 1393.0
%        

%        

%        
        & 813
%        

%        

%        
        & 0.01
%        

%        

%        
        & 0.788
%        

%        
        \\
        \hline

        

%        

%        
        -9.3
%        

%        

%        
        & 432.6
%        

%        

%        
        & 1698.8
%        

%        

%        
        & 431.0
%        

%        

%        
        & 654
%        

%        

%        
        & 1387.0
%        

%        

%        
        & 811
%        

%        

%        
        & 0.017
%        

%        

%        
        & 0.789
%        

%        
        \\
        \hline

        

%        

%        
        -1.3
%        

%        

%        
        & 1454.0
%        

%        

%        
        & 1696.6
%        

%        

%        
        & 1414.7
%        

%        

%        
        & 650
%        

%        

%        
        & 1315.0
%        

%        

%        
        & 969
%        

%        

%        
        & 0.113
%        

%        

%        
        & 0.575
%        

%        
        \\
        \hline

        

%        

%        
        6.7
%        

%        

%        
        & 722.1
%        

%        

%        
        & 1698.7
%        

%        

%        
        & 718.3
%        

%        

%        
        & 653
%        

%        

%        
        & 1382.0
%        

%        

%        
        & 881
%        

%        

%        
        & 0.024
%        

%        

%        
        & 0.695
%        

%        
        \\
        \hline

        

%        

%        
        14.8
%        

%        

%        
        & 720.0
%        

%        

%        
        & 1701.0
%        

%        

%        
        & 718.3
%        

%        

%        
        & 657
%        

%        

%        
        & 1392.0
%        

%        

%        
        & 886
%        

%        

%        
        & 0.011
%        

%        

%        
        & 0.692
%        

%        
        \\
        \hline

        

%        

%        
        22.8
%        

%        

%        
        & 719.4
%        

%        

%        
        & 1703.4
%        

%        

%        
        & 718.3
%        

%        

%        
        & 660
%        

%        

%        
        & 1395.0
%        

%        

%        
        & 889
%        

%        

%        
        & 0.007
%        

%        

%        
        & 0.691
%        

%        
        \\
        \hline

        

%        

%        
        30.8
%        

%        

%        
        & 880.6
%        

%        

%        
        & 1685.2
%        

%        

%        
        & 718.3
%        

%        

%        
        & 663
%        

%        

%        
        & 785.0
%        

%        

%        
        & 704
%        

%        

%        
        & 0.834
%        

%        

%        
        & 0.945
%        

%        
        \\
        \hline

        

%        

%        
        38.8
%        

%        

%        
        & 729.1
%        

%        

%        
        & 1687.0
%        

%        

%        
        & 718.3
%        

%        

%        
        & 666
%        

%        

%        
        & 1325.0
%        

%        

%        
        & 874
%        

%        

%        
        & 0.102
%        

%        

%        
        & 0.716
%        

%        
        \\
        \hline

        

%        

%        
        46.8
%        

%        

%        
        & 723.8
%        

%        

%        
        & 1689.1
%        

%        

%        
        & 718.3
%        

%        

%        
        & 669
%        

%        

%        
        & 1359.0
%        

%        

%        
        & 886
%        

%        

%        
        & 0.056
%        

%        

%        
        & 0.703
%        

%        
        \\
        \hline

        

%        

%        
        54.9
%        

%        

%        
        & 722.0
%        

%        

%        
        & 1691.3
%        

%        

%        
        & 718.3
%        

%        

%        
        & 672
%        

%        

%        
        & 1372.0
%        

%        

%        
        & 891
%        

%        

%        
        & 0.039
%        

%        

%        
        & 0.699
%        

%        
        \\
        \hline

        

%        

%        
        62.9
%        

%        

%        
        & 721.1
%        

%        

%        
        & 1693.6
%        

%        

%        
        & 718.3
%        

%        

%        
        & 675
%        

%        

%        
        & 1378.0
%        

%        

%        
        & 895
%        

%        

%        
        & 0.03
%        

%        

%        
        & 0.696
%        

%        
        \\
        \hline

        

%        

%        
        70.9
%        

%        

%        
        & 720.6
%        

%        

%        
        & 1695.8
%        

%        

%        
        & 718.3
%        

%        

%        
        & 678
%        

%        

%        
        & 1382.0
%        

%        

%        
        & 898
%        

%        

%        
        & 0.025
%        

%        

%        
        & 0.695
%        

%        
        \\
        \hline

        

%        

%        
        78.9
%        

%        

%        
        & 720.2
%        

%        

%        
        & 1698.0
%        

%        

%        
        & 718.3
%        

%        

%        
        & 681
%        

%        

%        
        & 1385.0
%        

%        

%        
        & 901
%        

%        

%        
        & 0.021
%        

%        

%        
        & 0.694
%        

%        
        \\
        \hline

        

%        

%        
        87.0
%        

%        

%        
        & 719.9
%        

%        

%        
        & 1700.2
%        

%        

%        
        & 718.3
%        

%        

%        
        & 685
%        

%        

%        
        & 1387.0
%        

%        

%        
        & 904
%        

%        

%        
        & 0.018
%        

%        

%        
        & 0.694
%        

%        
        \\
        \hline

        

%        

%        
        95.0
%        

%        

%        
        & 743.5
%        

%        

%        
        & 1659.4
%        

%        

%        
        & 718.3
%        

%        

%        
        & 687
%        

%        

%        
        & 1196.0
%        

%        

%        
        & 852
%        

%        

%        
        & 0.285
%        

%        

%        
        & 0.769
%        

%        
        \\
        \hline

        

%        

%        
        103.0
%        

%        

%        
        & 727.1
%        

%        

%        
        & 1661.3
%        

%        

%        
        & 718.3
%        

%        

%        
        & 690
%        

%        

%        
        & 1319.0
%        

%        

%        
        & 890
%        

%        

%        
        & 0.114
%        

%        

%        
        & 0.717
%        

%        
        \\
        \hline

        

%        

%        
        111.0
%        

%        

%        
        & 723.6
%        

%        

%        
        & 1663.3
%        

%        

%        
        & 718.3
%        

%        

%        
        & 693
%        

%        

%        
        & 1347.0
%        

%        

%        
        & 900
%        

%        

%        
        & 0.075
%        

%        

%        
        & 0.706
%        

%        
        \\
        \hline

        

%        

%        
        119.1
%        

%        

%        
        & 722.1
%        

%        

%        
        & 1665.3
%        

%        

%        
        & 718.3
%        

%        

%        
        & 696
%        

%        

%        
        & 1360.0
%        

%        

%        
        & 906
%        

%        

%        
        & 0.057
%        

%        

%        
        & 0.701
%        

%        
        \\
        \hline

        

%        

%        
        127.1
%        

%        

%        
        & 721.3
%        

%        

%        
        & 1667.4
%        

%        

%        
        & 718.3
%        

%        

%        
        & 699
%        

%        

%        
        & 1367.0
%        

%        

%        
        & 910
%        

%        

%        
        & 0.047
%        

%        

%        
        & 0.699
%        

%        
        \\
        \hline

        

%        

%        
        135.1
%        

%        

%        
        & 720.7
%        

%        

%        
        & 1669.4
%        

%        

%        
        & 718.3
%        

%        

%        
        & 702
%        

%        

%        
        & 1372.0
%        

%        

%        
        & 914
%        

%        

%        
        & 0.04
%        

%        

%        
        & 0.697
%        

%        
        \\
        \hline

        

%        

%        
        143.1
%        

%        

%        
        & 720.3
%        

%        

%        
        & 1671.5
%        

%        

%        
        & 718.3
%        

%        

%        
        & 705
%        

%        

%        
        & 1376.0
%        

%        

%        
        & 917
%        

%        

%        
        & 0.035
%        

%        

%        
        & 0.695
%        

%        
        \\
        \hline

        

%        

%        
        151.2
%        

%        

%        
        & 720.1
%        

%        

%        
        & 1673.5
%        

%        

%        
        & 718.3
%        

%        

%        
        & 708
%        

%        

%        
        & 1378.0
%        

%        

%        
        & 919
%        

%        

%        
        & 0.031
%        

%        

%        
        & 0.695
%        

%        
        \\
        \hline

        
    \end{longtable}
%    

    \subsection{Интегральные параметры охлаждения.}

    
    \begin{enumerate}

        \item Температура охлаждающего воздуха на входе в канал:
            $ T_{в0}^{*} = 650\ К$.
        \item Полное давление охлаждающего воздуха на входе в канаЛ:
            $ p_{в0}^* = 0.5445\ МПа $.
        \item Суммарный расход воздуха на одну лопатку: $G_{в0} = 0.2\ кг/с$.
        \item Относительный расход воздуха на охлаждение СА:
            $g_{охл\ са} = 0.096$.
        \item Толщина стенок: $\Delta = 1.0\ мм$.
        \item Ширина канала: $\delta = 1.0\ мм$.
        \item Теплопроводность покрытия: $\lambda_п = 2\ Вт/(м \cdot К)$.
        \item Толщина покрытия: $\Delta_п = 0\ мм$.
        \item Средняя эффективность охлаждения: $\theta_{охл\ ср} = 0.724$.
    \end{enumerate}
    





\end{document}