%%
%% Author: User1
%% 13.12.2017
%%

% Preamble
\documentclass[a4paper,10pt]{article}

% Packages
\usepackage{mathtext}
\usepackage[T2A]{fontenc}
\usepackage[utf8]{inputenc}
\usepackage[russian]{babel}
\usepackage{amsmath}
\usepackage{amsfonts}
\usepackage{amssymb}
\usepackage{graphicx}
\usepackage[left=2cm,right=2cm,
    top=2cm,bottom=2cm,bindingoffset=0cm]{geometry}
\usepackage{color}
\usepackage{gensymb}

\usepackage{enumitem}
\setlist[enumerate]{label*=\arabic*.}

\usepackage{indentfirst}

\usepackage{titlesec}


% Document
\begin{document}

    \section{Расчет турбины по средней линии тока}

    

    \subsection{Расчет первой ступени}

    \subsubsection{Исходные данные}

    
    \begin{enumerate}

        \item Температура торможения на входе в ступень: $T_0^* = 1400\ К $.
        \item Давление торможения на входе в ступень: $p_0^* = 0.55 \cdot 10^6 \ Па$.
        \item Расход газа на входе в ступень: $G_{вх} = 25\ кг/с$.
        \item Расход газа на входе в СА первой ступени: $ G_т = 25\ кг/с $.
        \item Расход топлива на входе в турбину: $ G_{топл} = 1\ кг/с $.
        \item Степень реактивности: $ \rho = 0.4 $.
        \item Коэффициент скорости в СА: $ \phi = 0.97 $.
        \item Коэффициент скорости в РК: $ \psi = 0.97 $.
        \item Длина лопатки на входе в РК: $ l_1 = 0.1204\ м $.
        \item Длина лопатки на выходе из РК: $ l_2 = 0.1334\ м $.
        \item Средний диаметр на входе в РК: $ D_1 = 0.4816\ м $.
        \item Средний диаметр на выходе в РК: $ D_2 = 0.4919\ м $.
        \item Радиальный зазор: $ \delta_r = 0.00133\ м $.
        \item Частота вращения ротора: $ n = 15000.0\ об/мин $
        \item Степень парциальности: $ \varepsilon = 1 $.
        \item Расход охлаждающего воздуха, отнесенный к расходу на входе в турбину: $ g_{охл} = 0.004 $.
        \item Температура торможения охлаждающего воздуха: $ T_{охл} = 700\ К $.

        
        \item Статический теплоперепад на ступени: $ H_0 = 0.1636 \cdot 10^6 \ Дж/кг $.

        

    \end{enumerate}
    

    \subsubsection{Расчет}

    
    \begin{enumerate}

        \item Относительный расход топлива на входе в ступень:
        \[
            g_{топл.вх} = \frac{ G_{топл} }{ G_{вх} - G_{топл} } =
                \frac{ 1 }{ 25 - 1 } =
            0.0417
        \]

        \item Коэффициент избытка воздуха на входе:
        \[
            \alpha_{вх} = \frac{ 1 }{ l_0 g_{топл.вх} } =
                \frac{ 1 }{ 14.61 \cdot 0.0417 } =
            1.643
        \]

        \item Средний в ступени коэффициент адиабаты из предпоследней итерации:
        \[
            k_г = 1.2937
        \]

        \item Средняя в ступени теплоемкость газа из предпоследней итерации:
        \[
            c_{pг} = 1265.84 \ Дж/(кг \cdot К)
        \]

        
        

        

        \item Определим теплоперепад на сопловом аппарате:

        \[
            H_с = \left( 1 - \rho \right) H_0 =
	        \left( 1 - 0.4 \right) \cdot 0.1636 \cdot 10^6 =
            0.0982 \cdot 10^6 \/\ Дж/кг
        \]

        \item Окружная скорость на диаметре $ D_1 $:

        \[
            u_1 = \frac{\pi D_1 n }{60} =
                \frac{\pi \cdot 0.4816 \cdot 15000.0} =
            378.22\ м/с
        \]

        \item Определим действительную скорость истечения из СА:

	    \[
            c_1 = \phi \sqrt{2 H_с} =
	        0.97 \cdot\sqrt{2 \cdot 0.1636 \cdot 10^6}  =
            429.82 \/\ м/с
        \]

        \item Определим температуру на выходе из СА:

	    \[
            T_1 = T_0^* - \frac{ H_с \phi^2 }{ c_{pг} } =
	        1400 -
            \frac{
                0.0982 \cdot 10^6 \cdot {0.97}^2
            }{
                2 \cdot 1265.84
            } = 1327.03 \/\ К
        \]

	    \item Определим температуру конца адиабатного расширения:

	    \[
            T_1^\prime = T_0^* - \frac{ H_c }{ c_{pг} } =
	        1400 -
            \frac{
                0.0982 \cdot 10^6
            }{
                1265.84
            }
            = 1322.44  \/\ К
        \]

        \item Определим давление на выходе из СА:

	    \[
            p_1 = p_0^* \left(
                                \frac{ T_1^\prime }{ T_0^* }
                        \right)^
                    \frac{ k_г }{ k_г - 1 } =
            0.55 \cdot 10^6 \cdot
                \left(
                        \frac{ 1322.44 }{ 1400 }
                \right)^
                \frac{ 1.2937 }{ 1.2937 - 1 } =
            0.4279 \cdot 10^6 \/\ МПа
        \]

        \item Определим площадь на выходе из СА:

	    \[
            A_{1a} = \pi l_1 D_1 =
	        \pi \cdot 0.1204 \cdot 0.4816 =
            0.18214 \/\ м^2
        \]

        \item Определим плотность газа на выходе из СА:

	    \[
            \rho_1 = \frac{p_1}{R_г T_1} =
	        \frac{
                0.4279 \cdot 10^6
            }{
                287.4 \cdot 1327.03
            } =
            1.122 \/\ кг/м^3
        \]

        \item Осевая составляющая абсолютной скорости на выходе из СА:

        \[
            c_{1a} = \frac{G_{вх} }{ \rho_1 A_{1a} } =
                \frac{
                    25
                }{
                    1.122 \cdot 0.18214
                } =
            122.34\ м/с
        \]

        \item Угол потока в абсолютном движении после СА:

        \[
            \alpha_1 = \arcsin{ \frac{ c_{1a} }{ c_1 } } =
            \arcsin{ \frac{ 122.34 }{ 429.82 } } =
            = 16.536 \degree
        \]

        \item Окружная составляющая абсолютной скорости на входе:

        \[
            c_{1u} = c_1 \cos{\alpha_1} = 429.82 \cdot \cos{16.536 \degree} =
            412.04\ м/с
        \]

        \item Определим относительную скорость на входе в РК:

	    \[
            w_1 = \sqrt{c_1^2 + u_1^2 - 2 c_1 u_1 \cos \alpha_1} =
	        \sqrt{
            429.82 ^ 2 +
            378.22 ^ 2 -
            2 \cdot 429.82 \cdot 378.22 \cdot \cos 16.16.536 \degree
            }
            = 126.93 \/\ м/с\]

        \item Угол потока в относительном движении:

        
        \[
            \beta_1 = \arctan{ \frac{c_{1a}}{c_{1u} - u_1} } =
                    \arctan{ \frac{ 122.34 }{412.04 - 378.22} } =
            74.545 \degree
        \]
        

        \item Осевая составляющая относительной скорости:

        \[
            w_{1a} = w_1 \sin{\beta_1} = 126.93 \cdot  \sin{74.545 \degree} =
            122.34\ м/с
        \]

        \item Окружная составляющая относительной скорости:

        \[
            w_{1u} = w_1 \cos{\beta_1} = 126.93 \cdot  \cos{74.545 \degree} =
            33.82\ м/с
        \]

         \item Определим теплоперепад на РК:

	    \[
            H_л = H_0 \rho \frac{T_1}{T_1^\prime} =
	        0.1636 \cdot 10^6 \cdot 0.4 \cdot
            \frac{ 1327.03 }{ 1322.44 } =
            0.0657 \cdot 10^6 \/\ Дж/кг
        \]

        \item Окружная скорость на диаметре:

        \[
            u_2 = \frac{ \pi D_2 n }{ 60 } =
                    \frac{ \pi \cdot 0.4919 \cdot 15000.0 }{ 60 } =
            386.36\ м/с
        \]

        \item Температура торможения в относительном движении после СА:

        \[
            T_{1w}^* = T_1 + \frac{ w_1^2 }{ 2 \cdot c_{pг}} =
                1327.03 + \frac{ 126.93 ^ 2 }{ 2 \cdot 1265.84}
        \]

        \item Определим относительную скорость истечения газа из РК:

	    \[
            w_2 = \psi \sqrt{w_1^2 + 2H_л +\left( u_2^2 - u_1^2 \right)} =
	        0.97 \cdot
            \sqrt{
                126.93 ^ 2 +
                2 \cdot 0.0657 \cdot 10^6 +
                \left( 386.36 ^ 2 - 378.22 ^ 2 \right)
            } =
            380.27 \/\ м/с
        \]

        \item Определим статическую температуру на выходе из РК:

	    \[
            T_2 = T_1 + \frac{
	 	        \left( w_1^2  - w_2^2 \right) + \left( u_2^2 - u_1^2 \right)
            }{
                2 c_{pг}
            } =
	        1327.03 + \frac{
	 	        \left( 126.93 ^ 2  - 380.27 ^ 2 \right) +
                \left( 386.36 ^ 2 - 378.22 ^ 2 \right)
	        }{
            2 \cdot 1265.84
            }
            = 1278.73 \/\ К
        \]

        \item Определим статическую температуру при адиабатическом процессе в РК:

	    \[
            T_2^\prime = T_1 - \frac{
	 	        H_л
	        }{ c_{p г}} =
	        1327.03 - \frac{
	 	        0.0657 \cdot 10^6
	        }{
                1265.84
            }
            = 1275.14 \/\ К
        \]

        \item Определим давление на выходе из РК:

	    \[
            p_2 = p_1 \left( \frac{T_2^\prime}{T_1} \right)^{\frac{k_г}{k_г - 1}} =
               0.4279 \cdot 10^6 \cdot
               \left(
               \frac{ 1275.14 }{ 1327.03 }
               \right) ^
               {\frac{
               1.294
               }{
               1.294 - 1
               }}
            = 0.359 \cdot 10^6 \/\ Па
        \]

        \item Определим плотность газа на выходе из РК:
	    \[
            \rho_2 = \frac{p_2}{R T_2} =
                \frac{
                    0.359 \cdot 10^6
                }{
                    287.4 \cdot 1278.73
                }
            = 0.977\ кг/м^3
        \]

        \item Определим площадь на выходе из РК:
        \[
            A_{2a} = \pi D_2 l_2 = \pi \cdot 0.4919 \cdot 0.1334 =
            0.2062\ м^2
        \]

        \item Осевая составляющая абсолютной скорости на выходе из РК:
        \[
            c_{2a} = \frac{ G_{вх} }{ A_{2a} \rho_2 } =
            \frac{ 25 }{ 0.2062 \cdot 0.977 }
            = 124.13\ м/с
        \]

        \item Угол потока в относительном движении на выходе из РК:
        \[
            \beta_2 = \arcsin{ \frac{ c_{2a} }{ w_2 } } =
                    \arcsin{ \frac{ 124.13 }{ 380.27 } }
            = 19.051 \degree
        \]

        \item Осевая составляющая относительной скорости потока на выходе из РК:
        \[
            w_{2a} = w_2 \cdot \sin{\beta_2} =
                    380.27 \cdot \sin{19.051 \degree}
            = 124.13\ м/с
        \]

        \item Окружная составляющая относительной скорости потока на выходе из РК:
        \[
            w_{2u} = w_2 \cdot \cos{\beta_2} =
                    380.27 \cdot \cos{19.051 \degree}
            = 359.44\ м/с
        \]

        \item Определим окружную составляющую скорости на выходе из РК:
	    \[
            c_{2u} = w_{2u} - u_2 =
	        359.44 - 386.36 = -26.92 \/\ м/с
        \]

        \item Опеределим угол потока на выходе из РК:
        
        \[
            \alpha_2 = \pi + \arctan{ \frac{ c_{2a} }{ c_{2u} } } =
                    \pi + \arctan{ \frac{ 124.13 }{ -26.92 } } =
            102.235 \degree
        \]
        

        \item Определим скорость потока на выходе из РК:
	    \[
            c_2 = \sqrt{c_{2u}^2 + c_{2a}^2} =
                \sqrt{-26.92 ^ 2 + 124.13 ^ 2} =
            127.01 \/\ м/с
        \]

        \item Определим работу на окружности колеса:
	    \[
            L_u = c_{1u} u_1 + c_{2u} u_2 =
                    412.04 \cdot 378.22 +
                    -26.92 \cdot 386.36 =
            0.1454 \cdot 10^6 \/\ Дж/кг
        \]

        \item Определим КПД на окружности колеса:
	    \[
            \eta_u = \frac{L_u}{H_0} =
                \frac{ 0.1454 \cdot 10^6 }{ 0.1636 \cdot 10^6 }
            = 0.8889
        \]

        \item Определим удельные потери в СА:
	    \[
            h_с = \left(
                        \frac{ 1 }{ \phi^2 } - 1
                \right)
                \frac{ c_1^2 }{ 2 } =
	        \left(
                \frac{ 1 }{ 0.97 ^ 2} - 1
            \right) \cdot
            \frac{ 429.82 ^ 2 }{ 2 } = 5.8022 \cdot 10^3 \/\ Дж/кг
        \]

        \item Удельные потери в СА с учетом их использования в рабочих лопатках:
        \[
            h_с^\prime = h_с \frac{ T_2^\prime }{ T_1 } =
                5.8022 \cdot 10^3 \cdot
                \frac{ 1275.14 }{ 1327.03 } =
            5.5753 \cdot 10^3 \/\ Дж/кг
        \]

        \item Относительные потери в СА:
        \[
            \zeta_с = \frac{ h_с }{ H_0 } =
                \frac{ 5.8022 \cdot 10^3 }{ 0.1636 \cdot 10^6 } =
            0.0355
        \]

        \item Относительные потери в СА с учетом их использования в рабочих лопатках:
        \[
            \zeta_с^\prime = \frac{ h_с^\prime }{ H_0 } =
                \frac{ 5.5753 \cdot 10^3 }{ 0.1636 \cdot 10^6 } =
            0.0341
        \]

        \item Удельные потери в рабочих лопатках:
        \[
            h_л = \left(
                    \frac{ 1 }{ \psi^2 } - 1
                \right)) \cdot
                \frac{ w_2^2 }{ 2 } =
            \left(
                \frac{ 1 }{ 0.97 ^ 2 } - 1
            \right) \cdot
            \frac{ 380.27 ^ 2} {2}
            = 4.5416 \cdot 10^3 \/\ Дж/кг
        \]

        \item Относительные потери в рабочих лопатках:
        \[
            \zeta_л = \frac{ h_л }{ H_0 } =
                \frac{ 4.5416 \cdot 10^3 }{ 0.1636 \cdot 10^6 } =
            0.0278
        \]

        \item Определим удельные потери с выходной скоростью:
        \[
            h_{вых} = \frac{ c_2 ^ 2 }{ 2} =
                    \frac{ 127.01 ^ 2 }{ 2 } =  8.0658 \cdot 10^3 \/\ Дж/кг
        \]

        \item Относительные потери с выходной скоростью:
        \[
            \zeta_{вых} = \frac{ h_{вых} }{ H_0 } =
                \frac{ 8.0658 \cdot 10^3 }{ 0.1636 \cdot 10^6 } =
            0.0493
        \]

        \item Проверка КПД на окружности колеса:
        \[
            \eta_u = 1 - \zeta_с^\prime - \zeta_л - \zeta_{вых}
        \]

        \item Средний диаметр:
        \[
            D_{ср} = 0.5 \cdot (D_1 + D_2) =
                    0.5 \cdot (0.4816 + 0.4919) =
            0.4867\ м
        \]

        \item Определим удельные потери в радиальном зазоре:

	    \begin{gather*}
	        h_з = 1.37 \cdot
                \left(
                    1 + 1.6 \rho
                \right)
                \left(
                    1 + \frac{l_2}{D_{ср}}
                \right)
            \frac{ \delta_r }{ l_2 } \cdot L_u =\\
	        = 1.37 \cdot
            \left(
                1 + 1.6 \cdot 0.4
            \right)
            \left(
                1 + \frac{ 0.1334 }{ 0.4867 }
            \right)
            \frac{ 0.00133 }{ 0.1334 } \cdot
            0.1454 \cdot 10^6 =
	        4.1636 \cdot 10^3 \/\ Дж/кг\\
	    \end{gather*}

        \item Относительные удельные потери в радиальном зазоре:
        \[
            \zeta_з = \frac{ h_з }{ H_0 } =
                \frac{ 4.1636 \cdot 10^3 }{ 0.1636 \cdot 10^6 } =
            0.0254
        \]

        \item Удельная работа ступени с учетом потери в радиальном зазоре:
        \[
            L_{uз} = L_u - h_з = 0.1454 \cdot 10^6 -
                4.1636 \cdot 10^3 =
            0.1413 \cdot 10^6 \ Дж/кг
        \]

        \item Мощностной КПД ступени:
        \[
            \eta_т^\prime = \eta_u - \zeta_з =
                0.8889 - 0.0254 = 0.8634
        \]

        \item Лопаточный КПД ступени:
        \[
            \eta_л^\prime = \eta_т^\prime + \zeta_{вых} =
                 0.8634 +  0.0493 =
            0.9127
        \]

        \item Средняя длина лопатки:
        \[
            l_{ср} = 0.5 \cdot (l_1 + l_2) =
                0.5 \cdot (0.1204 + 0.1334) =
            0.1269\ м
        \]

        \item Средняя окружная скоротсь:
        \[
            u_{ср} = 0.5 \cdot (u_1 + u_2) =
                0.5 \cdot (378.22 + 386.36) =
            382.29\ м/с
        \]

        \item Затраты мощности на трение и вентиляцию:
        \begin{gather*}
            N_{т.в} = \left[
                    1.07 \cdot D_{av}^2 + 61 \cdot (1 - \varepsilon) \cdot D_{av} l_{av}
            \right] \cdot
            \left(
                \frac{ u_{av} }{ 100 }
            \right) ^ 3 \cdot
            \rho =\\
            = \left[
                1.07 \cdot 0.4867^2 +
                61 \cdot (1 - 1) \cdot
                0.4867 \cdot 0.1269
            \right] \cdot
            \left(
                \frac{ 382.29 }{ 100 }
            \right) ^ 3 \cdot
            0.4
            = 0.0057 \cdot 10^3 \ Вт \\
        \end{gather*}

        \item Удельные потери на трение и вентиляцию:
        \[
            h_{т.в} = \frac{ N_{т.в} }{ G_{вх} } =
                \frac{
                    0.0057 \cdot 10^3
                }{
                    25
                }
            = 0.0002 \cdot 10^3 \ Дж/кг
        \]

        \item Относительные потери на трение и вентиляцию:
        \[
            \zeta_{т.в} = \frac{ h_{т.в} }{ H_0 } =
                \frac{ 0.0002 \cdot 10^3 }{ 0.1636 \cdot 10^6 } =
            0.0
        \]

        \item Мощностной КПД с учетом потерь на трению и вентиляцию:
        \[
            \eta_т = \eta_т^\prime - \zeta_{т.в} =
                0.8634 - 0.0 =
            0.8634
        \]

        \item Лопаточный КПД с учетом потерь на трению и вентиляцию:
        \[
            \eta_л = \eta_л^\prime - \zeta_{т.в} =
                0.9127 - 0.0 =
            0.9127
        \]

        
        \item Определим удельную работу ступени:
        \[
            L_т^\prime = H_0 \eta_т = 0.1636 \cdot 10^6 \cdot \eta_т =
            = 0.1413 \cdot 10^6 \ Дж/кг
        \]
        

        \item Удельная работа ступени с учетом добавления охлаждающего воздуха:
        \[
            L_т = L_т^\prime \frac{ G_{вх} }{ G_т }  - L_т^\prime g_{охл} =
                0.1413 \cdot 10^6 \cdot
                \frac{ 25 }{ 25 }  -
                0.1413 \cdot 10^6 \cdot 0.004 =
            0.1418 \cdot 10^6 \ Дж/кг
        \]

        \item Статическая температура за ступенью:
        \[
            T_{ст} = T_2 + \frac{ h_з }{ c_{pг} } + \frac{ h_{т.в} }{ c_{pг} } =
                1278.73 +
                \frac{4.1636 \cdot 10^3 }{ 1265.84 } +
                \frac{ 0.0002 \cdot 10^3 }{ 1265.84 } =
            1282.02 \ К
        \]

        \item Температура торможения за турбиной:
        \[
            T_{ст}^* = T_{ст} + \frac{ h_{вых} }{ c_{pг} } =
                1282.02 +
                \frac{ 8.0658 \cdot 10^3 }{ 1265.84 } =
            1288.39 \ К
        \]

        \item Средняя теплоемкость газа в интервале температур от 273 К до $T_0^*$:
        \[
            c_{pг\ ср} (T_0^*, \alpha_{вх}) =
            1142.12 \ Дж/(кг \cdot К)
        \]

        \item Средняя теплоемкость газа в интервале температур от 273 К до $T_{ст}$:
        \[
            c_{pг\ ср} (T_{ст}, \alpha_{вх}) =
            1127.49 \ Дж/(кг \cdot К)
        \]

        \item Средняя теплоемкость газа в интервале температур от $T_0^*$ до $T_{ст}$:
        \begin{gather*}
            c_{pг}^\prime = \frac{
		        c_{pг\ ср} (T_0^*, \alpha_{вх}) (T_0^* - T_0) - c_{pг\ ср} (T_{ст}, \alpha_{вх})(T_{ст} - T_0)
		    }{
		        T_0^* - T_{ст}} =\\
            =\frac{
		        1142.12 \cdot
                (1400 - 273) -
		        1127.49 \cdot
                (1282.02 - 273)
		    }{
		        1400 - 1282.02} =
		    1267.21 \ Дж / (кг \cdot К)\\
        \end{gather*}

        \item Новое значение показателя адиабаты:
        \[
            k_г^\prime = \frac{c_{pг}^\prime}{c_{pг}^\prime - R_г} =
                \frac{
                    1267.21
                }{
                    1267.21 - 287.4
                }
            = 1.2933
        \]

        \item Невязка по коэффициенту адиабаты:
        \[
            \delta = \frac{ \left| k_г - k_г^\prime \right| }{ k_г } \cdot 100 \%=
                \frac{
                    \left| 1.2937 - 1.2933 \right|
                }{
                    1.2937
                } \cdot 100 \% =
            0.0319 \%
        \]

        \item Давление торможения на выходе из ступени:
        \[
            p_2^* = p_2 \left(
                            \frac{ T_{ст}^* }{ T_{ст} }
                    \right) ^ \frac{ k_г }{ k_г - 1 } =
                 0.359 \cdot 10^6 \cdot \left(
                            \frac{ 1288.39 }{ 1282.02 }
                    \right) ^
                \frac{ 1.2937 }{ 1.2937 - 1 } =
            0.3669 \cdot 10^6 \ Па
        \]

        \item Теплоперепад по параметрам торможения:
        \[
            H_0^* = c_{pг} T_0^* \left[
                        1 - \left(
                                \frac{p_2^*}{p_0^*}
                            \right) ^
                        \frac{k_г - 1}{k_г}
                    \right] =
            1265.84 \cdot 1400 \cdot
                    \left[
                        1 - \left(
                                \frac{
                                    0.3669 \cdot 10^6
                                }{
                                    0.55 \cdot 10^6
                                }
                            \right) ^
                        \frac{1.2937 - 1}{1.2937}
                    \right] =
            0.1556 \cdot 10^6 \ Дж/кг
        \]

        \item КПД по параметрам торможения:
        \[
            \eta_т^* = \frac{ L_т }{ H_0^* } =
                \frac{
                    0.1418 \cdot 10^6
                }{
                    0.1556 \cdot 10^6 } =
            0.9114
        \]

        \item Расход на выходе из ступени:
        \[
            G_{вых} = G_{вх} + G_т g_{охл} =
                25 + 25 \cdot
                0.004 =
            25.1 \ кг/с
        \]

        \item Относительный расход топлива на выходе из ступени:
        \[
            g_{топл.вых} = \frac{ G_{топл} }{ G_{вых} - G_{топл} } =
                 \frac{ 1 }{ 25.1 - 1 } =
            0.0415
        \]

        \item Коэффициент избытка воздуха на выходе из ступени:
        \[
            \alpha_{вых} = \frac{ 1 }{ l_0 g_{топл.вых} } =
                \frac{ 1 }{ 14.61 \cdot 0.0415 } =
            1.65
        \]

        \item Абсолютный расход охлаждающего воздуха:
        \[
            G_{охл} = G_т g_{охл} = 25 \cdot 0.004 =
            0.1
        \]

        \item Определим температуру торможения на выходе из турбины после подмешивания охлаждающего воздуха.
        \begin{enumerate}

            \item Истинная теплоемкость охлаждающего воздуха при температуре $T_{охл} = 700\ К $:
            \[
                c_{pв} (T_{охл}) = 1074.86\ Дж/ (кг \cdot К)
            \]

            \item Истинная теплоемкость газа при температуре $T_{ст}^* = 1288.39 \ К $:
            \[
                c_{pг} (T_{ст}^*, \alpha_{вх}) =
                1280.34\ Дж/ (кг \cdot К)
            \]

            \item Значение температуры смеси с предпоследней итерации $T_{см}^{*} = 1285.98\ К$.

            \item Истинная теплоемкость смеси:
            \[
                c_{pг} (T_{см}^{*}, \alpha_{вых}) =
                1279.54\ Дж/ (кг \cdot К)
            \]

            \item Новое значение температуры смеси:
            \[
                T_{см}^*\prime = \frac{
                        c_{pг} (T_{ст}^*, \alpha_{вх}) T_{ст}^* G_{вх} + c_{pв} (T_{охл}) T_{охл} G_{охл}
                    }{
                        c_{pг} (T_{см}^{*}, \alpha_{вых}) G_{вых}
                    } =
                \frac{
                    1280.34
                    \cdot 1288.39 \cdot 25 +
                    1074.86
                    \cdot 700 \cdot 0.1
                }{
                    1279.54
                    \cdot  25.1
                } =
                1286.4\ К
            \]

            \item Значение невязки:
            \[
                \delta = \frac{ \left| T_{см}^{*} - T_{см}^*\prime \right| }{T_{см}^{*}} \cdot 100 \% =
                    \frac{
                        \left| 1285.98 - 1286.4 \right|
                    }{
                        1285.98
                    } \cdot 100 \% =
                0.033 \%
            \]

        \end{enumerate}

        

    \end{enumerate}
    

    \subsection{Расчет второй ступени}

    \subsubsection{Исходные данные}

    
    \begin{enumerate}

        \item Температура торможения на входе в ступень: $T_0^* = 1285.98\ К $.
        \item Давление торможения на входе в ступень: $p_0^* = 0.3669 \cdot 10^6 \ Па$.
        \item Расход газа на входе в ступень: $G_{вх} = 25.1\ кг/с$.
        \item Расход газа на входе в СА первой ступени: $ G_т = 25\ кг/с $.
        \item Расход топлива на входе в турбину: $ G_{топл} = 1\ кг/с $.
        \item Степень реактивности: $ \rho = 0.496 $.
        \item Коэффициент скорости в СА: $ \phi = 0.97 $.
        \item Коэффициент скорости в РК: $ \psi = 0.97 $.
        \item Длина лопатки на входе в РК: $ l_1 = 0.1691\ м $.
        \item Длина лопатки на выходе из РК: $ l_2 = 0.1874\ м $.
        \item Средний диаметр на входе в РК: $ D_1 = 0.5122\ м $.
        \item Средний диаметр на выходе в РК: $ D_2 = 0.5268\ м $.
        \item Радиальный зазор: $ \delta_r = 0.00187\ м $.
        \item Частота вращения ротора: $ n = 15000.0\ об/мин $
        \item Степень парциальности: $ \varepsilon = 1 $.
        \item Расход охлаждающего воздуха, отнесенный к расходу на входе в турбину: $ g_{охл} = 0.003 $.
        \item Температура торможения охлаждающего воздуха: $ T_{охл} = 700\ К $.

        
        \item Удельная работы турбины: $ L_т = 0.1088 \cdot 10^6 \ Дж/кг $.

        

    \end{enumerate}
    

    \subsubsection{Расчет}

    

    \begin{enumerate}

        \item Относительный расход топлива на входе в ступень:
        \[
            g_{топл.вх} = \frac{ G_{топл} }{ G_{вх} - G_{топл} } =
                \frac{ 1 }{ 25.1 - 1 } =
            0.0415
        \]

        \item Коэффициент избытка воздуха на входе:
        \[
            \alpha_{вх} = \frac{ 1 }{ l_0 g_{топл.вх} } =
                \frac{ 1 }{ 14.61 \cdot 0.0415 } =
            1.65
        \]

        \item Мощностной КПД из предпоследней итерации:
        \[
            \eta_т^\prime = 0.8516
        \]

        \item Статический теплоперепад на ступени:
        \[
            H_0 = \frac{L_т}{\eta_т^\prime} =
                \frac{ 0.1088 \cdot 10^6 }{ 0.8516 } =
            0.1278 \cdot 10^6 \ Дж/кг
        \]

        \item Средний в ступени коэффициент адиабаты из предпоследней итерации:
        \[
            k_г = 1.3015
        \]

        \item Средняя в ступени теплоемкость газа из предпоследней итерации:
        \[
            c_{pг} = 1240.65 \ Дж/(кг \cdot К)
        \]

        
        

        

        \item Определим теплоперепад на сопловом аппарате:

        \[
            H_с = \left( 1 - \rho \right) H_0 =
	        \left( 1 - 0.496 \right) \cdot 0.1278 \cdot 10^6 =
            0.0644 \cdot 10^6 \/\ Дж/кг
        \]

        \item Окружная скорость на диаметре $ D_1 $:

        \[
            u_1 = \frac{\pi D_1 n }{60} =
                \frac{\pi \cdot 0.5122 \cdot 15000.0} =
            402.32\ м/с
        \]

        \item Определим действительную скорость истечения из СА:

	    \[
            c_1 = \phi \sqrt{2 H_с} =
	        0.97 \cdot\sqrt{2 \cdot 0.1278 \cdot 10^6}  =
            348.04 \/\ м/с
        \]

        \item Определим температуру на выходе из СА:

	    \[
            T_1 = T_0^* - \frac{ H_с \phi^2 }{ c_{pг} } =
	        1285.98 -
            \frac{
                0.0644 \cdot 10^6 \cdot {0.97}^2
            }{
                2 \cdot 1240.65
            } = 1237.16 \/\ К
        \]

	    \item Определим температуру конца адиабатного расширения:

	    \[
            T_1^\prime = T_0^* - \frac{ H_c }{ c_{pг} } =
	        1285.98 -
            \frac{
                0.0644 \cdot 10^6
            }{
                1240.65
            }
            = 1234.09  \/\ К
        \]

        \item Определим давление на выходе из СА:

	    \[
            p_1 = p_0^* \left(
                                \frac{ T_1^\prime }{ T_0^* }
                        \right)^
                    \frac{ k_г }{ k_г - 1 } =
            0.3669 \cdot 10^6 \cdot
                \left(
                        \frac{ 1234.09 }{ 1285.98 }
                \right)^
                \frac{ 1.3015 }{ 1.3015 - 1 } =
            0.3071 \cdot 10^6 \/\ МПа
        \]

        \item Определим площадь на выходе из СА:

	    \[
            A_{1a} = \pi l_1 D_1 =
	        \pi \cdot 0.1691 \cdot 0.5122 =
            0.27214 \/\ м^2
        \]

        \item Определим плотность газа на выходе из СА:

	    \[
            \rho_1 = \frac{p_1}{R_г T_1} =
	        \frac{
                0.3071 \cdot 10^6
            }{
                287.4 \cdot 1237.16
            } =
            0.864 \/\ кг/м^3
        \]

        \item Осевая составляющая абсолютной скорости на выходе из СА:

        \[
            c_{1a} = \frac{G_{вх} }{ \rho_1 A_{1a} } =
                \frac{
                    25.1
                }{
                    0.864 \cdot 0.27214
                } =
            106.77\ м/с
        \]

        \item Угол потока в абсолютном движении после СА:

        \[
            \alpha_1 = \arcsin{ \frac{ c_{1a} }{ c_1 } } =
            \arcsin{ \frac{ 106.77 }{ 348.04 } } =
            = 17.865 \degree
        \]

        \item Окружная составляющая абсолютной скорости на входе:

        \[
            c_{1u} = c_1 \cos{\alpha_1} = 348.04 \cdot \cos{17.865 \degree} =
            331.26\ м/с
        \]

        \item Определим относительную скорость на входе в РК:

	    \[
            w_1 = \sqrt{c_1^2 + u_1^2 - 2 c_1 u_1 \cos \alpha_1} =
	        \sqrt{
            348.04 ^ 2 +
            402.32 ^ 2 -
            2 \cdot 348.04 \cdot 402.32 \cdot \cos 16.17.865 \degree
            }
            = 128.26 \/\ м/с\]

        \item Угол потока в относительном движении:

        
        \[
            \beta_1 = \pi + \arctan{ \frac{c_{1a}}{c_{1u} - u_1} } =
                    \pi + \arctan{ \frac{ 106.77 }{331.26 - 402.32} } =
            123.645 \degree
        \]
        

        \item Осевая составляющая относительной скорости:

        \[
            w_{1a} = w_1 \sin{\beta_1} = 128.26 \cdot  \sin{123.645 \degree} =
            106.77\ м/с
        \]

        \item Окружная составляющая относительной скорости:

        \[
            w_{1u} = w_1 \cos{\beta_1} = 128.26 \cdot  \cos{123.645 \degree} =
            -71.06\ м/с
        \]

         \item Определим теплоперепад на РК:

	    \[
            H_л = H_0 \rho \frac{T_1}{T_1^\prime} =
	        0.1278 \cdot 10^6 \cdot 0.4962 \cdot
            \frac{ 1237.16 }{ 1234.09 } =
            0.0635 \cdot 10^6 \/\ Дж/кг
        \]

        \item Окружная скорость на диаметре:

        \[
            u_2 = \frac{ \pi D_2 n }{ 60 } =
                    \frac{ \pi \cdot 0.5268 \cdot 15000.0 }{ 60 } =
            413.75\ м/с
        \]

        \item Температура торможения в относительном движении после СА:

        \[
            T_{1w}^* = T_1 + \frac{ w_1^2 }{ 2 \cdot c_{pг}} =
                1237.16 + \frac{ 128.26 ^ 2 }{ 2 \cdot 1240.65}
        \]

        \item Определим относительную скорость истечения газа из РК:

	    \[
            w_2 = \psi \sqrt{w_1^2 + 2H_л +\left( u_2^2 - u_1^2 \right)} =
	        0.97 \cdot
            \sqrt{
                128.26 ^ 2 +
                2 \cdot 0.0635 \cdot 10^6 +
                \left( 413.75 ^ 2 - 402.32 ^ 2 \right)
            } =
            379.26 \/\ м/с
        \]

        \item Определим статическую температуру на выходе из РК:

	    \[
            T_2 = T_1 + \frac{
	 	        \left( w_1^2  - w_2^2 \right) + \left( u_2^2 - u_1^2 \right)
            }{
                2 c_{pг}
            } =
	        1237.16 + \frac{
	 	        \left( 128.26 ^ 2  - 379.26 ^ 2 \right) +
                \left( 413.75 ^ 2 - 402.32 ^ 2 \right)
	        }{
            2 \cdot 1240.65
            }
            = 1189.58 \/\ К
        \]

        \item Определим статическую температуру при адиабатическом процессе в РК:

	    \[
            T_2^\prime = T_1 - \frac{
	 	        H_л
	        }{ c_{p г}} =
	        1237.16 - \frac{
	 	        0.0635 \cdot 10^6
	        }{
                1240.65
            }
            = 1185.94 \/\ К
        \]

        \item Определим давление на выходе из РК:

	    \[
            p_2 = p_1 \left( \frac{T_2^\prime}{T_1} \right)^{\frac{k_г}{k_г - 1}} =
               0.3071 \cdot 10^6 \cdot
               \left(
               \frac{ 1185.94 }{ 1237.16 }
               \right) ^
               {\frac{
               1.301
               }{
               1.301 - 1
               }}
            = 0.2559 \cdot 10^6 \/\ Па
        \]

        \item Определим плотность газа на выходе из РК:
	    \[
            \rho_2 = \frac{p_2}{R T_2} =
                \frac{
                    0.2559 \cdot 10^6
                }{
                    287.4 \cdot 1189.58
                }
            = 0.748\ кг/м^3
        \]

        \item Определим площадь на выходе из РК:
        \[
            A_{2a} = \pi D_2 l_2 = \pi \cdot 0.5268 \cdot 0.1874 =
            0.3102\ м^2
        \]

        \item Осевая составляющая абсолютной скорости на выходе из РК:
        \[
            c_{2a} = \frac{ G_{вх} }{ A_{2a} \rho_2 } =
            \frac{ 25.1 }{ 0.3102 \cdot 0.748 }
            = 108.11\ м/с
        \]

        \item Угол потока в относительном движении на выходе из РК:
        \[
            \beta_2 = \arcsin{ \frac{ c_{2a} }{ w_2 } } =
                    \arcsin{ \frac{ 108.11 }{ 379.26 } }
            = 16.563 \degree
        \]

        \item Осевая составляющая относительной скорости потока на выходе из РК:
        \[
            w_{2a} = w_2 \cdot \sin{\beta_2} =
                    379.26 \cdot \sin{16.563 \degree}
            = 108.11\ м/с
        \]

        \item Окружная составляющая относительной скорости потока на выходе из РК:
        \[
            w_{2u} = w_2 \cdot \cos{\beta_2} =
                    379.26 \cdot \cos{16.563 \degree}
            = 363.53\ м/с
        \]

        \item Определим окружную составляющую скорости на выходе из РК:
	    \[
            c_{2u} = w_{2u} - u_2 =
	        363.53 - 413.75 = -50.23 \/\ м/с
        \]

        \item Опеределим угол потока на выходе из РК:
        
        \[
            \alpha_2 = \pi + \arctan{ \frac{ c_{2a} }{ c_{2u} } } =
                    \pi + \arctan{ \frac{ 108.11 }{ -50.23 } } =
            114.918 \degree
        \]
        

        \item Определим скорость потока на выходе из РК:
	    \[
            c_2 = \sqrt{c_{2u}^2 + c_{2a}^2} =
                \sqrt{-50.23 ^ 2 + 108.11 ^ 2} =
            119.21 \/\ м/с
        \]

        \item Определим работу на окружности колеса:
	    \[
            L_u = c_{1u} u_1 + c_{2u} u_2 =
                    331.26 \cdot 402.32 +
                    -50.23 \cdot 413.75 =
            0.1125 \cdot 10^6 \/\ Дж/кг
        \]

        \item Определим КПД на окружности колеса:
	    \[
            \eta_u = \frac{L_u}{H_0} =
                \frac{ 0.1125 \cdot 10^6 }{ 0.1278 \cdot 10^6 }
            = 0.8805
        \]

        \item Определим удельные потери в СА:
	    \[
            h_с = \left(
                        \frac{ 1 }{ \phi^2 } - 1
                \right)
                \frac{ c_1^2 }{ 2 } =
	        \left(
                \frac{ 1 }{ 0.97 ^ 2} - 1
            \right) \cdot
            \frac{ 348.04 ^ 2 }{ 2 } = 3.8042 \cdot 10^3 \/\ Дж/кг
        \]

        \item Удельные потери в СА с учетом их использования в рабочих лопатках:
        \[
            h_с^\prime = h_с \frac{ T_2^\prime }{ T_1 } =
                3.8042 \cdot 10^3 \cdot
                \frac{ 1185.94 }{ 1237.16 } =
            3.6467 \cdot 10^3 \/\ Дж/кг
        \]

        \item Относительные потери в СА:
        \[
            \zeta_с = \frac{ h_с }{ H_0 } =
                \frac{ 3.8042 \cdot 10^3 }{ 0.1278 \cdot 10^6 } =
            0.0298
        \]

        \item Относительные потери в СА с учетом их использования в рабочих лопатках:
        \[
            \zeta_с^\prime = \frac{ h_с^\prime }{ H_0 } =
                \frac{ 3.6467 \cdot 10^3 }{ 0.1278 \cdot 10^6 } =
            0.0285
        \]

        \item Удельные потери в рабочих лопатках:
        \[
            h_л = \left(
                    \frac{ 1 }{ \psi^2 } - 1
                \right)) \cdot
                \frac{ w_2^2 }{ 2 } =
            \left(
                \frac{ 1 }{ 0.97 ^ 2 } - 1
            \right) \cdot
            \frac{ 379.26 ^ 2} {2}
            = 4.5175 \cdot 10^3 \/\ Дж/кг
        \]

        \item Относительные потери в рабочих лопатках:
        \[
            \zeta_л = \frac{ h_л }{ H_0 } =
                \frac{ 4.5175 \cdot 10^3 }{ 0.1278 \cdot 10^6 } =
            0.0354
        \]

        \item Определим удельные потери с выходной скоростью:
        \[
            h_{вых} = \frac{ c_2 ^ 2 }{ 2} =
                    \frac{ 119.21 ^ 2 }{ 2 } =  7.1056 \cdot 10^3 \/\ Дж/кг
        \]

        \item Относительные потери с выходной скоростью:
        \[
            \zeta_{вых} = \frac{ h_{вых} }{ H_0 } =
                \frac{ 7.1056 \cdot 10^3 }{ 0.1278 \cdot 10^6 } =
            0.0556
        \]

        \item Проверка КПД на окружности колеса:
        \[
            \eta_u = 1 - \zeta_с^\prime - \zeta_л - \zeta_{вых}
        \]

        \item Средний диаметр:
        \[
            D_{ср} = 0.5 \cdot (D_1 + D_2) =
                    0.5 \cdot (0.5122 + 0.5268) =
            0.5195\ м
        \]

        \item Определим удельные потери в радиальном зазоре:

	    \begin{gather*}
	        h_з = 1.37 \cdot
                \left(
                    1 + 1.6 \rho
                \right)
                \left(
                    1 + \frac{l_2}{D_{ср}}
                \right)
            \frac{ \delta_r }{ l_2 } \cdot L_u =\\
	        = 1.37 \cdot
            \left(
                1 + 1.6 \cdot 0.5
            \right)
            \left(
                1 + \frac{ 0.1874 }{ 0.5195 }
            \right)
            \frac{ 0.00187 }{ 0.1874 } \cdot
            0.1125 \cdot 10^6 =
	        3.7618 \cdot 10^3 \/\ Дж/кг\\
	    \end{gather*}

        \item Относительные удельные потери в радиальном зазоре:
        \[
            \zeta_з = \frac{ h_з }{ H_0 } =
                \frac{ 3.7618 \cdot 10^3 }{ 0.1278 \cdot 10^6 } =
            0.0294
        \]

        \item Удельная работа ступени с учетом потери в радиальном зазоре:
        \[
            L_{uз} = L_u - h_з = 0.1125 \cdot 10^6 -
                3.7618 \cdot 10^3 =
            0.1087 \cdot 10^6 \ Дж/кг
        \]

        \item Мощностной КПД ступени:
        \[
            \eta_т^\prime = \eta_u - \zeta_з =
                0.8805 - 0.0294 = 0.851
        \]

        \item Лопаточный КПД ступени:
        \[
            \eta_л^\prime = \eta_т^\prime + \zeta_{вых} =
                 0.851 +  0.0556 =
            0.9067
        \]

        \item Средняя длина лопатки:
        \[
            l_{ср} = 0.5 \cdot (l_1 + l_2) =
                0.5 \cdot (0.1691 + 0.1874) =
            0.1783\ м
        \]

        \item Средняя окружная скоротсь:
        \[
            u_{ср} = 0.5 \cdot (u_1 + u_2) =
                0.5 \cdot (402.32 + 413.75) =
            408.03\ м/с
        \]

        \item Затраты мощности на трение и вентиляцию:
        \begin{gather*}
            N_{т.в} = \left[
                    1.07 \cdot D_{av}^2 + 61 \cdot (1 - \varepsilon) \cdot D_{av} l_{av}
            \right] \cdot
            \left(
                \frac{ u_{av} }{ 100 }
            \right) ^ 3 \cdot
            \rho =\\
            = \left[
                1.07 \cdot 0.5195^2 +
                61 \cdot (1 - 1) \cdot
                0.5195 \cdot 0.1783
            \right] \cdot
            \left(
                \frac{ 408.03 }{ 100 }
            \right) ^ 3 \cdot
            0.4962
            = 0.0097 \cdot 10^3 \ Вт \\
        \end{gather*}

        \item Удельные потери на трение и вентиляцию:
        \[
            h_{т.в} = \frac{ N_{т.в} }{ G_{вх} } =
                \frac{
                    0.0097 \cdot 10^3
                }{
                    25.1
                }
            = 0.0004 \cdot 10^3 \ Дж/кг
        \]

        \item Относительные потери на трение и вентиляцию:
        \[
            \zeta_{т.в} = \frac{ h_{т.в} }{ H_0 } =
                \frac{ 0.0004 \cdot 10^3 }{ 0.1278 \cdot 10^6 } =
            0.0
        \]

        \item Мощностной КПД с учетом потерь на трению и вентиляцию:
        \[
            \eta_т = \eta_т^\prime - \zeta_{т.в} =
                0.851 - 0.0 =
            0.851
        \]

        \item Лопаточный КПД с учетом потерь на трению и вентиляцию:
        \[
            \eta_л = \eta_л^\prime - \zeta_{т.в} =
                0.9067 - 0.0 =
            0.9066
        \]

        

        \item Удельная работа ступени с учетом добавления охлаждающего воздуха:
        \[
            L_т = L_т^\prime \frac{ G_{вх} }{ G_т }  - L_т^\prime g_{охл} =
                0.1088 \cdot 10^6 \cdot
                \frac{ 25.1 }{ 25 }  -
                0.1088 \cdot 10^6 \cdot 0.003 =
            0.1096 \cdot 10^6 \ Дж/кг
        \]

        \item Статическая температура за ступенью:
        \[
            T_{ст} = T_2 + \frac{ h_з }{ c_{pг} } + \frac{ h_{т.в} }{ c_{pг} } =
                1189.58 +
                \frac{3.7618 \cdot 10^3 }{ 1240.65 } +
                \frac{ 0.0004 \cdot 10^3 }{ 1240.65 } =
            1192.61 \ К
        \]

        \item Температура торможения за турбиной:
        \[
            T_{ст}^* = T_{ст} + \frac{ h_{вых} }{ c_{pг} } =
                1192.61 +
                \frac{ 7.1056 \cdot 10^3 }{ 1240.65 } =
            1198.34 \ К
        \]

        \item Средняя теплоемкость газа в интервале температур от 273 К до $T_0^*$:
        \[
            c_{pг\ ср} (T_0^*, \alpha_{вх}) =
            1127.75 \ Дж/(кг \cdot К)
        \]

        \item Средняя теплоемкость газа в интервале температур от 273 К до $T_{ст}$:
        \[
            c_{pг\ ср} (T_{ст}, \alpha_{вх}) =
            1116.19 \ Дж/(кг \cdot К)
        \]

        \item Средняя теплоемкость газа в интервале температур от $T_0^*$ до $T_{ст}$:
        \begin{gather*}
            c_{pг}^\prime = \frac{
		        c_{pг\ ср} (T_0^*, \alpha_{вх}) (T_0^* - T_0) - c_{pг\ ср} (T_{ст}, \alpha_{вх})(T_{ст} - T_0)
		    }{
		        T_0^* - T_{ст}} =\\
            =\frac{
		        1127.75 \cdot
                (1285.98 - 273) -
		        1116.19 \cdot
                (1192.61 - 273)
		    }{
		        1285.98 - 1192.61} =
		    1241.64 \ Дж / (кг \cdot К)\\
        \end{gather*}

        \item Новое значение показателя адиабаты:
        \[
            k_г^\prime = \frac{c_{pг}^\prime}{c_{pг}^\prime - R_г} =
                \frac{
                    1241.64
                }{
                    1241.64 - 287.4
                }
            = 1.3012
        \]

        \item Невязка по коэффициенту адиабаты:
        \[
            \delta = \frac{ \left| k_г - k_г^\prime \right| }{ k_г } \cdot 100 \%=
                \frac{
                    \left| 1.3015 - 1.3012 \right|
                }{
                    1.3015
                } \cdot 100 \% =
            0.0241 \%
        \]

        \item Давление торможения на выходе из ступени:
        \[
            p_2^* = p_2 \left(
                            \frac{ T_{ст}^* }{ T_{ст} }
                    \right) ^ \frac{ k_г }{ k_г - 1 } =
                 0.2559 \cdot 10^6 \cdot \left(
                            \frac{ 1198.34 }{ 1192.61 }
                    \right) ^
                \frac{ 1.3015 }{ 1.3015 - 1 } =
            0.2612 \cdot 10^6 \ Па
        \]

        \item Теплоперепад по параметрам торможения:
        \[
            H_0^* = c_{pг} T_0^* \left[
                        1 - \left(
                                \frac{p_2^*}{p_0^*}
                            \right) ^
                        \frac{k_г - 1}{k_г}
                    \right] =
            1240.65 \cdot 1285.98 \cdot
                    \left[
                        1 - \left(
                                \frac{
                                    0.2612 \cdot 10^6
                                }{
                                    0.3669 \cdot 10^6
                                }
                            \right) ^
                        \frac{1.3015 - 1}{1.3015}
                    \right] =
            0.1207 \cdot 10^6 \ Дж/кг
        \]

        \item КПД по параметрам торможения:
        \[
            \eta_т^* = \frac{ L_т }{ H_0^* } =
                \frac{
                    0.1096 \cdot 10^6
                }{
                    0.1207 \cdot 10^6 } =
            0.9077
        \]

        \item Расход на выходе из ступени:
        \[
            G_{вых} = G_{вх} + G_т g_{охл} =
                25.1 + 25 \cdot
                0.003 =
            25.18 \ кг/с
        \]

        \item Относительный расход топлива на выходе из ступени:
        \[
            g_{топл.вых} = \frac{ G_{топл} }{ G_{вых} - G_{топл} } =
                 \frac{ 1 }{ 25.18 - 1 } =
            0.0414
        \]

        \item Коэффициент избытка воздуха на выходе из ступени:
        \[
            \alpha_{вых} = \frac{ 1 }{ l_0 g_{топл.вых} } =
                \frac{ 1 }{ 14.61 \cdot 0.0414 } =
            1.655
        \]

        \item Абсолютный расход охлаждающего воздуха:
        \[
            G_{охл} = G_т g_{охл} = 25 \cdot 0.003 =
            0.075
        \]

        \item Определим температуру торможения на выходе из турбины после подмешивания охлаждающего воздуха.
        \begin{enumerate}

            \item Истинная теплоемкость охлаждающего воздуха при температуре $T_{охл} = 700\ К $:
            \[
                c_{pв} (T_{охл}) = 1074.86\ Дж/ (кг \cdot К)
            \]

            \item Истинная теплоемкость газа при температуре $T_{ст}^* = 1198.34 \ К $:
            \[
                c_{pг} (T_{ст}^*, \alpha_{вх}) =
                1263.48\ Дж/ (кг \cdot К)
            \]

            \item Значение температуры смеси с предпоследней итерации $T_{см}^{*} = 1196.8\ К$.

            \item Истинная теплоемкость смеси:
            \[
                c_{pг} (T_{см}^{*}, \alpha_{вых}) =
                1262.92\ Дж/ (кг \cdot К)
            \]

            \item Новое значение температуры смеси:
            \[
                T_{см}^*\prime = \frac{
                        c_{pг} (T_{ст}^*, \alpha_{вх}) T_{ст}^* G_{вх} + c_{pв} (T_{охл}) T_{охл} G_{охл}
                    }{
                        c_{pг} (T_{см}^{*}, \alpha_{вых}) G_{вых}
                    } =
                \frac{
                    1263.48
                    \cdot 1198.34 \cdot 25.1 +
                    1074.86
                    \cdot 700 \cdot 0.075
                }{
                    1262.92
                    \cdot  25.18
                } =
                1197.08\ К
            \]

            \item Значение невязки:
            \[
                \delta = \frac{ \left| T_{см}^{*} - T_{см}^*\prime \right| }{T_{см}^{*}} \cdot 100 \% =
                    \frac{
                        \left| 1196.8 - 1197.08 \right|
                    }{
                        1196.8
                    } \cdot 100 \% =
                0.023 \%
            \]

        \end{enumerate}

        

        \item Невязка по мощностному КПД:
        \[
            \delta_\eta = \frac{ \left| \eta_t - \eta_т^\prime \right| }{ \eta_т^\prime } \cdot 100 \% =
                \frac{
                    \left| 0.851 - 0.8516 \right|
                }{
                    0.8516 } \cdot 100 \% =
            0.07 \%
        \]

    \end{enumerate}
     

    \subsection{Расчет интегральных параметров турбины}

    

    
    \begin{enumerate}

        \item Суммарная работа всех ступеней:
        \[
            L_{т\Sigma} = 0.1418\cdot 10^6+0.1096\cdot 10^6 = 0.2514 \cdot 10^6 \ Дж/кг
        \]

        \item Средняя теплоемкость газа в интервале температур от 273 К до $T_г^*$:
        \[
            c_{pг\ ср} (T_г^*, \alpha_{вх}) =
            1142.12 \ Дж/(кг \cdot К)
        \]

        \item Средняя теплоемкость газа в интервале температур от 273 К до $T_т$:
        \[
            c_{pг\ ср} (T_т, \alpha_{вх}) =
            1116.41 \ Дж/(кг \cdot К)
        \]

        \item Средняя теплоемкость газа в интервале температур от $T_0^*$ до $T_т$:
        \begin{gather*}
            c_{pг} = \frac{
		         c_{pг\ ср} (T_г^*, \alpha_{вх}) (T_г^* - T_0) - c_{pг\ ср} (T_{т}, \alpha_{вх})(T_т - T_0)
		    }{
		        T_г^* - T_т} =\\
            =\frac{
                1142.12 \cdot
                (1400 - 273) -
		        1116.41 \cdot
                (1192.61 - 273)
		    }{
		        1400 - 1192.61} =
		    1256.13 \ Дж / (кг \cdot К)\\
        \end{gather*}

        \item Средний показателя адиабаты:
        \[
            k_г = \frac{c_{pг}}{c_{pг} - R_г} =
                \frac{
                    1256.13
                }{
                    1256.13 - 287.4
                }
            = 1.2967
        \]

        \item Статический теплоперепад на турбине:
        \[
            H_т = c_{pг} T_г^* \left[
                        1 - \left(
                                \frac{p_г^*}{p_т} ^
                                \frac{1 - k_г}{k_г}
                    \right)
                \right] =
            1256.13 \cdot 1400
                \left[
                    1 - \left(
                            \frac{
                                0.55 \cdot 10^6
                            }{
                                0.2559 \cdot 10^6 } ^
                            \frac{ 1 - 1.2967 }{ 1.2967 }
                    \right)
            \right] =
            0.2824 \cdot 10^6 \ Дж/кг
        \]

        \item Средняя теплоемкость газа в интервале температур от 273 К до $T_т^*$:
        \[
            c_{pг\ ср} (T_т^*, \alpha_{вх}) =
            1116.93 \ Дж/(кг \cdot К)
        \]

        \item Средняя теплоемкость газа в интервале температур от $T_0^*$ до $T_т^*$:
        \begin{gather*}
            c_{pг}^* = \frac{
		         c_{pг\ ср} (T_г^*, \alpha_{вх}) (T_г^* - T_0) - c_{pг\ ср} (T_т^*, \alpha_{вх})(T_т^* - T_0)
		    }{
		        T_г^* - T_т^*} =\\
            =\frac{
                1142.12 \cdot
                (1400 - 273) -
		        1116.93 \cdot
                (1196.8 - 273)
		    }{
		        1400 - 1196.8} =
		    1256.65 \ Дж / (кг \cdot К)\\
        \end{gather*}

        \item Средний показателя адиабаты по параметрам торможения:
        \[
            k_г^* = \frac{ c_{pг}^* }{ c_{pг}^* - R_г } =
                \frac{
                    1256.65
                }{
                    1256.65 - 287.4
                }
            = 1.2965
        \]

        \item Теплоерепад на турбине оп параметрам торможения:
        \[
             H_т^* = c_{pг}^* T_г^* \left[
                        1 - \left(
                                \frac{p_г^*}{p_т} ^
                                \frac{1 - k_г^*}{k_г^*}
                    \right)
                \right] =
            1256.65 \cdot 1400
                \left[
                    1 - \left(
                            \frac{
                                0.55 \cdot 10^6
                            }{
                                0.2559 \cdot 10^6 } ^
                            \frac{ 1 - 1.2965 }{ 1.2965 }
                    \right)
            \right] =
            0.2754 \cdot 10^6 \ Дж/кг
        \]

        \item Мощностной КПД турбины:
        \[
            \eta_т = \frac{ L_{т\Sigma} }{ H_т } =
                \frac{ 0.2514 \cdot 10^6 }{ 0.2824 \cdot 10^6 } =
            0.8902
        \]

        \item Лопаточный КПД турбины:
        \[
            \eta_л = \frac{
                        L_{т\Sigma} + 0.5 \cdot c_{вых}^2
                    }{ H_т } =
            \frac{
                0.2514 \cdot 10^6 + 0.5 \cdot 119.21 ^ 2
            }{ 0.2824 \cdot 10^6 } =
            0.9153
        \]

        \item КПД турбины по параметрам торможения:
        \[
            \eta_т^* = \frac{ L_{т\Sigma} }{ H_т^* } =
                \frac{ 0.2514 \cdot 10^6 }{ 0.2754 \cdot 10^6 } =
            0.9128
        \]

    \end{enumerate}
    

\end{document}