%%
%% Author: User1
%% 09.04.2018
%%

% Preamble
\documentclass[a4paper,10pt]{article}

% Packages
\usepackage{mathtext}
\usepackage[T2A]{fontenc}
\usepackage[utf8]{inputenc}
\usepackage{longtable}
\usepackage[russian]{babel}
\usepackage{amsmath}
\usepackage{amsfonts}
\usepackage{amssymb}
\usepackage{graphicx}
\usepackage[left=2cm,right=2cm,
    top=2cm,bottom=2cm,bindingoffset=0cm]{geometry}
\usepackage{color}
\usepackage{gensymb}

\usepackage{enumitem}
\setlist[enumerate]{label*=\arabic*.}

\usepackage{indentfirst}

\usepackage{titlesec}


% Document
\begin{document}

\section{Расчет охлаждения СА первой ступени.}

    </ import 'turb_cooling.tex' as cool />

    \subsection{Описание алгоритма расчета температурного поля лопатки.}
    << cool.description() >>

    \subsection{Алгоритм расчета пленки.}

    << cool.film() >>

    \subsection{Алгоритм температуры водуха.}

    << cool.air_temp() >>

    \subsection{Методика расчета температуры стенки.}
    << cool.wall_temp() >>

    \subsection{Алгоритм расчета коэффициент теплоотдачи со стороны газа.}

    << cool.alpha_gas() >>

    \subsection{Результаты расчета.}

    << cool.sector_params(film_params, caption='Значения параметров в отверстиях.') >>

    << cool.sector_params(local_params, caption='Значения локальных параметров.') >>

    \subsection{Интегральные параметры охлаждения.}

    << cool.integrate_params(cooling_results, blade_num, G_comp) >>

\end{document}