%%
%% Author: User1
%% 13.12.2017
%%

% Preamble
\documentclass[a4paper,10pt]{article}

% Packages
\usepackage{mathtext}
\usepackage[T2A]{fontenc}
\usepackage[utf8]{inputenc}
\usepackage{longtable}
\usepackage[russian]{babel}
\usepackage{amsmath}
\usepackage{amsfonts}
\usepackage{amssymb}
\usepackage{graphicx}
\usepackage[left=2cm,right=2cm,
    top=2cm,bottom=2cm,bindingoffset=0cm]{geometry}
\usepackage{color}
\usepackage{gensymb}

\usepackage{enumitem}
\setlist[enumerate]{label*=\arabic*.}

\usepackage{indentfirst}

\usepackage{titlesec}


% Document
\begin{document}

    \section{Расчет турбины по средней линии тока}

    
    

    \subsection{Расчет первой ступени}

    \subsubsection{Исходные данные}

    
    \begin{enumerate}

        \item Температура торможения на входе в ступень: $T_0^* = 1400\ К $.
        \item Давление торможения на входе в ступень: $p_0^* = 0.55 \cdot 10^6 \ Па$.
        \item Температура торможения на входе в ступень при адиабатическом процессе в турбине: $T_{0ад\ т}^* = 1400\ К$
        \item Расход газа на входе в ступень: $G_{вх} = 25\ кг/с$.
        \item Расход газа на входе в СА первой ступени: $ G_т = 25\ кг/с $.
        \item Расход топлива на входе в турбину: $ G_{топл} = 1\ кг/с $.
        \item Степень реактивности: $ \rho = 0.4 $.
        \item Коэффициент скорости в СА: $ \phi = 0.97 $.
        \item Коэффициент скорости в РК: $ \psi = 0.97 $.
        \item Длина лопатки на входе в РК: $ l_1 = 0.1204\ м $.
        \item Длина лопатки на выходе из РК: $ l_2 = 0.1334\ м $.
        \item Средний диаметр на входе в РК: $ D_1 = 0.4816\ м $.
        \item Средний диаметр на выходе в РК: $ D_2 = 0.4919\ м $.
        \item Радиальный зазор: $ \delta_r = 0.00133\ м $.
        \item Частота вращения ротора: $ n = 15000.0\ об/мин $
        \item Степень парциальности: $ \varepsilon = 1 $.
        \item Расход охлаждающего воздуха, отнесенный к расходу на входе в турбину: $ g_{охл} = 0.004 $.
        \item Температура торможения охлаждающего воздуха: $ T_{охл} = 700\ К $.

        
        \item Статический теплоперепад на ступени: $ H_0 = 0.1637 \cdot 10^6 \ Дж/кг $.

        

    \end{enumerate}
    

    \subsubsection{Расчет}

    
    \begin{enumerate}

        \item Относительный расход топлива на входе в ступень:
        \[
            g_{топл.вх} = \frac{ G_{топл} }{ G_{вх} - G_{топл} } =
                \frac{ 1 }{ 25 - 1 } =
            0.0417
        \]

        \item Коэффициент избытка воздуха на входе:
        \[
            \alpha_{вх} = \frac{ 1 }{ l_0 g_{топл.вх} } =
                \frac{ 1 }{ 14.61 \cdot 0.0417 } =
            1.643
        \]

        \item Средний в ступени коэффициент адиабаты из предпоследней итерации:
        \[
            k_г = 1.2935
        \]

        \item Средняя в ступени теплоемкость газа из предпоследней итерации:
        \[
            c_{pг} = 1266.68 \ Дж/(кг \cdot К)
        \]

        \item Средний в ступени коэффициент адиабаты при адиабатическом расширении в турбине до статических параметров из предпоследней итерации:
        \[
            k_{г\ ад\ т} = 1.2942
        \]

        \item Средняя в ступени теплоемкость при адиабатическом расширении в турбине до статических параметров из предпоследней итерации:
        \[
            c_{pг\ ад\ т} = 1264.3 \ Дж/(кг \cdot К)
        \]

        \item Средний в ступени коэффициент адиабаты при адиабатическом расширении в турбине до параметров торможения из предпоследней итерации:
        \[
            k_{г\ ад\ т}^* = 1.2939
        \]

        \item Средняя в ступени теплоемкость при адиабатическом расширении в турбине до параметров торможения из предпоследней итерации:
        \[
            c_{pг\ ад\ т}^* = 1265.15 \ Дж/(кг \cdot К)
        \]

        
        

        

        \item Определим теплоперепад на сопловом аппарате:

        \[
            H_с = \left( 1 - \rho \right) H_0 =
	        \left( 1 - 0.4 \right) \cdot 0.1637 \cdot 10^6 =
            0.0982 \cdot 10^6 \/\ Дж/кг
        \]

        \item Окружная скорость на диаметре $ D_1 $:

        \[
            u_1 = \frac{\pi D_1 n }{60} =
                \frac{\pi \cdot 0.4816 \cdot 15000.0}{60} =
            378.22\ м/с
        \]

        \item Определим действительную скорость истечения из СА:

	    \[
            c_1 = \phi \sqrt{2 H_с} =
	        0.97 \cdot\sqrt{2 \cdot 0.1637 \cdot 10^6}  =
            429.87 \/\ м/с
        \]

        \item Определим температуру на выходе из СА:

	    \[
            T_1 = T_0^* - \frac{ H_с \phi^2 }{ c_{pг} } =
	        1400 -
            \frac{
                0.0982 \cdot 10^6 \cdot {0.97}^2
            }{
                2 \cdot 1266.68
            } = 1327.06 \/\ К
        \]

	    \item Определим температуру конца адиабатного расширения:

	    \[
            T_1^\prime = T_0^* - \frac{ H_c }{ c_{pг} } =
	        1400 -
            \frac{
                0.0982 \cdot 10^6
            }{
                1266.68
            }
            = 1322.48  \/\ К
        \]

        \item Определим давление на выходе из СА:

	    \[
            p_1 = p_0^* \left(
                                \frac{ T_1^\prime }{ T_0^* }
                        \right)^
                    \frac{ k_г }{ k_г - 1 } =
            0.55 \cdot 10^6 \cdot
                \left(
                        \frac{ 1322.48 }{ 1400 }
                \right)^
                \frac{ 1.2935 }{ 1.2935 - 1 } =
            0.4279 \cdot 10^6 \/\ МПа
        \]

        \item Определим площадь на выходе из СА:

	    \[
            A_{1a} = \pi l_1 D_1 =
	        \pi \cdot 0.1204 \cdot 0.4816 =
            0.18214 \/\ м^2
        \]

        \item Определим плотность газа на выходе из СА:

	    \[
            \rho_1 = \frac{p_1}{R_г T_1} =
	        \frac{
                0.4279 \cdot 10^6
            }{
                287.4 \cdot 1327.06
            } =
            1.122 \/\ кг/м^3
        \]

        \item Осевая составляющая абсолютной скорости на выходе из СА:

        \[
            c_{1a} = \frac{G_{вх} }{ \rho_1 A_{1a} } =
                \frac{
                    25
                }{
                    1.122 \cdot 0.18214
                } =
            122.35\ м/с
        \]

        \item Угол потока в абсолютном движении после СА:

        \[
            \alpha_1 = \arcsin{ \frac{ c_{1a} }{ c_1 } } =
            \arcsin{ \frac{ 122.35 }{ 429.87 } } =
            = 16.536 \degree
        \]

        \item Окружная составляющая абсолютной скорости на входе:

        \[
            c_{1u} = c_1 \cos{\alpha_1} = 429.87 \cdot \cos{16.536 \degree} =
            412.09\ м/с
        \]

        \item Определим относительную скорость на входе в РК:

	    \[
	        w_1 = \sqrt{c_1^2 + u_1^2 - 2 c_1 u_1 \cos \alpha_1} =
        \]
	    \[    = \sqrt{
            429.87 ^ 2 +
            378.22 ^ 2 -
            2 \cdot 429.87 \cdot 378.22 \cdot \cos 16.16.536 \degree
            }
            = 126.95 \/\ м/с
        \]

        \item Угол потока в относительном движении:

        
        \[
            \beta_1 = \arctan{ \frac{c_{1a}}{c_{1u} - u_1} } =
                    \arctan{ \frac{ 122.35 }{412.09 - 378.22} } =
            74.524 \degree
        \]
        

        \item Осевая составляющая относительной скорости:

        \[
            w_{1a} = w_1 \sin{\beta_1} = 126.95 \cdot  \sin{74.524 \degree} =
            122.35\ м/с
        \]

        \item Окружная составляющая относительной скорости:

        \[
            w_{1u} = w_1 \cos{\beta_1} = 126.95 \cdot  \cos{74.524 \degree} =
            33.87\ м/с
        \]

         \item Определим теплоперепад на РК:

	    \[
            H_л = H_0 \rho \frac{T_1}{T_1^\prime} =
	        0.1637 \cdot 10^6 \cdot 0.4 \cdot
            \frac{ 1327.06 }{ 1322.48 } =
            0.0657 \cdot 10^6 \/\ Дж/кг
        \]

        \item Окружная скорость на диаметре:

        \[
            u_2 = \frac{ \pi D_2 n }{ 60 } =
                    \frac{ \pi \cdot 0.4919 \cdot 15000.0 }{ 60 } =
            386.36\ м/с
        \]

        \item Температура торможения в относительном движении после СА:

        \[
            T_{1w}^* = T_1 + \frac{ w_1^2 }{ 2 \cdot c_{pг}} =
                1327.06 + \frac{ 126.95 ^ 2 }{ 2 \cdot 1266.68}
        \]

        \item Определим относительную скорость истечения газа из РК:

	    \[
	        w_2 = \psi \sqrt{w_1^2 + 2H_л +\left( u_2^2 - u_1^2 \right)} =
        \]
        \[   = 0.97 \cdot
            \sqrt{
                126.95 ^ 2 +
                2 \cdot 0.0657 \cdot 10^6 +
                \left( 386.36 ^ 2 - 378.22 ^ 2 \right)
            } =
            380.32 \/\ м/с
        \]


        \item Определим статическую температуру на выходе из РК:

	    \[
	        T_2 = T_1 + \frac{
	 	        \left( w_1^2  - w_2^2 \right) + \left( u_2^2 - u_1^2 \right)
            }{
                2 c_{pг}
            } =
        \]
	    \[
            = 1327.06 + \frac{
	 	        \left( 126.95 ^ 2  - 380.32 ^ 2 \right) +
                \left( 386.36 ^ 2 - 378.22 ^ 2 \right)
	        }{
            2 \cdot 1266.68
            }
            = 1278.78 \/\ К
        \]

        \item Определим статическую температуру при адиабатическом процессе в РК:

	    \[
            T_2^\prime = T_1 - \frac{
	 	        H_л
	        }{ c_{p г}} =
	        1327.06 - \frac{
	 	        0.0657 \cdot 10^6
	        }{
                1266.68
            }
            = 1275.2 \/\ К
        \]

        \item Определим давление на выходе из РК:

	    \[
            p_2 = p_1 \left( \frac{T_2^\prime}{T_1} \right)^{\frac{k_г}{k_г - 1}} =
               0.4279 \cdot 10^6 \cdot
               \left(
               \frac{ 1275.2 }{ 1327.06 }
               \right) ^
               {\frac{
               1.293
               }{
               1.293 - 1
               }}
            = 0.3589 \cdot 10^6 \/\ Па
        \]

        \item Определим плотность газа на выходе из РК:
	    \[
            \rho_2 = \frac{p_2}{R T_2} =
                \frac{
                    0.3589 \cdot 10^6
                }{
                    287.4 \cdot 1278.78
                }
            = 0.977\ кг/м^3
        \]

        \item Определим площадь на выходе из РК:
        \[
            A_{2a} = \pi D_2 l_2 = \pi \cdot 0.4919 \cdot 0.1334 =
            0.2062\ м^2
        \]

        \item Осевая составляющая абсолютной скорости на выходе из РК:
        \[
            c_{2a} = \frac{ G_{вх} }{ A_{2a} \rho_2 } =
            \frac{ 25 }{ 0.2062 \cdot 0.977 }
            = 124.14\ м/с
        \]

        \item Угол потока в относительном движении на выходе из РК:
        \[
            \beta_2 = \arcsin{ \frac{ c_{2a} }{ w_2 } } =
                    \arcsin{ \frac{ 124.14 }{ 380.32 } }
            = 19.051 \degree
        \]

        \item Осевая составляющая относительной скорости потока на выходе из РК:
        \[
            w_{2a} = w_2 \cdot \sin{\beta_2} =
                    380.32 \cdot \sin{19.051 \degree}
            = 124.14\ м/с
        \]

        \item Окружная составляющая относительной скорости потока на выходе из РК:
        \[
            w_{2u} = w_2 \cdot \cos{\beta_2} =
                    380.32 \cdot \cos{19.051 \degree}
            = 359.49\ м/с
        \]

        \item Определим окружную составляющую скорости на выходе из РК:
	    \[
            c_{2u} = w_{2u} - u_2 =
	        359.49 - 386.36 = -26.87 \/\ м/с
        \]

        \item Опеределим угол потока на выходе из РК:
        
        \[
            \alpha_2 = \pi + \arctan{ \frac{ c_{2a} }{ c_{2u} } } =
                    \pi + \arctan{ \frac{ 124.14 }{ -26.87 } } =
            102.215 \degree
        \]
        

        \item Определим скорость потока на выходе из РК:
	    \[
            c_2 = \sqrt{c_{2u}^2 + c_{2a}^2} =
                \sqrt{-26.87 ^ 2 + 124.14 ^ 2} =
            127.02 \/\ м/с
        \]

        \item Определим работу на окружности колеса:
	    \[
            L_u = c_{1u} u_1 + c_{2u} u_2 =
                    412.09 \cdot 378.22 +
                    -26.87 \cdot 386.36 =
            0.1455 \cdot 10^6 \/\ Дж/кг
        \]

        \item Определим КПД на окружности колеса:
	    \[
            \eta_u = \frac{L_u}{H_0} =
                \frac{ 0.1455 \cdot 10^6 }{ 0.1637 \cdot 10^6 }
            = 0.8889
        \]

        \item Определим удельные потери в СА:
	    \[
            h_с = \left(
                        \frac{ 1 }{ \phi^2 } - 1
                \right)
                \frac{ c_1^2 }{ 2 } =
	        \left(
                \frac{ 1 }{ 0.97 ^ 2} - 1
            \right) \cdot
            \frac{ 429.87 ^ 2 }{ 2 } = 5.8036 \cdot 10^3 \/\ Дж/кг
        \]

        \item Удельные потери в СА с учетом их использования в рабочих лопатках:
        \[
            h_с^\prime = h_с \frac{ T_2^\prime }{ T_1 } =
                5.8036 \cdot 10^3 \cdot
                \frac{ 1275.2 }{ 1327.06 } =
            5.5768 \cdot 10^3 \/\ Дж/кг
        \]

        \item Относительные потери в СА:
        \[
            \zeta_с = \frac{ h_с }{ H_0 } =
                \frac{ 5.8036 \cdot 10^3 }{ 0.1637 \cdot 10^6 } =
            0.0355
        \]

        \item Относительные потери в СА с учетом их использования в рабочих лопатках:
        \[
            \zeta_с^\prime = \frac{ h_с^\prime }{ H_0 } =
                \frac{ 5.5768 \cdot 10^3 }{ 0.1637 \cdot 10^6 } =
            0.0341
        \]

        \item Удельные потери в рабочих лопатках:
        \[
            h_л = \left(
                    \frac{ 1 }{ \psi^2 } - 1
                \right)) \cdot
                \frac{ w_2^2 }{ 2 } =
            \left(
                \frac{ 1 }{ 0.97 ^ 2 } - 1
            \right) \cdot
            \frac{ 380.32 ^ 2} {2}
            = 4.5426 \cdot 10^3 \/\ Дж/кг
        \]

        \item Относительные потери в рабочих лопатках:
        \[
            \zeta_л = \frac{ h_л }{ H_0 } =
                \frac{ 4.5426 \cdot 10^3 }{ 0.1637 \cdot 10^6 } =
            0.0278
        \]

        \item Определим удельные потери с выходной скоростью:
        \[
            h_{вых} = \frac{ c_2 ^ 2 }{ 2} =
                    \frac{ 127.02 ^ 2 }{ 2 } =  8.0667 \cdot 10^3 \/\ Дж/кг
        \]

        \item Относительные потери с выходной скоростью:
        \[
            \zeta_{вых} = \frac{ h_{вых} }{ H_0 } =
                \frac{ 8.0667 \cdot 10^3 }{ 0.1637 \cdot 10^6 } =
            0.0493
        \]

        \item Проверка КПД на окружности колеса:
        \[
            \eta_u = 1 - \zeta_с^\prime - \zeta_л - \zeta_{вых} = 1 - 0.0341 -
                    0.0278 - 0.0493 = 0.8889
        \]

        \item Средний диаметр:
        \[
            D_{ср} = 0.5 \cdot (D_1 + D_2) =
                    0.5 \cdot (0.4816 + 0.4919) =
            0.4867\ м
        \]

        \item Определим удельные потери в радиальном зазоре:

	    \[
	        h_з = 1.37 \cdot
                \left(
                    1 + 1.6 \rho
                \right)
                \left(
                    1 + \frac{l_2}{D_{ср}}
                \right)
            \frac{ \delta_r }{ l_2 } \cdot L_u =\]
        \[   = 1.37 \cdot
            \left(
                1 + 1.6 \cdot 0.4
            \right)
            \left(
                1 + \frac{ 0.1334 }{ 0.4867 }
            \right)
            \frac{ 0.00133 }{ 0.1334 } \cdot
            0.1455 \cdot 10^6 =
	        4.1646 \cdot 10^3 \/\ Дж/кг\]


        \item Относительные удельные потери в радиальном зазоре:
        \[
            \zeta_з = \frac{ h_з }{ H_0 } =
                \frac{ 4.1646 \cdot 10^3 }{ 0.1637 \cdot 10^6 } =
            0.0254
        \]

        \item Удельная работа ступени с учетом потери в радиальном зазоре:
        \[
            L_{uз} = L_u - h_з = 0.1455 \cdot 10^6 -
                4.1646 \cdot 10^3 =
            0.1413 \cdot 10^6 \ Дж/кг
        \]

        \item Мощностной КПД ступени:
        \[
            \eta_т^\prime = \eta_u - \zeta_з =
                0.8889 - 0.0254 = 0.8634
        \]

        \item Лопаточный КПД ступени:
        \[
            \eta_л^\prime = \eta_т^\prime + \zeta_{вых} =
                 0.8634 +  0.0493 =
            0.9127
        \]

        \item Средняя длина лопатки:
        \[
            l_{ср} = 0.5 \cdot (l_1 + l_2) =
                0.5 \cdot (0.1204 + 0.1334) =
            0.1269\ м
        \]

        \item Средняя окружная скоротсь:
        \[
            u_{ср} = 0.5 \cdot (u_1 + u_2) =
                0.5 \cdot (378.22 + 386.36) =
            382.29\ м/с
        \]

        \item Затраты мощности на трение и вентиляцию:
        \[
            N_{т.в} = \left[
                    1.07 \cdot D_{av}^2 + 61 \cdot (1 - \varepsilon) \cdot D_{av} l_{av}
            \right] \cdot
            \left(
                \frac{ u_{av} }{ 100 }
            \right) ^ 3 \cdot
            \rho =
        \]
        \[    = \left[
                1.07 \cdot 0.4867^2 +
                61 \cdot (1 - 1) \cdot
                0.4867 \cdot 0.1269
            \right] \cdot
            \left(
                \frac{ 382.29 }{ 100 }
            \right) ^ 3 \cdot
            0.4=
        \]
        \[
            = 0.0057 \cdot 10^3 \ Вт
        \]

        \item Удельные потери на трение и вентиляцию:
        \[
            h_{т.в} = \frac{ N_{т.в} }{ G_{вх} } =
                \frac{
                    0.0057 \cdot 10^3
                }{
                    25
                }
            = 0.0002 \cdot 10^3 \ Дж/кг
        \]

        \item Относительные потери на трение и вентиляцию:
        \[
            \zeta_{т.в} = \frac{ h_{т.в} }{ H_0 } =
                \frac{ 0.0002 \cdot 10^3 }{ 0.1637 \cdot 10^6 } =
            0.0
        \]

        \item Мощностной КПД с учетом потерь на трению и вентиляцию:
        \[
            \eta_т = \eta_т^\prime - \zeta_{т.в} =
                0.8634 - 0.0 =
            0.8634
        \]

        \item Лопаточный КПД с учетом потерь на трению и вентиляцию:
        \[
            \eta_л = \eta_л^\prime - \zeta_{т.в} =
                0.9127 - 0.0 =
            0.9127
        \]

        \item Определим удельную работу ступени:
        \[
            L_т = H_0 \eta_т = 0.1637 \cdot 10^6 \cdot 0.8634 =
            = 0.1413 \cdot 10^6 \ Дж/кг
        \]

        \item Удельная работа ступени, отнесенная к расходу на в СА первой ступени:
        \[
            L_т^\prime = L_т \frac{ G_{вх} }{ G_т }  =
                0.1413 \cdot 10^6 \cdot
                \frac{ 25 }{ 25 } =
            0.1413 \cdot 10^6 \ Дж/кг
        \]

        \item Статическая температура за ступенью:
        \[
            T_{ст} = T_2 + \frac{ h_з }{ c_{pг} } + \frac{ h_{т.в} }{ c_{pг} } =
                1278.78 +
                \frac{4.1646 \cdot 10^3 }{ 1266.68 } +
                \frac{ 0.0002 \cdot 10^3 }{ 1266.68 } =
            1282.07 \ К
        \]

        \item Температура торможения за турбиной:
        \[
            T_{ст}^* = T_{ст} + \frac{ h_{вых} }{ c_{pг} } =
                1282.07 +
                \frac{ 8.0667 \cdot 10^3 }{ 1266.68 } =
            1288.44 \ К
        \]

        \item Средняя теплоемкость газа в интервале температур от 273 К до $T_0^*$:
        \[
            c_{pг\ ср} (T_0^*, \alpha_{вх}) =
            1142.12 \ Дж/(кг \cdot К)
        \]

        \item Средняя теплоемкость газа в интервале температур от 273 К до $T_{ст}^*$:
        \[
            c_{pг\ ср} (T_{ст}^*, \alpha_{вх}) =
            1128.29 \ Дж/(кг \cdot К)
        \]

        \item Средняя теплоемкость газа в интервале температур от $T_0^*$ до $T_{ст}^*$:
        \[
            c_{pг}^\prime = \frac{
		        c_{pг\ ср} (T_0^*, \alpha_{вх}) (T_0^* - T_0) - c_{pг\ ср} (T_{ст}^*, \alpha_{вх})(T_{ст}^* - T_0)
		    }{
		        T_0^* - T_{ст}^*} =\]
        \[    =\frac{
		        1142.12 \cdot
                (1400 - 273) -
		        1128.29 \cdot
                (1288.44 - 273)
		    }{
		        1400 - 1288.44} =
		    1268.01 \ Дж / (кг \cdot К)
        \]

        \item Новое значение показателя адиабаты:
        \[
            k_г^\prime = \frac{c_{pг}^\prime}{c_{pг}^\prime - R_г} =
                \frac{
                    1268.01
                }{
                    1268.01 - 287.4
                }
            = 1.2931
        \]

        \item Невязка по коэффициенту адиабаты:
        \[
            \delta = \frac{ \left| k_г - k_г^\prime \right| }{ k_г } \cdot 100 \%=
                \frac{
                    \left| 1.2935 - 1.2931 \right|
                }{
                    1.2935
                } \cdot 100 \% =
            0.0306 \%
        \]

        \item Давление торможения на выходе из ступени:
        \[
            p_2^* = p_2 \left(
                            \frac{ T_{ст}^* }{ T_{ст} }
                    \right) ^ \frac{ k_г }{ k_г - 1 } =
                 0.3589 \cdot 10^6 \cdot \left(
                            \frac{ 1288.44 }{ 1282.07 }
                    \right) ^
                \frac{ 1.2935 }{ 1.2935 - 1 } =
            0.3669 \cdot 10^6 \ Па
        \]

        \item Статическая температура на выходе из ступени при адиабатическом процессе в турбине:
        \[
            T_{2ад\ т} = T_{0ад\ т}^* \cdot \frac{p_2}{p_0^*} ^ {
                    \frac{k_{г\ ад\ т} - 1}{k_{г\ ад\ т} }
            } = 1400 \cdot
            \frac{ 0.3589
            }{
            0.55
            } ^ {
                    \frac{1.294 - 1}{1.294}
            } =
            1270.56\ К
        \]

        \item Полная температура на выходе из ступени при адиабатическом процессе в турбине:
        \[
            T_{2ад\ т}^* = T_{0ад\ т}^* \cdot \frac{p_2^*}{p_0^*} ^ {
                    \frac{k_{г\ ад\ т}^* - 1}{k_{г\ ад\ т}^*}
            } = 1400 \cdot
            \frac{ 0.3669
            }{
            0.55
            } ^ {
                    \frac{1.294 - 1}{1.294}
            } =
            1276.96\ К
        \]

        \item Статический теплоперепад при адиабатическом процессе в турбине:
        \[
            H_{0ад\ т} = c_{pг\ ад\ т} \cdot \left(
            T_{0ад\ т}^* - T_{2ад\ т}
            \right) =
            1264.3 \cdot \left(
            1400 - 1270.56
            \right) =
            0.1637 \cdot 10^6 \ Дж/кг
        \]

        \item Статический теплоперепад при адиабатическом процессе в турбине, отнесенный к расходу на входе:
        \[
            H_{0ад\ т}^\prime = H_{0ад\ т} \cdot \frac{ G_{вх} }{ G_т }  =
                0.1637 \cdot 10^6 \cdot
                \frac{ 25 }{ 25 } =
            0.1637 \cdot 10^6 \ Дж/кг
        \]

        \item Теплоперепад по параметрам торможения при адиабатическом процессе в турбине:
        \[
            H_{0ад\ т}^* = c_{pг\ ад\ т}^* \cdot \left(
            T_{0ад\ т}^* - T_{2ад\ т}^*
            \right) =
            1265.15 \cdot \left(
            1400 - 1276.96
            \right) =
            0.1557 \cdot 10^6 \ Дж/кг
        \]

        \item Теплоперепад по параметрам торможения при адиабатическом процессе в турбине, отнесенный к расходу на входе:
        \[
            H_{0ад\ т}^{*\prime} = H_{0ад\ т}^* \cdot \frac{ G_{вх} }{ G_т }  =
                0.1557 \cdot 10^6 \cdot
                \frac{ 25 }{ 25 } =
            0.1557 \cdot 10^6 \ Дж/кг
        \]

        \item Средняя теплоемкость газа в интервале температур от 273 К до $T_{0ад\ т}^*$:
        \[
            c_{pг\ ср} (T_{0ад\ т}^*, \alpha_{вх}) =
            1142.12 \ Дж/(кг \cdot К)
        \]

        \item Средняя теплоемкость газа в интервале температур от 273 К до $T_{2ад\ т}$:
        \[
            c_{pг\ ср} (T_{2ад\ т}, \alpha_{вх}) =
            1126.07 \ Дж/(кг \cdot К)
        \]

        \item Средняя теплоемкость газа в интервале температур от $T_{0ад\ т}^*$ до $T_{2ад\ т}$:
        \[
            c_{pг\ ад\ т}^\prime = \frac{
		        c_{pг\ ср} (T_{0ад\ т}^*, \alpha_{вх}) (T_{0ад\ т}^* - T_0) - c_{pг\ ср} (T_{2ад\ т}, \alpha_{вх})(T_{2ад\ т} - T_0)
		    }{
		        T_{0ад\ т}^* - T_{2ад\ т}} =\]
        \[    =\frac{
		        1142.12 \cdot
                (1400 - 273) -
		        1126.07 \cdot
                (1270.56 - 273)
		    }{
		        1400 - 1270.56} =
		    1265.79 \ Дж / (кг \cdot К)
        \]

        \item Новое значение показателя адиабаты:
        \[
            k_{г\ ад\ т}^\prime = \frac{c_{pг\ ад\ т}^\prime}{c_{pг\ ад\ т}^\prime - R_г} =
                \frac{
                    1265.79
                }{
                    1265.79 - 287.4
                }
            = 1.2937
        \]

        \item Средняя теплоемкость газа в интервале температур от 273 К до $T_{2ад\ т}^*$:
        \[
            c_{pг\ ср} (T_{2ад\ т}^*, \alpha_{вх}) =
            1126.87 \ Дж/(кг \cdot К)
        \]

        \item Средняя теплоемкость газа в интервале температур от $T_{0ад\ т}^*$ до $T_{2ад\ т}$:
        \[
            c_{pг\ ад\ т}^{*\prime} = \frac{
		        c_{pг\ ср}(T_{0ад\ т}^*, \alpha_{вх}) (T_{0ад\ т}^* - T_0) - c_{pг\ ср}(T_{2ад\ т}^*, \alpha_{вх}) (T_{2ад\ т}^* - T_0)
		    }{
		        T_{0ад\ т}^* - T_{2ад\ т}^*} =\]
        \[    =\frac{
		        1142.12 \cdot
                (1400 - 273) -
		        1126.87 \cdot
                (1276.96 - 273)
		    }{
		        1400 - 1276.96} =
		    1266.58 \ Дж / (кг \cdot К)
        \]

        \item Новое значение показателя адиабаты:
        \[
            k_{г\ ад\ т}^{*\prime} = \frac{c_{pг\ ад\ т}^{*\prime}}{c_{pг\ ад\ т}^{*\prime} - R_г} =
                \frac{
                    1266.58
                }{
                    1266.58 - 287.4
                }
            = 1.2935
        \]

        \item Теплоперепад по параметрам торможения:
        \[
            H_0^* = c_{pг} T_0^* \left[
                        1 - \left(
                                \frac{p_2^*}{p_0^*}
                            \right) ^
                        \frac{k_г - 1}{k_г}
                    \right] =\]
        \[    1266.68 \cdot 1400 \cdot
                    \left[
                        1 - \left(
                                \frac{
                                    0.3669 \cdot 10^6
                                }{
                                    0.55 \cdot 10^6
                                }
                            \right) ^
                        \frac{1.2935 - 1}{1.2935}
                    \right]
            = 0.1557 \cdot 10^6 \ Дж/кг
        \]

        \item КПД по параметрам торможения:
        \[
            \eta_т^* = \frac{ L_т }{ H_0^* } =
                \frac{
                    0.1413 \cdot 10^6
                }{
                    0.1557 \cdot 10^6 } =
            0.9078
        \]

        \item Расход на выходе из ступени:
        \[
            G_{вых} = G_{вх} + G_т g_{охл} =
                25 + 25 \cdot
                0.004 =
            25.1 \ кг/с
        \]

        \item Относительный расход топлива на выходе из ступени:
        \[
            g_{топл.вых} = \frac{ G_{топл} }{ G_{вых} - G_{топл} } =
                 \frac{ 1 }{ 25.1 - 1 } =
            0.0415
        \]

        \item Коэффициент избытка воздуха на выходе из ступени:
        \[
            \alpha_{вых} = \frac{ 1 }{ l_0 g_{топл.вых} } =
                \frac{ 1 }{ 14.61 \cdot 0.0415 } =
            1.65
        \]

        \item Абсолютный расход охлаждающего воздуха:
        \[
            G_{охл} = G_т g_{охл} = 25 \cdot 0.004 =
            0.1
        \]

        \item Определим температуру торможения на выходе из ступени после подмешивания охлаждающего воздуха.
        \begin{enumerate}

            \item Средняя теплоемкость охлаждающего воздуха при температуре $T_{охл} = 700\ К $:
            \[
                c_{pв\ ср} (T_{охл}) = 1031.11\ Дж/ (кг \cdot К)
            \]

            \item Средняя теплоемкость газа при температуре $T_{ст}^* = 1288.44 \ К $:
            \[
                c_{pг\ ср} (T_{ст}^*, \alpha_{вх}) =
                1128.29\ Дж/ (кг \cdot К)
            \]

            \item Значение температуры смеси с предпоследней итерации $T_{см}^{*} = 1286.15\ К$.

            \item Средняя теплоемкость смеси:
            \[
                c_{pг\ ср} (T_{см}^{*}, \alpha_{вых}) =
                1127.78\ Дж/ (кг \cdot К)
            \]

            \item Новое значение температуры смеси:
            \[
                T_{см}^*\prime = \frac{
                        c_{pг\ ср} (T_{ст}^*, \alpha_{вх}) T_{ст}^* G_{вх} + c_{pв\ ср} (T_{охл}) T_{охл} G_{охл}
                    }{
                        c_{pг\ ср} (T_{см}^{*}, \alpha_{вых}) G_{вых}
                    } =
            \]
            \[    = \frac{
                    1128.29
                    \cdot 1288.44 \cdot 25 +
                    1031.11
                    \cdot 700 \cdot 0.1
                }{
                    1127.78
                    \cdot  25.1
                } =
                1286.41\ К
            \]

            \item Значение невязки:
            \[
                \delta = \frac{ \left| T_{см}^{*} - T_{см}^*\prime \right| }{T_{см}^{*}} \cdot 100 \% =
                    \frac{
                        \left| 1286.15 - 1286.41 \right|
                    }{
                        1286.15
                    } \cdot 100 \% =
                0.02 \%
            \]
        \end{enumerate}


        \item Определим температуру торможения на выходе из ступени после подмешивания охлаждающего воздуха при адиабатическом процессе в турбине..
        \begin{enumerate}

            \item Средняя теплоемкость охлаждающего воздуха при температуре $T_{охл} = 700\ К $:
            \[
                c_{pв\ ср} (T_{охл}) = 1031.11\ Дж/ (кг \cdot К)
            \]

            \item Средняя теплоемкость газа при температуре $T_{2ад\ т}^* = 1276.96 \ К $:
            \[
                c_{pг\ ср} (T_{2ад\ т}^*, \alpha_{вх}) =
                1126.87\ Дж/ (кг \cdot К)
            \]

            \item Значение температуры смеси с предпоследней итерации $T_{см\ ад\ т}^{*} = 1274.72\ К$.

            \item Средняя теплоемкость смеси:
            \[
                c_{pг\ ср} (T_{см\ ад\ т}^{*}, \alpha_{вых}) =
                1126.36\ Дж/ (кг \cdot К)
            \]

            \item Новое значение температуры смеси:
            \[
                T_{см\ ад\ т}^*\prime = \frac{
                        c_{pг\ ср} (T_{2ад\ т}^*, \alpha_{вх}) T_{2ад\ т}^* G_{вх} + c_{pв\ ср} (T_{охл}) T_{охл} G_{охл}
                    }{
                        c_{pг\ ср} (T_{см\ ад\ т}^{*}, \alpha_{вых}) G_{вых}
                    } =
            \]
            \[    = \frac{
                    1126.87
                    \cdot 1276.96 \cdot 25 +
                    1031.11
                    \cdot 700 \cdot 0.1
                }{
                    1126.36
                    \cdot  25.1
                } =
                1274.97\ К
            \]

            \item Значение невязки:
            \[
                \delta = \frac{ \left| T_{см\ ад\ т}^{*} - T_{см\ ад\ т}^*\prime \right| }{T_{см\ ад\ т}^{*}} \cdot 100 \% =
                    \frac{
                        \left| 1274.72 - 1274.97 \right|
                    }{
                        1274.72
                    } \cdot 100 \% =
                0.019 \%
            \]
        \end{enumerate}

        

    \end{enumerate}
    

    \subsection{Расчет второй ступени}

    \subsubsection{Исходные данные}

    
    \begin{enumerate}

        \item Температура торможения на входе в ступень: $T_0^* = 1286.15\ К $.
        \item Давление торможения на входе в ступень: $p_0^* = 0.3669 \cdot 10^6 \ Па$.
        \item Температура торможения на входе в ступень при адиабатическом процессе в турбине: $T_{0ад\ т}^* = 1274.72\ К$
        \item Расход газа на входе в ступень: $G_{вх} = 25.1\ кг/с$.
        \item Расход газа на входе в СА первой ступени: $ G_т = 25\ кг/с $.
        \item Расход топлива на входе в турбину: $ G_{топл} = 1\ кг/с $.
        \item Степень реактивности: $ \rho = 0.496 $.
        \item Коэффициент скорости в СА: $ \phi = 0.97 $.
        \item Коэффициент скорости в РК: $ \psi = 0.97 $.
        \item Длина лопатки на входе в РК: $ l_1 = 0.1691\ м $.
        \item Длина лопатки на выходе из РК: $ l_2 = 0.1874\ м $.
        \item Средний диаметр на входе в РК: $ D_1 = 0.5122\ м $.
        \item Средний диаметр на выходе в РК: $ D_2 = 0.5268\ м $.
        \item Радиальный зазор: $ \delta_r = 0.00187\ м $.
        \item Частота вращения ротора: $ n = 15000.0\ об/мин $
        \item Степень парциальности: $ \varepsilon = 1 $.
        \item Расход охлаждающего воздуха, отнесенный к расходу на входе в турбину: $ g_{охл} = 0.003 $.
        \item Температура торможения охлаждающего воздуха: $ T_{охл} = 700\ К $.

        
        \item Удельная работы турбины: $ L_{т\ зад} = 0.1096 \cdot 10^6 \ Дж/кг $.

        

    \end{enumerate}
    

    \subsubsection{Расчет}

    

    \begin{enumerate}

        \item Относительный расход топлива на входе в ступень:
        \[
            g_{топл.вх} = \frac{ G_{топл} }{ G_{вх} - G_{топл} } =
                \frac{ 1 }{ 25.1 - 1 } =
            0.0415
        \]

        \item Коэффициент избытка воздуха на входе:
        \[
            \alpha_{вх} = \frac{ 1 }{ l_0 g_{топл.вх} } =
                \frac{ 1 }{ 14.61 \cdot 0.0415 } =
            1.65
        \]

        \item Мощностной КПД из предпоследней итерации:
        \[
            \eta_{т0} = 0.8522
        \]

        \item Статический теплоперепад на ступени:
        \[
            H_0 = \frac{L_{т\ зад}}{\eta_{т0}} =
                \frac{ 0.1097 \cdot 10^6 }{ 0.8522 } =
            0.1287 \cdot 10^6 \ Дж/кг
        \]

        \item Средний в ступени коэффициент адиабаты из предпоследней итерации:
        \[
            k_г = 1.3013
        \]

        \item Средняя в ступени теплоемкость газа из предпоследней итерации:
        \[
            c_{pг} = 1241.36 \ Дж/(кг \cdot К)
        \]

        \item Средний в ступени коэффициент адиабаты при адиабатическом расширении в турбине до статических параметров из предпоследней итерации:
        \[
            k_{г\ ад\ т} = 1.3028
        \]

        \item Средняя в ступени теплоемкость при адиабатическом расширении в турбине до статических параметров из предпоследней итерации:
        \[
            c_{pг\ ад\ т} = 1236.58 \ Дж/(кг \cdot К)
        \]

        \item Средний в ступени коэффициент адиабаты при адиабатическом расширении в турбине до параметров торможения из предпоследней итерации:
        \[
            k_{г\ ад\ т}^* = 1.3026
        \]

        \item Средняя в ступени теплоемкость при адиабатическом расширении в турбине до параметров торможения из предпоследней итерации:
        \[
            c_{pг\ ад\ т}^* = 1237.32 \ Дж/(кг \cdot К)
        \]

        
        

        

        \item Определим теплоперепад на сопловом аппарате:

        \[
            H_с = \left( 1 - \rho \right) H_0 =
	        \left( 1 - 0.496 \right) \cdot 0.1287 \cdot 10^6 =
            0.0648 \cdot 10^6 \/\ Дж/кг
        \]

        \item Окружная скорость на диаметре $ D_1 $:

        \[
            u_1 = \frac{\pi D_1 n }{60} =
                \frac{\pi \cdot 0.5122 \cdot 15000.0}{60} =
            402.32\ м/с
        \]

        \item Определим действительную скорость истечения из СА:

	    \[
            c_1 = \phi \sqrt{2 H_с} =
	        0.97 \cdot\sqrt{2 \cdot 0.1287 \cdot 10^6}  =
            349.28 \/\ м/с
        \]

        \item Определим температуру на выходе из СА:

	    \[
            T_1 = T_0^* - \frac{ H_с \phi^2 }{ c_{pг} } =
	        1286.15 -
            \frac{
                0.0648 \cdot 10^6 \cdot {0.97}^2
            }{
                2 \cdot 1241.36
            } = 1237.01 \/\ К
        \]

	    \item Определим температуру конца адиабатного расширения:

	    \[
            T_1^\prime = T_0^* - \frac{ H_c }{ c_{pг} } =
	        1286.15 -
            \frac{
                0.0648 \cdot 10^6
            }{
                1241.36
            }
            = 1233.93  \/\ К
        \]

        \item Определим давление на выходе из СА:

	    \[
            p_1 = p_0^* \left(
                                \frac{ T_1^\prime }{ T_0^* }
                        \right)^
                    \frac{ k_г }{ k_г - 1 } =
            0.3669 \cdot 10^6 \cdot
                \left(
                        \frac{ 1233.93 }{ 1286.15 }
                \right)^
                \frac{ 1.3013 }{ 1.3013 - 1 } =
            0.3067 \cdot 10^6 \/\ МПа
        \]

        \item Определим площадь на выходе из СА:

	    \[
            A_{1a} = \pi l_1 D_1 =
	        \pi \cdot 0.1691 \cdot 0.5122 =
            0.27214 \/\ м^2
        \]

        \item Определим плотность газа на выходе из СА:

	    \[
            \rho_1 = \frac{p_1}{R_г T_1} =
	        \frac{
                0.3067 \cdot 10^6
            }{
                287.4 \cdot 1237.01
            } =
            0.863 \/\ кг/м^3
        \]

        \item Осевая составляющая абсолютной скорости на выходе из СА:

        \[
            c_{1a} = \frac{G_{вх} }{ \rho_1 A_{1a} } =
                \frac{
                    25.1
                }{
                    0.863 \cdot 0.27214
                } =
            106.9\ м/с
        \]

        \item Угол потока в абсолютном движении после СА:

        \[
            \alpha_1 = \arcsin{ \frac{ c_{1a} }{ c_1 } } =
            \arcsin{ \frac{ 106.9 }{ 349.28 } } =
            = 17.822 \degree
        \]

        \item Окружная составляющая абсолютной скорости на входе:

        \[
            c_{1u} = c_1 \cos{\alpha_1} = 349.28 \cdot \cos{17.822 \degree} =
            332.52\ м/с
        \]

        \item Определим относительную скорость на входе в РК:

	    \[
	        w_1 = \sqrt{c_1^2 + u_1^2 - 2 c_1 u_1 \cos \alpha_1} =
        \]
	    \[    = \sqrt{
            349.28 ^ 2 +
            402.32 ^ 2 -
            2 \cdot 349.28 \cdot 402.32 \cdot \cos 16.17.822 \degree
            }
            = 127.67 \/\ м/с
        \]

        \item Угол потока в относительном движении:

        
        \[
            \beta_1 = \pi + \arctan{ \frac{c_{1a}}{c_{1u} - u_1} } =
                    \pi + \arctan{ \frac{ 106.9 }{332.52 - 402.32} } =
            123.139 \degree
        \]
        

        \item Осевая составляющая относительной скорости:

        \[
            w_{1a} = w_1 \sin{\beta_1} = 127.67 \cdot  \sin{123.139 \degree} =
            106.9\ м/с
        \]

        \item Окружная составляющая относительной скорости:

        \[
            w_{1u} = w_1 \cos{\beta_1} = 127.67 \cdot  \cos{123.139 \degree} =
            -69.79\ м/с
        \]

         \item Определим теплоперепад на РК:

	    \[
            H_л = H_0 \rho \frac{T_1}{T_1^\prime} =
	        0.1287 \cdot 10^6 \cdot 0.4962 \cdot
            \frac{ 1237.01 }{ 1233.93 } =
            0.064 \cdot 10^6 \/\ Дж/кг
        \]

        \item Окружная скорость на диаметре:

        \[
            u_2 = \frac{ \pi D_2 n }{ 60 } =
                    \frac{ \pi \cdot 0.5268 \cdot 15000.0 }{ 60 } =
            413.75\ м/с
        \]

        \item Температура торможения в относительном движении после СА:

        \[
            T_{1w}^* = T_1 + \frac{ w_1^2 }{ 2 \cdot c_{pг}} =
                1237.01 + \frac{ 127.67 ^ 2 }{ 2 \cdot 1241.36}
        \]

        \item Определим относительную скорость истечения газа из РК:

	    \[
	        w_2 = \psi \sqrt{w_1^2 + 2H_л +\left( u_2^2 - u_1^2 \right)} =
        \]
        \[   = 0.97 \cdot
            \sqrt{
                127.67 ^ 2 +
                2 \cdot 0.064 \cdot 10^6 +
                \left( 413.75 ^ 2 - 402.32 ^ 2 \right)
            } =
            380.21 \/\ м/с
        \]


        \item Определим статическую температуру на выходе из РК:

	    \[
	        T_2 = T_1 + \frac{
	 	        \left( w_1^2  - w_2^2 \right) + \left( u_2^2 - u_1^2 \right)
            }{
                2 c_{pг}
            } =
        \]
	    \[
            = 1237.01 + \frac{
	 	        \left( 127.67 ^ 2  - 380.21 ^ 2 \right) +
                \left( 413.75 ^ 2 - 402.32 ^ 2 \right)
	        }{
            2 \cdot 1241.36
            }
            = 1189.11 \/\ К
        \]

        \item Определим статическую температуру при адиабатическом процессе в РК:

	    \[
            T_2^\prime = T_1 - \frac{
	 	        H_л
	        }{ c_{p г}} =
	        1237.01 - \frac{
	 	        0.064 \cdot 10^6
	        }{
                1241.36
            }
            = 1185.46 \/\ К
        \]

        \item Определим давление на выходе из РК:

	    \[
            p_2 = p_1 \left( \frac{T_2^\prime}{T_1} \right)^{\frac{k_г}{k_г - 1}} =
               0.3067 \cdot 10^6 \cdot
               \left(
               \frac{ 1185.46 }{ 1237.01 }
               \right) ^
               {\frac{
               1.301
               }{
               1.301 - 1
               }}
            = 0.2552 \cdot 10^6 \/\ Па
        \]

        \item Определим плотность газа на выходе из РК:
	    \[
            \rho_2 = \frac{p_2}{R T_2} =
                \frac{
                    0.2552 \cdot 10^6
                }{
                    287.4 \cdot 1189.11
                }
            = 0.747\ кг/м^3
        \]

        \item Определим площадь на выходе из РК:
        \[
            A_{2a} = \pi D_2 l_2 = \pi \cdot 0.5268 \cdot 0.1874 =
            0.3102\ м^2
        \]

        \item Осевая составляющая абсолютной скорости на выходе из РК:
        \[
            c_{2a} = \frac{ G_{вх} }{ A_{2a} \rho_2 } =
            \frac{ 25.1 }{ 0.3102 \cdot 0.747 }
            = 108.37\ м/с
        \]

        \item Угол потока в относительном движении на выходе из РК:
        \[
            \beta_2 = \arcsin{ \frac{ c_{2a} }{ w_2 } } =
                    \arcsin{ \frac{ 108.37 }{ 380.21 } }
            = 16.56 \degree
        \]

        \item Осевая составляющая относительной скорости потока на выходе из РК:
        \[
            w_{2a} = w_2 \cdot \sin{\beta_2} =
                    380.21 \cdot \sin{16.56 \degree}
            = 108.37\ м/с
        \]

        \item Окружная составляющая относительной скорости потока на выходе из РК:
        \[
            w_{2u} = w_2 \cdot \cos{\beta_2} =
                    380.21 \cdot \cos{16.56 \degree}
            = 364.44\ м/с
        \]

        \item Определим окружную составляющую скорости на выходе из РК:
	    \[
            c_{2u} = w_{2u} - u_2 =
	        364.44 - 413.75 = -49.31 \/\ м/с
        \]

        \item Опеределим угол потока на выходе из РК:
        
        \[
            \alpha_2 = \pi + \arctan{ \frac{ c_{2a} }{ c_{2u} } } =
                    \pi + \arctan{ \frac{ 108.37 }{ -49.31 } } =
            114.469 \degree
        \]
        

        \item Определим скорость потока на выходе из РК:
	    \[
            c_2 = \sqrt{c_{2u}^2 + c_{2a}^2} =
                \sqrt{-49.31 ^ 2 + 108.37 ^ 2} =
            119.06 \/\ м/с
        \]

        \item Определим работу на окружности колеса:
	    \[
            L_u = c_{1u} u_1 + c_{2u} u_2 =
                    332.52 \cdot 402.32 +
                    -49.31 \cdot 413.75 =
            0.1134 \cdot 10^6 \/\ Дж/кг
        \]

        \item Определим КПД на окружности колеса:
	    \[
            \eta_u = \frac{L_u}{H_0} =
                \frac{ 0.1134 \cdot 10^6 }{ 0.1287 \cdot 10^6 }
            = 0.8811
        \]

        \item Определим удельные потери в СА:
	    \[
            h_с = \left(
                        \frac{ 1 }{ \phi^2 } - 1
                \right)
                \frac{ c_1^2 }{ 2 } =
	        \left(
                \frac{ 1 }{ 0.97 ^ 2} - 1
            \right) \cdot
            \frac{ 349.28 ^ 2 }{ 2 } = 3.8315 \cdot 10^3 \/\ Дж/кг
        \]

        \item Удельные потери в СА с учетом их использования в рабочих лопатках:
        \[
            h_с^\prime = h_с \frac{ T_2^\prime }{ T_1 } =
                3.8315 \cdot 10^3 \cdot
                \frac{ 1185.46 }{ 1237.01 } =
            3.6718 \cdot 10^3 \/\ Дж/кг
        \]

        \item Относительные потери в СА:
        \[
            \zeta_с = \frac{ h_с }{ H_0 } =
                \frac{ 3.8315 \cdot 10^3 }{ 0.1287 \cdot 10^6 } =
            0.0298
        \]

        \item Относительные потери в СА с учетом их использования в рабочих лопатках:
        \[
            \zeta_с^\prime = \frac{ h_с^\prime }{ H_0 } =
                \frac{ 3.6718 \cdot 10^3 }{ 0.1287 \cdot 10^6 } =
            0.0285
        \]

        \item Удельные потери в рабочих лопатках:
        \[
            h_л = \left(
                    \frac{ 1 }{ \psi^2 } - 1
                \right)) \cdot
                \frac{ w_2^2 }{ 2 } =
            \left(
                \frac{ 1 }{ 0.97 ^ 2 } - 1
            \right) \cdot
            \frac{ 380.21 ^ 2} {2}
            = 4.54 \cdot 10^3 \/\ Дж/кг
        \]

        \item Относительные потери в рабочих лопатках:
        \[
            \zeta_л = \frac{ h_л }{ H_0 } =
                \frac{ 4.54 \cdot 10^3 }{ 0.1287 \cdot 10^6 } =
            0.0353
        \]

        \item Определим удельные потери с выходной скоростью:
        \[
            h_{вых} = \frac{ c_2 ^ 2 }{ 2} =
                    \frac{ 119.06 ^ 2 }{ 2 } =  7.0875 \cdot 10^3 \/\ Дж/кг
        \]

        \item Относительные потери с выходной скоростью:
        \[
            \zeta_{вых} = \frac{ h_{вых} }{ H_0 } =
                \frac{ 7.0875 \cdot 10^3 }{ 0.1287 \cdot 10^6 } =
            0.0551
        \]

        \item Проверка КПД на окружности колеса:
        \[
            \eta_u = 1 - \zeta_с^\prime - \zeta_л - \zeta_{вых} = 1 - 0.0285 -
                    0.0353 - 0.0551 = 0.8811
        \]

        \item Средний диаметр:
        \[
            D_{ср} = 0.5 \cdot (D_1 + D_2) =
                    0.5 \cdot (0.5122 + 0.5268) =
            0.5195\ м
        \]

        \item Определим удельные потери в радиальном зазоре:

	    \[
	        h_з = 1.37 \cdot
                \left(
                    1 + 1.6 \rho
                \right)
                \left(
                    1 + \frac{l_2}{D_{ср}}
                \right)
            \frac{ \delta_r }{ l_2 } \cdot L_u =\]
        \[   = 1.37 \cdot
            \left(
                1 + 1.6 \cdot 0.5
            \right)
            \left(
                1 + \frac{ 0.1874 }{ 0.5195 }
            \right)
            \frac{ 0.00187 }{ 0.1874 } \cdot
            0.1134 \cdot 10^6 =
	        3.7914 \cdot 10^3 \/\ Дж/кг\]


        \item Относительные удельные потери в радиальном зазоре:
        \[
            \zeta_з = \frac{ h_з }{ H_0 } =
                \frac{ 3.7914 \cdot 10^3 }{ 0.1287 \cdot 10^6 } =
            0.0295
        \]

        \item Удельная работа ступени с учетом потери в радиальном зазоре:
        \[
            L_{uз} = L_u - h_з = 0.1134 \cdot 10^6 -
                3.7914 \cdot 10^3 =
            0.1096 \cdot 10^6 \ Дж/кг
        \]

        \item Мощностной КПД ступени:
        \[
            \eta_т^\prime = \eta_u - \zeta_з =
                0.8811 - 0.0295 = 0.8516
        \]

        \item Лопаточный КПД ступени:
        \[
            \eta_л^\prime = \eta_т^\prime + \zeta_{вых} =
                 0.8516 +  0.0551 =
            0.9067
        \]

        \item Средняя длина лопатки:
        \[
            l_{ср} = 0.5 \cdot (l_1 + l_2) =
                0.5 \cdot (0.1691 + 0.1874) =
            0.1783\ м
        \]

        \item Средняя окружная скоротсь:
        \[
            u_{ср} = 0.5 \cdot (u_1 + u_2) =
                0.5 \cdot (402.32 + 413.75) =
            408.03\ м/с
        \]

        \item Затраты мощности на трение и вентиляцию:
        \[
            N_{т.в} = \left[
                    1.07 \cdot D_{av}^2 + 61 \cdot (1 - \varepsilon) \cdot D_{av} l_{av}
            \right] \cdot
            \left(
                \frac{ u_{av} }{ 100 }
            \right) ^ 3 \cdot
            \rho =
        \]
        \[    = \left[
                1.07 \cdot 0.5195^2 +
                61 \cdot (1 - 1) \cdot
                0.5195 \cdot 0.1783
            \right] \cdot
            \left(
                \frac{ 408.03 }{ 100 }
            \right) ^ 3 \cdot
            0.4962=
        \]
        \[
            = 0.0097 \cdot 10^3 \ Вт
        \]

        \item Удельные потери на трение и вентиляцию:
        \[
            h_{т.в} = \frac{ N_{т.в} }{ G_{вх} } =
                \frac{
                    0.0097 \cdot 10^3
                }{
                    25.1
                }
            = 0.0004 \cdot 10^3 \ Дж/кг
        \]

        \item Относительные потери на трение и вентиляцию:
        \[
            \zeta_{т.в} = \frac{ h_{т.в} }{ H_0 } =
                \frac{ 0.0004 \cdot 10^3 }{ 0.1287 \cdot 10^6 } =
            0.0
        \]

        \item Мощностной КПД с учетом потерь на трению и вентиляцию:
        \[
            \eta_т = \eta_т^\prime - \zeta_{т.в} =
                0.8516 - 0.0 =
            0.8516
        \]

        \item Лопаточный КПД с учетом потерь на трению и вентиляцию:
        \[
            \eta_л = \eta_л^\prime - \zeta_{т.в} =
                0.9067 - 0.0 =
            0.9067
        \]

        \item Определим удельную работу ступени:
        \[
            L_т = H_0 \eta_т = 0.1287 \cdot 10^6 \cdot 0.8516 =
            = 0.1096 \cdot 10^6 \ Дж/кг
        \]

        \item Удельная работа ступени, отнесенная к расходу на в СА первой ступени:
        \[
            L_т^\prime = L_т \frac{ G_{вх} }{ G_т }  =
                0.1096 \cdot 10^6 \cdot
                \frac{ 25.1 }{ 25 } =
            0.11 \cdot 10^6 \ Дж/кг
        \]

        \item Статическая температура за ступенью:
        \[
            T_{ст} = T_2 + \frac{ h_з }{ c_{pг} } + \frac{ h_{т.в} }{ c_{pг} } =
                1189.11 +
                \frac{3.7914 \cdot 10^3 }{ 1241.36 } +
                \frac{ 0.0004 \cdot 10^3 }{ 1241.36 } =
            1192.17 \ К
        \]

        \item Температура торможения за турбиной:
        \[
            T_{ст}^* = T_{ст} + \frac{ h_{вых} }{ c_{pг} } =
                1192.17 +
                \frac{ 7.0875 \cdot 10^3 }{ 1241.36 } =
            1197.88 \ К
        \]

        \item Средняя теплоемкость газа в интервале температур от 273 К до $T_0^*$:
        \[
            c_{pг\ ср} (T_0^*, \alpha_{вх}) =
            1127.78 \ Дж/(кг \cdot К)
        \]

        \item Средняя теплоемкость газа в интервале температур от 273 К до $T_{ст}^*$:
        \[
            c_{pг\ ср} (T_{ст}^*, \alpha_{вх}) =
            1116.84 \ Дж/(кг \cdot К)
        \]

        \item Средняя теплоемкость газа в интервале температур от $T_0^*$ до $T_{ст}^*$:
        \[
            c_{pг}^\prime = \frac{
		        c_{pг\ ср} (T_0^*, \alpha_{вх}) (T_0^* - T_0) - c_{pг\ ср} (T_{ст}^*, \alpha_{вх})(T_{ст}^* - T_0)
		    }{
		        T_0^* - T_{ст}^*} =\]
        \[    =\frac{
		        1127.78 \cdot
                (1286.15 - 273) -
		        1116.84 \cdot
                (1197.88 - 273)
		    }{
		        1286.15 - 1197.88} =
		    1242.31 \ Дж / (кг \cdot К)
        \]

        \item Новое значение показателя адиабаты:
        \[
            k_г^\prime = \frac{c_{pг}^\prime}{c_{pг}^\prime - R_г} =
                \frac{
                    1242.31
                }{
                    1242.31 - 287.4
                }
            = 1.301
        \]

        \item Невязка по коэффициенту адиабаты:
        \[
            \delta = \frac{ \left| k_г - k_г^\prime \right| }{ k_г } \cdot 100 \%=
                \frac{
                    \left| 1.3013 - 1.301 \right|
                }{
                    1.3013
                } \cdot 100 \% =
            0.0231 \%
        \]

        \item Давление торможения на выходе из ступени:
        \[
            p_2^* = p_2 \left(
                            \frac{ T_{ст}^* }{ T_{ст} }
                    \right) ^ \frac{ k_г }{ k_г - 1 } =
                 0.2552 \cdot 10^6 \cdot \left(
                            \frac{ 1197.88 }{ 1192.17 }
                    \right) ^
                \frac{ 1.3013 }{ 1.3013 - 1 } =
            0.2605 \cdot 10^6 \ Па
        \]

        \item Статическая температура на выходе из ступени при адиабатическом процессе в турбине:
        \[
            T_{2ад\ т} = T_{0ад\ т}^* \cdot \frac{p_2}{p_0^*} ^ {
                    \frac{k_{г\ ад\ т} - 1}{k_{г\ ад\ т} }
            } = 1274.72 \cdot
            \frac{ 0.2552
            }{
            0.3669
            } ^ {
                    \frac{1.303 - 1}{1.303}
            } =
            1171.61\ К
        \]

        \item Полная температура на выходе из ступени при адиабатическом процессе в турбине:
        \[
            T_{2ад\ т}^* = T_{0ад\ т}^* \cdot \frac{p_2^*}{p_0^*} ^ {
                    \frac{k_{г\ ад\ т}^* - 1}{k_{г\ ад\ т}^*}
            } = 1274.72 \cdot
            \frac{ 0.2605
            }{
            0.3669
            } ^ {
                    \frac{1.303 - 1}{1.303}
            } =
            1177.3\ К
        \]

        \item Статический теплоперепад при адиабатическом процессе в турбине:
        \[
            H_{0ад\ т} = c_{pг\ ад\ т} \cdot \left(
            T_{0ад\ т}^* - T_{2ад\ т}
            \right) =
            1236.58 \cdot \left(
            1274.72 - 1171.61
            \right) =
            0.1275 \cdot 10^6 \ Дж/кг
        \]

        \item Статический теплоперепад при адиабатическом процессе в турбине, отнесенный к расходу на входе:
        \[
            H_{0ад\ т}^\prime = H_{0ад\ т} \cdot \frac{ G_{вх} }{ G_т }  =
                0.1275 \cdot 10^6 \cdot
                \frac{ 25.1 }{ 25 } =
            0.128 \cdot 10^6 \ Дж/кг
        \]

        \item Теплоперепад по параметрам торможения при адиабатическом процессе в турбине:
        \[
            H_{0ад\ т}^* = c_{pг\ ад\ т}^* \cdot \left(
            T_{0ад\ т}^* - T_{2ад\ т}^*
            \right) =
            1237.32 \cdot \left(
            1274.72 - 1177.3
            \right) =
            0.1205 \cdot 10^6 \ Дж/кг
        \]

        \item Теплоперепад по параметрам торможения при адиабатическом процессе в турбине, отнесенный к расходу на входе:
        \[
            H_{0ад\ т}^{*\prime} = H_{0ад\ т}^* \cdot \frac{ G_{вх} }{ G_т }  =
                0.1205 \cdot 10^6 \cdot
                \frac{ 25.1 }{ 25 } =
            0.121 \cdot 10^6 \ Дж/кг
        \]

        \item Средняя теплоемкость газа в интервале температур от 273 К до $T_{0ад\ т}^*$:
        \[
            c_{pг\ ср} (T_{0ад\ т}^*, \alpha_{вх}) =
            1126.36 \ Дж/(кг \cdot К)
        \]

        \item Средняя теплоемкость газа в интервале температур от 273 К до $T_{2ад\ т}$:
        \[
            c_{pг\ ср} (T_{2ад\ т}, \alpha_{вх}) =
            1113.59 \ Дж/(кг \cdot К)
        \]

        \item Средняя теплоемкость газа в интервале температур от $T_{0ад\ т}^*$ до $T_{2ад\ т}$:
        \[
            c_{pг\ ад\ т}^\prime = \frac{
		        c_{pг\ ср} (T_{0ад\ т}^*, \alpha_{вх}) (T_{0ад\ т}^* - T_0) - c_{pг\ ср} (T_{2ад\ т}, \alpha_{вх})(T_{2ад\ т} - T_0)
		    }{
		        T_{0ад\ т}^* - T_{2ад\ т}} =\]
        \[    =\frac{
		        1126.36 \cdot
                (1274.72 - 273) -
		        1113.59 \cdot
                (1171.61 - 273)
		    }{
		        1274.72 - 1171.61} =
		    1239.06 \ Дж / (кг \cdot К)
        \]

        \item Новое значение показателя адиабаты:
        \[
            k_{г\ ад\ т}^\prime = \frac{c_{pг\ ад\ т}^\prime}{c_{pг\ ад\ т}^\prime - R_г} =
                \frac{
                    1237.64
                }{
                    1237.64 - 287.4
                }
            = 1.3024
        \]

        \item Средняя теплоемкость газа в интервале температур от 273 К до $T_{2ад\ т}^*$:
        \[
            c_{pг\ ср} (T_{2ад\ т}^*, \alpha_{вх}) =
            1114.29 \ Дж/(кг \cdot К)
        \]

        \item Средняя теплоемкость газа в интервале температур от $T_{0ад\ т}^*$ до $T_{2ад\ т}$:
        \[
            c_{pг\ ад\ т}^{*\prime} = \frac{
		        c_{pг\ ср}(T_{0ад\ т}^*, \alpha_{вх}) (T_{0ад\ т}^* - T_0) - c_{pг\ ср}(T_{2ад\ т}^*, \alpha_{вх}) (T_{2ад\ т}^* - T_0)
		    }{
		        T_{0ад\ т}^* - T_{2ад\ т}^*} =\]
        \[    =\frac{
		        1126.36 \cdot
                (1274.72 - 273) -
		        1114.29 \cdot
                (1177.3 - 273)
		    }{
		        1274.72 - 1177.3} =
		    1238.35 \ Дж / (кг \cdot К)
        \]

        \item Новое значение показателя адиабаты:
        \[
            k_{г\ ад\ т}^{*\prime} = \frac{c_{pг\ ад\ т}^{*\prime}}{c_{pг\ ад\ т}^{*\prime} - R_г} =
                \frac{
                    1238.35
                }{
                    1238.35 - 287.4
                }
            = 1.3022
        \]

        \item Теплоперепад по параметрам торможения:
        \[
            H_0^* = c_{pг} T_0^* \left[
                        1 - \left(
                                \frac{p_2^*}{p_0^*}
                            \right) ^
                        \frac{k_г - 1}{k_г}
                    \right] =\]
        \[    1241.36 \cdot 1286.15 \cdot
                    \left[
                        1 - \left(
                                \frac{
                                    0.2605 \cdot 10^6
                                }{
                                    0.3669 \cdot 10^6
                                }
                            \right) ^
                        \frac{1.3013 - 1}{1.3013}
                    \right]
            = 0.1216 \cdot 10^6 \ Дж/кг
        \]

        \item КПД по параметрам торможения:
        \[
            \eta_т^* = \frac{ L_т }{ H_0^* } =
                \frac{
                    0.1096 \cdot 10^6
                }{
                    0.1216 \cdot 10^6 } =
            0.9008
        \]

        \item Расход на выходе из ступени:
        \[
            G_{вых} = G_{вх} + G_т g_{охл} =
                25.1 + 25 \cdot
                0.003 =
            25.18 \ кг/с
        \]

        \item Относительный расход топлива на выходе из ступени:
        \[
            g_{топл.вых} = \frac{ G_{топл} }{ G_{вых} - G_{топл} } =
                 \frac{ 1 }{ 25.18 - 1 } =
            0.0414
        \]

        \item Коэффициент избытка воздуха на выходе из ступени:
        \[
            \alpha_{вых} = \frac{ 1 }{ l_0 g_{топл.вых} } =
                \frac{ 1 }{ 14.61 \cdot 0.0414 } =
            1.655
        \]

        \item Абсолютный расход охлаждающего воздуха:
        \[
            G_{охл} = G_т g_{охл} = 25 \cdot 0.003 =
            0.075
        \]

        \item Определим температуру торможения на выходе из ступени после подмешивания охлаждающего воздуха.
        \begin{enumerate}

            \item Средняя теплоемкость охлаждающего воздуха при температуре $T_{охл} = 700\ К $:
            \[
                c_{pв\ ср} (T_{охл}) = 1031.11\ Дж/ (кг \cdot К)
            \]

            \item Средняя теплоемкость газа при температуре $T_{ст}^* = 1197.88 \ К $:
            \[
                c_{pг\ ср} (T_{ст}^*, \alpha_{вх}) =
                1116.84\ Дж/ (кг \cdot К)
            \]

            \item Значение температуры смеси с предпоследней итерации $T_{см}^{*} = 1196.43\ К$.

            \item Средняя теплоемкость смеси:
            \[
                c_{pг\ ср} (T_{см}^{*}, \alpha_{вых}) =
                1116.5\ Дж/ (кг \cdot К)
            \]

            \item Новое значение температуры смеси:
            \[
                T_{см}^*\prime = \frac{
                        c_{pг\ ср} (T_{ст}^*, \alpha_{вх}) T_{ст}^* G_{вх} + c_{pв\ ср} (T_{охл}) T_{охл} G_{охл}
                    }{
                        c_{pг\ ср} (T_{см}^{*}, \alpha_{вых}) G_{вых}
                    } =
            \]
            \[    = \frac{
                    1116.84
                    \cdot 1197.88 \cdot 25.1 +
                    1031.11
                    \cdot 700 \cdot 0.075
                }{
                    1116.5
                    \cdot  25.18
                } =
                1196.58\ К
            \]

            \item Значение невязки:
            \[
                \delta = \frac{ \left| T_{см}^{*} - T_{см}^*\prime \right| }{T_{см}^{*}} \cdot 100 \% =
                    \frac{
                        \left| 1196.43 - 1196.58 \right|
                    }{
                        1196.43
                    } \cdot 100 \% =
                0.012 \%
            \]
        \end{enumerate}


        \item Определим температуру торможения на выходе из ступени после подмешивания охлаждающего воздуха при адиабатическом процессе в турбине..
        \begin{enumerate}

            \item Средняя теплоемкость охлаждающего воздуха при температуре $T_{охл} = 700\ К $:
            \[
                c_{pв\ ср} (T_{охл}) = 1031.11\ Дж/ (кг \cdot К)
            \]

            \item Средняя теплоемкость газа при температуре $T_{2ад\ т}^* = 1177.3 \ К $:
            \[
                c_{pг\ ср} (T_{2ад\ т}^*, \alpha_{вх}) =
                1114.29\ Дж/ (кг \cdot К)
            \]

            \item Значение температуры смеси с предпоследней итерации $T_{см\ ад\ т}^{*} = 1175.91\ К$.

            \item Средняя теплоемкость смеси:
            \[
                c_{pг\ ср} (T_{см\ ад\ т}^{*}, \alpha_{вых}) =
                1113.96\ Дж/ (кг \cdot К)
            \]

            \item Новое значение температуры смеси:
            \[
                T_{см\ ад\ т}^*\prime = \frac{
                        c_{pг\ ср} (T_{2ад\ т}^*, \alpha_{вх}) T_{2ад\ т}^* G_{вх} + c_{pв\ ср} (T_{охл}) T_{охл} G_{охл}
                    }{
                        c_{pг\ ср} (T_{см\ ад\ т}^{*}, \alpha_{вых}) G_{вых}
                    } =
            \]
            \[    = \frac{
                    1114.29
                    \cdot 1177.3 \cdot 25.1 +
                    1031.11
                    \cdot 700 \cdot 0.075
                }{
                    1113.96
                    \cdot  25.18
                } =
                1176.05\ К
            \]

            \item Значение невязки:
            \[
                \delta = \frac{ \left| T_{см\ ад\ т}^{*} - T_{см\ ад\ т}^*\prime \right| }{T_{см\ ад\ т}^{*}} \cdot 100 \% =
                    \frac{
                        \left| 1175.91 - 1176.05 \right|
                    }{
                        1175.91
                    } \cdot 100 \% =
                0.012 \%
            \]
        \end{enumerate}

        

        \item Невязка по работе ступени:
        \[
            \delta_L = \frac{ \left| L_{т\ зад} - L_т \right| }{ L_{т\ зад} } \cdot 100 \% =
                \frac{
                    \left| 0.1097 - 0.1096 \right|
                }{
                    0.1097 } \cdot 100 \% =
            0.0664 \%
        \]

    \end{enumerate}
     

    \subsection{Расчет интегральных параметров турбины}

    

    
    \begin{enumerate}

        \item Суммарная работа всех ступеней:
        \[
            L_{т\Sigma} = \sum_{i=1}^{i=n}{L_{тi}^{\prime}} =
            0.1413\cdot 10^6+0.11\cdot 10^6 = 0.2513 \cdot 10^6 \ Дж/кг
        \]

        \item Средняя теплоемкость газа в интервале температур от 273 К до $T_г^*$:
        \[
            c_{pг\ ср} (T_г^*, \alpha_{вх}) =
            1142.12 \ Дж/(кг \cdot К)
        \]

        \item Средняя теплоемкость газа в интервале температур от 273 К до $T_т$:
        \[
            c_{pг\ ср} (T_т, \alpha_{вх}) =
            1116.35 \ Дж/(кг \cdot К)
        \]

        \item Средняя теплоемкость газа в интервале температур от $T_0^*$ до $T_т$:
        \[
            c_{pг} = \frac{
		         c_{pг\ ср} (T_г^*, \alpha_{вх}) (T_г^* - T_0) - c_{pг\ ср} (T_{т}, \alpha_{вх})(T_т - T_0)
		    }{
		        T_г^* - T_т} =
        \]
        \[    =\frac{
                1142.12 \cdot
                (1400 - 273) -
		        1116.35 \cdot
                (1192.17 - 273)
		    }{
		        1400 - 1192.17} =
		    1255.71 \ Дж / (кг \cdot К)
        \]

        \item Средний показателя адиабаты:
        \[
            k_г = \frac{c_{pг}}{c_{pг} - R_г} =
                \frac{
                    1255.71
                }{
                    1255.71 - 287.4
                }
            = 1.2968
        \]

        \item Статический теплоперепад на турбине:
        \[
            H_т = \sum_{i=1}^{i=n-1}H_{0ад\ т\ i}^{*\prime} + H_{0\ ад\ т\ n}^\prime =
            0.2837 \cdot 10^6 \ Дж/кг
        \]

        \item Средняя теплоемкость газа в интервале температур от 273 К до $T_т^*$:
        \[
            c_{pг\ ср} (T_т^*, \alpha_{вх}) =
            1116.88 \ Дж/(кг \cdot К)
        \]

        \item Средняя теплоемкость газа в интервале температур от $T_0^*$ до $T_т^*$:
        \[
            c_{pг}^* = \frac{
		         c_{pг\ ср} (T_г^*, \alpha_{вх}) (T_г^* - T_0) - c_{pг\ ср} (T_т^*, \alpha_{вх})(T_т^* - T_0)
		    }{
		        T_г^* - T_т^*} =
        \]
        \[    =\frac{
                1142.12 \cdot
                (1400 - 273) -
		        1116.88 \cdot
                (1196.43 - 273)
		    }{
		        1400 - 1196.43} =
        \]
		\[     = 1256.41 \ Дж / (кг \cdot К)
        \]

        \item Средний показателя адиабаты по параметрам торможения:
        \[
            k_г^* = \frac{ c_{pг}^* }{ c_{pг}^* - R_г } =
                \frac{
                    1256.41
                }{
                    1256.41 - 287.4
                }
            = 1.2966
        \]

        \item Теплоперепад по параметрам торможения на турбине:
        \[
            H_т^* = \sum_{i=1}^{i=n}H_{0ад\ т\ i}^{*\prime} =
            0.2767 \cdot 10^6 \ Дж/кг
        \]

        \item Мощностной КПД турбины:
        \[
            \eta_т = \frac{ L_{т\Sigma} }{ H_т } =
                \frac{ 0.2513 \cdot 10^6 }{ 0.2837 \cdot 10^6 } =
            0.886
        \]

        \item Лопаточный КПД турбины:
        \[
            \eta_л = \frac{
                        L_{т\Sigma} + 0.5 \cdot c_{вых}^2
                    }{ H_т } =
            \frac{
                0.2513 \cdot 10^6 + 0.5 \cdot 119.06 ^ 2
            }{ 0.2837 \cdot 10^6 } =
            0.911
        \]

        \item КПД турбины по параметрам торможения:
        \[
            \eta_т^* = \frac{ L_{т\Sigma} }{ H_т^* } =
                \frac{ 0.2513 \cdot 10^6 }{ 0.2767 \cdot 10^6 } =
            0.9084
        \]

        \item Степень понижения давления по статическим параметрам:
        \[
            \pi_{т} = \frac{p_{01}^*}{p_{2 2}} =
            \frac{0.55}{0.2552} =
            2.155
        \]

        \item Степень понижения давления по параметрам торможения:
        \[
            \pi_{т}^* = \frac{p_{01}^*}{p_{2 2}^*} =
            \frac{0.55}{0.2605} =
            2.111
        \]

    \end{enumerate}
    

    \subsection{Параметры ступеней турбины}
    

    
    \begin{longtable}{
    |p{8cm}|
%    
    c|
%    
    c|
%    
    }
        \caption{Параметры ступеней турбины.} \\ \hline
        Номер ступени
%        
        & 1
%        
        & 2
%        
        \\ \hline
        Средний диаметр на входе в РК $D_1$, м
%        
        & 0.482
%        
        & 0.512
%        
        \\ \hline
        Средний диаметр на выходе в РК $D_1$, м
%        
        & 0.492
%        
        & 0.527
%        
        \\ \hline
        Длина лопатки на входе  $l_1$, м
%        
        & 0.12
%        
        & 0.169
%        
        \\ \hline
        Длина лопатки на выходе  $l_2$, м
%        
        & 0.133
%        
        & 0.187
%        
        \\ \hline
        Степень реактивности
%        
        & 0.4
%        
        & 0.496
%        
        \\ \hline
        Давление на входе в ступень $p_0^*$, МПа
%        
        & 0.55
%        
        & 0.3669
%        
        \\ \hline
        Температура на входе в ступень $T_0^*$, К
%        
        & 1400
%        
        & 1286.15
%        
        \\ \hline
        Расход на входе в ступень $G_{вх}$, кг/с
%        
        & 25
%        
        & 25.1
%        
        \\ \hline
        Статический теплоперепад $H_0$, МДж/кг
%        
        & 0.1637
%        
        & 0.1287
%        
        \\ \hline
        Относительный расход охлаждающего воздуха $g_{охл}$
%        
        & 0.004
%        
        & 0.003
%        
        \\ \hline
        Теплоперепад на СА $H_с$, МДж/кг
%        
        & 0.0982
%        
        & 0.0648
%        
        \\ \hline
        Окружная скорость $u_1$, м/с
%        
        & 378.2
%        
        & 402.3
%        
        \\ \hline
        Скорость истечения из СА $c_1$, м/с
%        
        & 429.9
%        
        & 349.3
%        
        \\ \hline
        Статическая температура на выходе из СА $T_1$, К
%        
        & 1327.1
%        
        & 1237.0
%        
        \\ \hline
        Статическое давление на выходе из СА $p_1$, МПа
%        
        & 0.4279
%        
        & 0.3067
%        
        \\ \hline
        Угол потока на выходе из СА $\alpha_1$, град
%        
        & 16.5
%        
        & 17.8
%        
        \\ \hline
        Относительная скорость на выходе из СА $w_1$, м/с
%        
        & 127.0
%        
        & 127.7
%        
        \\ \hline
        Угол потока в относительном движении на выходе из СА $\beta_1$, град
%        
        & 74.5
%        
        & 123.1
%        
        \\ \hline
        Теплоперепад на РК $H_л$, МДж/кг
%        
        & 0.0657
%        
        & 0.064
%        
        \\ \hline
        Окружная скорость на выходе из РК $u_2$, м/с
%        
        & 386.4
%        
        & 413.8
%        
        \\ \hline
        Полная температура в относительном движении на выходе из СА $T_{1w}^*$, К
%        
        & 1333.4
%        
        & 1243.6
%        
        \\ \hline
        Относительная скорость на выходе из РК $w_2$, м/с
%        
        & 380.3
%        
        & 380.2
%        
        \\ \hline
        Статическая температура на выходе из HR $T_2$, К
%        
        & 1278.8
%        
        & 1189.1
%        
        \\ \hline
        Статическое давление на выходе из РК $p_2$, МПа
%        
        & 0.3589
%        
        & 0.2552
%        
        \\ \hline
        Абсолютная скорость истечения из РК $c_2$, м/с
%        
        & 127.0
%        
        & 119.1
%        
        \\ \hline
        Угол потока на выходе из РК $\alpha_2$, град
%        
        & 102.2
%        
        & 114.5
%        
        \\ \hline
        Угол потока в относительном движении на выходе из РК $\beta_2$, град
%        
        & 19.1
%        
        & 16.6
%        
        \\ \hline
        Работа на окружности колеса $L_u$, МДж/кг
%        
        & 0.1455
%        
        & 0.1134
%        
        \\ \hline
        КПД на окружности колеса $\eta_u$
%        
        & 0.8889
%        
        & 0.8811
%        
        \\ \hline
        Мощностной КПД $\eta_т$
%        
        & 0.8634
%        
        & 0.8516
%        
        \\ \hline
        Лопаточный КПД $\eta_л$
%        
        & 0.9127
%        
        & 0.9067
%        
        \\ \hline
        Удельная работа ступени $L_т$, МДж/кг
%        
        & 0.1413
%        
        & 0.1096
%        
        \\ \hline
        Статическая температура за ступенью $T_{ст}$, К
%        
        & 1282.1
%        
        & 1192.2
%        
        \\ \hline
        Температура торможения за ступенью $T_{ст}^*$, К
%        
        & 1288.4
%        
        & 1197.9
%        
        \\ \hline
        Давление торможения за ступенью $p_2^*$, МПа
%        
        & 0.3669
%        
        & 0.2605
%        
        \\ \hline
        Теплоперепад по параметрам торможения $H_0^*$, МДж/кг
%        
        & 0.1413
%        
        & 0.1096
%        
        \\ \hline
        КПД по параметрам торможения $\eta_т^*$
%        
        & 0.9078
%        
        & 0.9008
%        
        \\ \hline
        Температура после подмешивания охлаждающего воздуха $T_{см^*}$
%        
        & 1286.2
%        
        & 1196.4
%        
        \\ \hline
    \end{longtable}
    

    \section{Расчет параметров потока по высоте.}

    

    При расчете параметров потока по высоте лопаточного венца будем пользоваться описанной ниже методикой.

    \begin{enumerate}

        \item Допущения:

        \begin{enumerate}

            \item Постоянство температуры торможения на входе в СА: $T_0^*(r) = const$.
            \item Постояноство скорости на входе: $c_0(r) = const$.
            \item Постоянство угла потока на входе: $\alpha_0(r) = const$.
            \item Постоянство полного давления на входе: $p_0^*(r) = const$.
            \item Постоянство работы на окружности колеса: $L_u(r) = const$.
            \item Потерь в лопаточных венцах нет.
            \item Ступень цилиндрическая.

        \end{enumerate}

%        

        \item Статическая температура на входе в СА:
        \[
            T_0(r) = T_0^* - \frac{c_0 ^ 2}{2 c_p}
        \]

        \item Окружная скорость на выходе из СА:
%        

        \item Осевая скорость на выходе из СА:
%        

        \item Скорость в абсолютном движении на выходе из СА:
        \[
            c_1(r) = \sqrt{c_{1a}(r)^2 + c_{1u}(r)^2}
        \]

        \item Угол потока в абсолютном движении на выходе из СА:
        \[
            \alpha_1 (r) = \arcsin{\frac{c_{1a}(r)}{c_1(r)}}
        \]

        \item Теплоперепад на СА:
        \[
            H_с (r) = \frac{c_1(r) ^ 2}{2}
        \]

        \item Статическое давление на выходе из СА:
        \[
            p_1 = p_0^*(r) \left( 1 - \frac{H_с (r)}{T_0^* c_p} \right) ^ {\frac{k}{k - 1}}
        \]

        \item Статическая температура на выходе из СА:
        \[
            T_1 (r) = T_0^*(r) - \frac{H_с (r)}{c_p}
        \]

        \item Окружная скорость:
        \[
            u(r) = \frac{2 \pi n r}{60}
        \]

        \item Температура торможения на выходе из РК:
        \[
            T_2^* = T_0^*(r) - \frac{L_u (r)}{c_p}
        \]

        \item Окружная скорость на выходе из РК:
        \[
            c_{2u}(r) = \frac{L_u(r) - c_{1u} u(r)}{u(r)}
        \]

        \item Осевая скорость на выходе из РК:
        \[
            c_{2a}(r) = \sqrt{
                    c_{2a\ ср}^2 + с_{2u\ ср}^2 - c_{2u}(r)^2 -
                    2 \cdot \int_{r_{ср}}^{r} \frac{c_{2u}(r)^2}{r} dr
            }
        \]

        \item Скорость в абсолютном движении на выходе из РК:
        \[
            c_2(r) = \sqrt{c_{2a}(r)^2 + c_{2u}(r)^2}
        \]

        \item Угол потока в абсолютном движении на выходе и РК:
        \[
            \alpha_2 (r) = \arctan{\frac{c_{2a}(r)}{c_{2u}(r)}}
        \]

        \item Окружная составляющая относительной скорости на выходе из РК:
        \[
            w_{2u}(r) = c_{2u}(r) + u(r)
        \]

        \item Относитеьная скорость на выходе из РК:
        \[
            w_2 (r) = \sqrt{w_{2u}(r)^2 + c_{2a}(r)^2}
        \]

        \item Относительная скорость на выходе из СА:
        \[
            w_1 (r) = \sqrt{c_1(r)^2 + u(r)^2 - 2 \cdot u(r) c_1(r) \cos{\alpha_1(r)}}
        \]

        \item Температура торможения в относительном движении на выходе из РК:
        \[
            T_{1w}^* = T_1(r) + \frac{w_1(r) ^ 2}{2 c_p}
        \]

        \item Теплоперепад в РК:
        \[
            H_л (r) = 0.5 \cdot (w_2(r)^2 - w_1(r)^2)
        \]

        \item Статическое давление на выходе из РК:
        \[
            p_2 (r) = p_1(r) \cdot \left( 1 - \frac{H_л}{c_p T_1(r)} \right) ^ {\frac{k}{k - 1}}
        \]

        \item Статическая температура на выходе из РК:
        \[
            T_2 (r) = T_1 (r) - \frac{w_2 (r)^2 - w_1(r)^2}{2 c_p}
        \]

        \item Статический теплоперепад на ступени:
        \[
            H_0 (r) = c_p \cdot T_0^*(r) \cdot \left( 1 - \frac{p_0^*(r)}{p_2(r)} \right) ^ {\frac{1 - k}{k}}
        \]

        \item Степень реактивности:
        \[
            \rho (r) = \frac{H_л (r)}{H_0 (r)}
        \]

    \end{enumerate}
%        

    
    \begin{longtable}{
    |
%    
    c|
%    
    c|
%    
    c|
%    
    c|
%    
    }
        \caption{Параметры первой ступени на различных радиусах.}\\
        \hline

%        
        $\frac{r - r_{вт}}{r_{п} - r_{вт}}$
%        
        & 0.0
%        
        & 0.5
%        
        & 1.0
%        
        \\
        \hline
%        
        $r,\ мм$
%        
        & 180.6
%        
        & 240.8
%        
        & 301.0
%        
        \\
        \hline
%        
        $\rho$
%        
        & -0.02
%        
        & 0.398
%        
        & 0.6
%        
        \\
        \hline
%        
        $c_1,\ м/с$
%        
        & 560.0
%        
        & 429.9
%        
        & 350.2
%        
        \\
        \hline
%        
        $c_{1a},\ м/с$
%        
        & 159.4
%        
        & 122.3
%        
        & 99.7
%        
        \\
        \hline
%        
        $c_{1u},\ м/с$
%        
        & 536.8
%        
        & 412.1
%        
        & 335.7
%        
        \\
        \hline
%        
        $\alpha_1,\ ^\circ$
%        
        & 16.5
%        
        & 16.5
%        
        & 16.5
%        
        \\
        \hline
%        
        $w_1,\ м/с$
%        
        & 299.1
%        
        & 127.0
%        
        & 169.5
%        
        \\
        \hline
%        
        $w_{1a},\ м/с$
%        
        & 159.4
%        
        & 122.3
%        
        & 99.7
%        
        \\
        \hline
%        
        $w_{1u},\ м/с$
%        
        & 253.1
%        
        & 33.9
%        
        & -137.1
%        
        \\
        \hline
%        
        $\beta_1,\ ^\circ$
%        
        & 32.2
%        
        & 74.5
%        
        & 144.0
%        
        \\
        \hline
%        
        $u,\ м/с$
%        
        & 283.7
%        
        & 378.2
%        
        & 472.8
%        
        \\
        \hline
%        
        $M_{c0}$
%        
        & 0.139
%        
        & 0.139
%        
        & 0.139
%        
        \\
        \hline
%        
        $M_{c1}$
%        
        & 0.813
%        
        & 0.612
%        
        & 0.494
%        
        \\
        \hline
%        
        $M_{w1}$
%        
        & 0.434
%        
        & 0.181
%        
        & 0.239
%        
        \\
        \hline
%        
        $T_1,\ К$
%        
        & 1276.2
%        
        & 1327.1
%        
        & 1351.6
%        
        \\
        \hline
%        
        $p_1,\ МПа$
%        
        & 0.3658
%        
        & 0.4345
%        
        & 0.471
%        
        \\
        \hline
%        
        $T_{1w}^*,\ К$
%        
        & 1311.5
%        
        & 1333.4
%        
        & 1362.9
%        
        \\
        \hline
%        
        $c_2,\ м/с$
%        
        & 128.5
%        
        & 127.0
%        
        & 125.6
%        
        \\
        \hline
%        
        $c_{2a},\ м/с$
%        
        & 126.3
%        
        & 124.0
%        
        & 122.5
%        
        \\
        \hline
%        
        $c_{2u},\ м/с$
%        
        & -23.9
%        
        & -27.5
%        
        & -28.0
%        
        \\
        \hline
%        
        $\alpha_2,\ ^\circ$
%        
        & 100.7
%        
        & 102.5
%        
        & 102.9
%        
        \\
        \hline
%        
        $w_2,\ м/с$
%        
        & 288.8
%        
        & 372.0
%        
        & 461.4
%        
        \\
        \hline
%        
        $w_{2a},\ м/с$
%        
        & 126.3
%        
        & 124.0
%        
        & 122.5
%        
        \\
        \hline
%        
        $w_{2u},\ м/с$
%        
        & 259.7
%        
        & 350.8
%        
        & 444.8
%        
        \\
        \hline
%        
        $\beta_2,\ ^\circ$
%        
        & 25.9
%        
        & 19.5
%        
        & 15.4
%        
        \\
        \hline
%        
        $M_{w2}$
%        
        & 0.419
%        
        & 0.54
%        
        & 0.669
%        
        \\
        \hline
%        
        $T_2,\ К$
%        
        & 1278.6
%        
        & 1278.8
%        
        & 1278.9
%        
        \\
        \hline
%        
        $p_2,\ МПа$
%        
        & 0.3688
%        
        & 0.369
%        
        & 0.3692
%        
        \\
        \hline
%        
        $p_2^*,\ МПа$
%        
        & 0.3772
%        
        & 0.3772
%        
        & 0.3772
%        
        \\
        \hline
%        
        $\pi_т$
%        
        & 1.491
%        
        & 1.491
%        
        & 1.49
%        
        \\
        \hline
%        
        $\pi_т^*$
%        
        & 1.458
%        
        & 1.458
%        
        & 1.458
%        
        \\
        \hline
%        
        $H_л,\ \frac{кДж}{кг}$
%        
        & -3.0
%        
        & 61.2
%        
        & 92.1
%        
        \\
        \hline
%        
        $H_0,\ \frac{кДж}{кг}$
%        
        & 153.7
%        
        & 153.5
%        
        & 153.4
%        
        \\
        \hline
%        

    \end{longtable}

%        




\end{document}