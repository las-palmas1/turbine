%%
%% Author: User1
%% 09.04.2018
%%

% Preamble
\documentclass[a4paper,10pt]{article}

% Packages
\usepackage{mathtext}
\usepackage[T2A]{fontenc}
\usepackage[utf8]{inputenc}
\usepackage{longtable}
\usepackage[russian]{babel}
\usepackage{amsmath}
\usepackage{amsfonts}
\usepackage{amssymb}
\usepackage{graphicx}
\usepackage[left=2cm,right=2cm,
    top=2cm,bottom=2cm,bindingoffset=0cm]{geometry}
\usepackage{color}
\usepackage{gensymb}

\usepackage{enumitem}
\setlist[enumerate]{label*=\arabic*.}

\usepackage{indentfirst}

\usepackage{titlesec}


% Document
\begin{document}

\section{Расчет охлаждения СА первой ступени.}

    

    \subsection{Описание алгоритма расчета температурного поля лопатки.}
    
    Алгоритм расчета распределения температуры поверхности лопатки следующий:
    \begin{enumerate}
        \item Лопатка разбивается в радиальном напрвлении на сектора, в пределах каждого из которых
        температуры газа на входе в решетку предполагается постоянной.
        \item Затем для каждого сектора по изложенному ниже алгоритму рассчитывается распределение температуры
        охлаждающего воздуха, предполагая, что температура пленки равна температуре на входе в решетку.
        \item Далее для кадого сектора по полученному распределению температуры воздуха рассчитывается
        температура пленки по описанной ниже методике.
        \item Затем заново рассчитывается температура охлаждающего воздуха и цикл рассчета повторяется до схождения к решению.
        \item После нахождения решения рассчитывается температура стенки лопатки с учетом теплопроводности в продольном направлении.
    \end{enumerate}
    Таким образом, описанный алгоритм позволяет учесть при расчете температурного поля лопатки неравномерность
    температруы  потока на входе в решетку, а также продольную теплопроводность материала лопатки, что позволяет
    получить непрерывную и гладкую эпюру температур для каждого сектора.
%    

    \subsection{Алгоритм расчета пленки.}

    

    \begin{enumerate}

        \item Исходные данные и допущения для алгоритма расчета пленки по одной стороне лопатки.

        \begin{enumerate}

            \item Параметры газа по высоте постоянны: $T_г^* = const$ и $p_г^* = const$.
            \item Координаты отверстий: $x_{отв\ i}$.
            \item Диаметры отверстий: $d_{отв\ i}$.
            \item Числа отверстий в рядах: $N_{отв\ i}$.
            \item Коэффициенты скорости в отверстиях: $\phi_{отв\ i}$.
            \item Коэффициенты расхода в отверстиях: $\mu_{отв\ i}$.
            \item Давление торможения воздуха на входе в канал: $p_{в0}^*$.
            \item Расход охраждающего воздуха на входе в канал: $G_{в0}$.
            \item Распределение коэффициента теплоотдачи со стороны газа вдоль профиля: $\alpha_г (x)$.
            \item Распределение температуры охлаждающего воздуха вдоль профиля: $T_в^* (x)$.
            \item Высота участка лопатки: $l$.

        \end{enumerate}

        \item Зададим распределение приведенной скорости по корыту $\lambda_к \left( \overline{x} \right)$ и
        спинке $\lambda_с \left( \overline{x} \right)$:

		\begin{gather*}
		    \lambda_к \left( \overline{x} \right) =
			\left\{
				1 +
				\left[
					\left(
						\frac{\lambda_1}{\lambda_0}
					\right)^{0.5}
				\right]\overline{x}
			\right\}^{2} \lambda_0, \/\ \overline{x} = \frac{x}{l_к}\\
		    \lambda_с \left( \overline{x} \right) =
			\left\{
				1 +
				\left[
					\left(
						\frac{\lambda_1}{\lambda_0}
					\right)^{4}
				\right]\overline{x}
			\right\}^{0.25}\lambda_0, \/\ \overline{x} = \frac{x}{l_с}
		,\\
		\end{gather*}
		где $l_к$ - длина профиля со стороны корыта, $l_с$ - длина профиля со стороны спинки,
        $\lambda_0$ - приведенная скорость на входе в лопаточный венец, $\lambda_1$ - приведенная
        скорость на выходе из лопаточного венца.

        \item Определим критическую скорость звука $a_{кр}$:
		\[
			a_{кр} = \sqrt{
				\frac{2k_г}{k_г + 1} R_г T_г^*
			}
		\]

        \item Определим скорость газа на корыте $v_к$ и на спинке $v_с$:
            \begin{gather*}
                v_к\left( x \right) = \lambda_к \left( \frac{x}{l_к} \right)\\
                v_c\left( x \right) = \lambda_к \left( \frac{x}{l_c} \right)\\
            \end{gather*}

        Дальнейший расчет идентичен для спинки и корыта, поэтому скорость газа будем обозначать как $v_г$.
        \item Определим эквивалентную ширину щели:
            \[
                s = N_{отв} \frac{\pi d_{отв}^2}{4} \cdot \frac{1}{l},
            \]
            где $N_{отв}$ - количество отверстий, $d_{отв}$ - диаметр отверстия, $l$ - высота профильной части лопатки.

        \item Определим скорость газа в точке выдува воздуха:
            \[
                v_{г \/\ отв} = v_г\left( x_{отв} \right),
            \]
            где $x_{отв}$ - криволинейная координата отверстия.

        \item Определим статическую температуру газа в точке выдува воздуха:
            \[
                T_{г \/\ отв} = T_г^* - \frac{v_{г \/\ отв}}{2 c_{p \/\ г}}
            \]

        \item Определим статическое давление газа в точке выдува воздуха:
            \[
                p_{г \/\ отв} = \frac{p_г^*}{
                    \left(
                        \frac{
                            T_г^*
                        }{
                            T_{г \/\ отв}
                        }
                    \right)^\frac{k_г}{k_г - 1}
                }
            \]

        \item Определим статическую плотность газа в точке выдува воздуха:
            \[
                \rho_{г \/\ отв} = \frac{
                    p_{г \/\ отв}
                }{
                    R_г \cdot T_{г \/\ отв}
                }
            \]

        \item Определим скорость истечения воздуха из отверстия:
            \[
                v_{в \/\ отв} = \phi_{отв} \sqrt{
                    \frac{2k_в}{k_в - 1}
                } R_в T_в^* \left( x_{отв} \right)
                \left[
                    1 -
                    \left(
                        \frac{
                            p_{г \/\ отв}
                        }{
                            p_{в0}^*
                        }
                    \right)^\frac{k_в - 1}{k_в}
                \right],
            \]
            где $\phi_{отв}$ - коэффициент скорости, $T_в^* \left( x_{отв} \right)$ -
            температура воздуха в точке выдува, $p_{в0}^*$ - давление воздуха.

        \item Определим статическую плотность воздуха на выходе из отверстия:
		\[
			\rho_{в \/\ отв} = \frac{
				p_{г \/\ отв}
			}{
				R_в
				\left[
					T_в^* \left( x_{отв} \right) - \frac{v_{в \/\ отв}^2}{2c_{p \/\ в}}
				\right]
			}
		\]

        \item Определим плотность торможения воздуха на входе в отверстия:
            \[
                \rho_{в \/\ отв}^* = \frac{p_{в0}^*}{R_в T_в^* \left( x_{отв} \right) }
            \]

        \item Определим параметр вдува:
            \[
                m = \frac{\rho_{в \/\ отв} v_{в \/\ отв}}{\rho_{г \/\ отв} v_{г \/\ отв}}
            \]

        \item Определим число Рейнольдса по ширине щели:
            \[
                Re_s = \frac{
                    \rho_{г \/\ отв} v_{г \/\ отв} s
                }{\mu_г\left( T_{г \/\ отв} \right)}
            \]

        \item Определим температурный фактор:
            \[
                \phi = T_в^* \left( x_{отв} \right) / T_г^*
            \]

        \item Определим эффективность пленки $\theta_{пл}\left( x \right)$:
            \[
                A\left( x \right) = Re_s^{-0.25} m^{-1.3} \phi^{-1.25}
                \left(
                    \frac{
                        x - x_{отв}
                    }{
                        s
                    }
                \right)
            \]
            \[
                \theta_{пл}\left( x \right) = \left\{
                    \begin{array}{@{}ll@{}}
                        1.0, & \text{если }\ 0 < A \leq 3 \\
                        \left( \frac{A}{3} \right)^{-0.285}, & \text{если } 3 \leq A < 11 \\
                        \left( \frac{A}{7.43} \right)^{-0.95}, & \text{если } A \geq 11 \\
                    \end{array}\right.
            \]

        \item Определим темперутуру пленки в случае нескольких рядов отверстий:
		\[
			T_{пл}^*\left( x \right) = \left\{
                \begin{array}{lc}
                    T_г^*, & \text{если}\ x < x_{отв\ 1} \\
                    T_г^* \cdot \prod_{i = 1}^{x_i \leq x}
                    \left[
                        \left(
                            1 - \theta_{пл \/\ i}
                        \right)
                    \right] + &\\
                    +\sum_{i = 1}^{x_i \leq x} \left[
                        \theta_{пл \/\ i}T_в^*\left( x_{отв \/\ j} \right)
                        \prod_{j = i + 1}^{x_j \leq x}
                        \left(
                            1 - \theta_{пл \/\ j}
                        \right)
                    \right], & \text{если}\ x_{отв\ 1} \leq x \\
                \end{array} \right.
		\]

        \item Определим коэффициент теплоотдачи пленки в случае нескольких рядов отверстий:
		\[
			\alpha_{пл}\left( x \right) = \left\{
                \begin{array}{lc}
                    \alpha_{г}(x), & \text{если }  0 \leq x < x_{отв\ 1}  \\
                    \alpha_{г}(x) \left(
                        1 + \frac{
                            2m_1
                        }{
                        \frac{
                            x - x_{отв \/\ 1}
                        }{s_1}
                    }
                    \right), & \text{если }  x_{отв\ 1} \leq x < x_{отв\ 2}  \\
                    .........\\
                    \alpha_{г}(x) \left(
                        1 + \frac{
                            2m_n
                        }{
                        \frac{
                            x - x_{отв \/\ n}
                        }{s_n}
                    }
                    \right), & \text{если }  x_{отв\ n} \leq x   \\
                \end{array}\right.
		\]

        \item По формуле истечения из сопла определим расход через ряд отверстий:
		\[
			G_{отв} = s \cdot l \cdot  \mu_{отв} \sqrt{
				\frac{2k_в}{k_в - 1} p_{в0}^*\rho_{в \/\ отв}^*
				\left(
					\frac{
						p_{г \/\ отв}
					}{
						p_{в0}^*
					}
				\right)^\frac{2}{k_в}
				\left[
					1 -
					\left(
						\frac{
							p_{г \/\ отв}
						}{
							p_{в0}^*
						}
					\right)^\frac{k_в - 1}{k_в}
				\right]
			}
		\]

        \item В общем случае зависимость расхода воздуха в зазоре от криволинейной координаты имеет вид:
		\[
			G_в \left( x \right) = G_{в0} - \sum_{i = 1}^{x_i \leq x} G_{отв \/\ i}
		\]

        \item Выходные данные расчета пленки:

        \begin{enumerate}

            \item Распределение вдоль профиля коэффициента теплоотдачи со стороны пленки: $\alpha_{пл} (x)$.
            \item Распределение температуры пленки вдоль профиля: $T_{пл} (x)$.
            \item Распределение расхода охлаждающего воздуха вдоль профиля: $G_в (x)$.

        \end{enumerate}

    \end{enumerate}

  %  

    \subsection{Алгоритм температуры водуха.}

    

    \begin{enumerate}

        \item Исходные данные и допущения для расчета локальных параметров с одной стороны лопатки:

        \begin{enumerate}
            \item Теплопроводность материала лопатки: $\lambda_м$.
            \item Толщина стенки лопатки: $\Delta$.
            \item Теплопроводность защитного покрытия: $\lambda_п$.
            \item Толщина покрытия: $\Delta_п$.
            \item Распределение вдоль профиля коэффициента теплоотдачи со стороны пленки: $\alpha_{пл} (x)$.
            \item Распределение температуры пленки вдоль профиля: $T_{пл} (x)$.
            \item Распределение расхода охлаждающего воздуха вдоль профиля: $G_в (x)$.
            \item Ширина канала для охлаждения лопатки: $\delta$.
            \item Высота участка лопатки: $l$.
            \item Теплопроводность воздуха в зависимости от температуры: $\lambda_в (T)$.
            \item Вязкость воздуха в зависимости от температуры: $\mu_в (T)$.
        \end{enumerate}

        \item Определим зависимость коэффициента теплотдачи со строны воздуха от температуры и координаты вдоль профиля:
        \[
            \alpha_в(T_в^*, x) = 0.02 \cdot \lambda_в \left( T_в^* \right) \cdot \frac{1}{2 \delta} \cdot
                \left(
                \frac{G_в }{l \cdot \mu \left( T_в^* \right)}
            \right)
        \]

        \item Определим зависимость коэффициента теплопередачи от температуры и координаты вдоль профиля:
        \[
            k (T_в^*, x) = \frac{1
                    }{
                \frac{1}{\alpha_в(T_в^*, x)} + \frac{1}{\alpha_{пл} (x)} + \frac{\Delta}{\lambda_м} +
                \frac{\Delta_п}{\lambda_п}
            }
        \]

        \item Получаем зависимость производной температуры воздуха от координаты от температуры и координаты
        вдоль профиля:
        \[
            \frac{dT_в^*}{x} = \frac{
                    k (T_в^*, x) (T_пл(x) - T_в^*) l
            }{
                    G_в(x) c_{pв}
            }
        \]

        \item В итоге получаем дифференциальное уравнение вида $\frac{dy}{dx} = f(y, x)$, которое решаем любым из
        известных методов численного решения ДУ, например, методом Эйлера.

        \item Выходные результаты расчета локальных параметров:

        \begin{enumerate}
            \item Рапределение температуры воздуха вдоль профиля: $T_в^*(x)$.
            \item Распределение коэффициента теплоотдачи со стороны воздуха вдоль профиля: $\alpha_в (x)$.
        \end{enumerate}

    \end{enumerate}

%    

    \subsection{Методика расчета температуры стенки.}
    
    Методика расчета температуры стенки предполагает известными распределения вдоль профиля коэффциента
    тепоотдачи от стенки к пленке $\alpha_{пл}(x)$, коэффициента теплоотадчи от воздуха к стенке $\alpha_в(х)$,
    температуры пленки $T_{пл}(x)$ и температуры охлаждающего воздуха $T_в(x)$.
    Метод основан на численном решении дифференциального уравнения теплового баланса для участка стенки,
    учитывающего теплопроводность только вдоль профиля.
    Оно имеет вид:

    \[
        \frac{d^2 T_{ст}}{dx^2} = \frac{1}{\lambda \Delta} \left[
                    \left( \alpha_{пл}(x) + \alpha_в(x) \right) T_{ст} -
                    \left( \alpha_{пл}(x) T_{пл}(x) + \alpha_в(x) T_в(x) \right)
        \right],
    \]
    где $T_{ст}$ - температура стенки, $\lambda$ - теплопроводность материала лопатки, $\Delta$ - толщина стенки.
    Граничные условия:

    \begin{enumerate}
        \item Граничное условие в лобовой точке - наличие экстремума температуры:
        \[
            \left. \frac{dT_{ст}}{dx} \right\|_0 = 0
        \]

        \item Граничное условие на выходной кросмке в общем случае более сложно, так как через нее
        существует тепловой поток. Однако в простейшем случае можно принять торец выходной кромки теплоизолированным,
        и в этом случае вторым граничным условием можно принять:
        \[
            \left. \frac{dT_{ст}}{dx} \right\|_l = 0
        \]
    \end{enumerate}

    Данна задача может быть приближенно решена методом конечных разностей.
    Для этого необходимо представить значение второй производной температуры стенки через значения температуры в
    узловых точках.
    Используя разложение в ряд Тейлора, получим представление второй производной через значения температуры в
    узловых точках для случая равномерного шага:
    \[
        \frac{d^2 T_{ст}}{dx^2}(x_i) = \frac{T_{ст\ i-1} - 2 \cdot T_{ст\ i} + T_{ст\ i+1}}{\Delta x^2},
    \]
    где индекс $i$ соответствует текущей точке расчета, а $\Delta x$ - шаг сетки.

    Таким образом, линеаризованной уравнение теплового баланса малого элемента лопатки имеет вид:
    \[
        A_{i-1} T_{ст\ i-1} + A_{i} T_{ст\ i} + A_{i+1} T_{ст\ i+1} = S,
    \]
    где
    \begin{gather*}
        A_{i-1} = \frac{1}{\Delta x^2},\\
        A_i = -\frac{2}{\Delta x^2} - \frac{\alpha_{пл} + \alpha_в}{\lambda \Delta},\\
        A_{i+1} =\frac{1}{\Delta x^2},\\
        S = - \frac{\alpha_{пл} T_{пл} + \alpha_в T_в}{\lambda \Delta}\\
    \end{gather*}
    Линеаризованные граничные условия имеют вид (для случая теплоизолированных торцев):
    \begin{gather*}
        T_{ст\ 1} - T_{ст\ 0} = 0,\\
        T_{ст\ N} - T_{ст\ N-1} = 0\\
    \end{gather*}
    Далее по полученной линеаризованной модели составляется матрица коэффициентов уравнений и столбец правых частей,
    и полученная система уравнений решается одним из стандартных методов решения СЛАУ.

%    

    \subsection{Алгоритм расчета коэффициент теплоотдачи со стороны газа.}

    
        \begin{enumerate}

            \item Исходные данные для расчета распределения коэффициента теплоотдачи со стороны газа:
            \begin{enumerate}
                \item Расход газа через участок решетки: $G_г$.
                \item Средний диаметр лопаточной решетки: $D_{ср}$.
                \item Высота участка лопатки: $l$.
                \item Длина хорды профиля: $b$.
                \item Длины спинки и корыта: $l_к$ и $l_с$.
                \item Длины участков входной кромки на спинке и корыте: $l_{вх\ c}$ и $l_{вх\ к}$.
                \item Углы профиля на входе и на выходе: $\alpha_{1л}$ и $\alpha_{2л}$.
                \item Радиус закругления входной кромки: $r_1$.
            \end{enumerate}

            \item Число Рейнольдса по газу:
            \[
                Re_г = \frac{G_г b }{\pi D_{ср} l \mu_г \sin{\alpha_{2л}}}
            \]

            \item Число Нуссельта:
            \[
                Nu_г = \left[ 0.07 + 100 \cdot ( \alpha_{1л} + \alpha_{2л} )^{-2} \right] \cdot Re_г ^{0.68}
            \]

            \item Средний коэффициент теплоотдачи со стороны газа:
            \[
                \alpha_{г\ ср} = \frac{Nu_г \lambda_г}{b}
            \]

            \item Коээфициент теплоотдачи со стороны газа на входной кромке:
            \[
                \alpha_{г\ вх} = \frac{0.74 \lambda_г}{2 r_1} \cdot
                    \left(
                        \frac{2 G_г r_1}{\pi D_{ср} l \sin{\alpha_{л}} \mu_г}
                \right) ^ {0.5}
            \]

            \item Распределение коэффициента теплоотдачи от газа по профилю:
            \[
                \alpha_г(x) = \left\{
                    \begin{array}{lc}
                        1.5 \alpha_{г\ ср}, & \text{если}\ -l_c \leq x < -l_с + \frac{b}{3} \\
                        0.6 \alpha_{г\ ср}, & \text{если}\ -l_с + \frac{b}{3} \leq x < -l_{вх\ c} \\
                        \alpha_{г\ вх}, & \text{если}\ -l_{вх\ c} \leq x < l_{вх\ к} \\
                        \alpha_{г\ ср}, & \text{если}\ l_{вх\ к} \leq x < l_к \\
                    \end{array}
                \right.
            \]

        \end{enumerate}
%    

    \subsection{Результаты расчета.}

    
    \begin{longtable}{
     |
%    
    c|
%    
    c|
%    
    c|
%    
    c|
%    
    c|
%    
    c|
%    
    c|
%    
    c|
%    
    c|
%    
    c|
%    
    }
        \caption{Значения параметров в отверстиях.} \\
        \hline
%        

%        
        $x,\ мм$
%        

%        

%        
        & $s,\ 10^{-3}\ мм$
%        

%        

%        
        & $\phi_{отв}$
%        

%        

%        
        & $\mu_{отв}$
%        

%        

%        
        & $m$
%        

%        

%        
        & $\phi$
%        

%        

%        
        & $G_{отв},\ г/с$
%        

%        

%        
        & $G_{отв}/G_{в0}$
%        

%        

%        
        & $v_в,\ м/с$
%        

%        
        \\
        \hline

%        

%        

%        
        -80.9
%        

%        

%        
        & 57.1
%        

%        

%        
        & 0.98
%        

%        

%        
        & 0.95
%        

%        

%        
        & 1.373
%        

%        

%        
        & 0.48
%        

%        

%        
        & 2.0
%        

%        

%        
        & 0.02
%        

%        

%        
        & 239.0
%        

%        
        \\
        \hline

        

%        

%        
        -40.4
%        

%        

%        
        & 57.1
%        

%        

%        
        & 0.98
%        

%        

%        
        & 0.95
%        

%        

%        
        & 1.363
%        

%        

%        
        & 0.473
%        

%        

%        
        & 1.736
%        

%        

%        
        & 0.017
%        

%        

%        
        & 197.0
%        

%        
        \\
        \hline

        

%        

%        
        0.0
%        

%        

%        
        & 28.5
%        

%        

%        
        & 0.98
%        

%        

%        
        & 0.95
%        

%        

%        
        & 0.63
%        

%        

%        
        & 0.464
%        

%        

%        
        & 0.142
%        

%        

%        
        & 0.001
%        

%        

%        
        & 29.0
%        

%        
        \\
        \hline

        

%        

%        
        0.0
%        

%        

%        
        & 28.5
%        

%        

%        
        & 0.98
%        

%        

%        
        & 0.95
%        

%        

%        
        & 0.63
%        

%        

%        
        & 0.464
%        

%        

%        
        & 0.142
%        

%        

%        
        & 0.001
%        

%        

%        
        & 29.0
%        

%        
        \\
        \hline

        

%        

%        
        30.2
%        

%        

%        
        & 57.1
%        

%        

%        
        & 0.98
%        

%        

%        
        & 0.95
%        

%        

%        
        & 1.129
%        

%        

%        
        & 0.474
%        

%        

%        
        & 0.744
%        

%        

%        
        & 0.007
%        

%        

%        
        & 79.0
%        

%        
        \\
        \hline

        

%        

%        
        90.7
%        

%        

%        
        & 57.1
%        

%        

%        
        & 0.98
%        

%        

%        
        & 0.95
%        

%        

%        
        & 1.322
%        

%        

%        
        & 0.49
%        

%        

%        
        & 1.521
%        

%        

%        
        & 0.015
%        

%        

%        
        & 176.0
%        

%        
        \\
        \hline

        
    \end{longtable}
%    

    
    \begin{longtable}{
     |
%    
    c|
%    
    c|
%    
    c|
%    
    c|
%    
    c|
%    
    c|
%    
    c|
%    
    c|
%    
    c|
%    
    c|
%    
    }
        \caption{Значения локальных параметров.} \\
        \hline
%        

%        
        $x,\ мм$
%        

%        

%        
        & $\alpha_{пл},\ \frac{Вт}{м^2 \cdot К}$
%        

%        

%        
        & $\alpha_{в},\ \frac{Вт}{м^2 \cdot К}$
%        

%        

%        
        & $\alpha_{г},\ \frac{Вт}{м^2 \cdot К}$
%        

%        

%        
        & $T_{в}^*,\ К$
%        

%        

%        
        & $T_{пл}^*,\ К$
%        

%        

%        
        & $T_{ст},\ К$
%        

%        

%        
        & $\theta_{пл}$
%        

%        

%        
        & $\theta_{охл}$
%        

%        
        \\
        \hline

%        

%        

%        
        -161.8
%        

%        

%        
        & 1079.5
%        

%        

%        
        & 1630.5
%        

%        

%        
        & 1077.4
%        

%        

%        
        & 706
%        

%        

%        
        & 1379.0
%        

%        

%        
        & 973
%        

%        

%        
        & 0.031
%        

%        

%        
        & 0.616
%        

%        
        \\
        \hline

        

%        

%        
        -153.8
%        

%        

%        
        & 1079.7
%        

%        

%        
        & 1627.8
%        

%        

%        
        & 1077.4
%        

%        

%        
        & 702
%        

%        

%        
        & 1377.0
%        

%        

%        
        & 971
%        

%        

%        
        & 0.034
%        

%        

%        
        & 0.615
%        

%        
        \\
        \hline

        

%        

%        
        -145.7
%        

%        

%        
        & 1080.0
%        

%        

%        
        & 1625.2
%        

%        

%        
        & 1077.4
%        

%        

%        
        & 698
%        

%        

%        
        & 1374.0
%        

%        

%        
        & 968
%        

%        

%        
        & 0.037
%        

%        

%        
        & 0.616
%        

%        
        \\
        \hline

        

%        

%        
        -137.7
%        

%        

%        
        & 1080.4
%        

%        

%        
        & 1622.5
%        

%        

%        
        & 1077.4
%        

%        

%        
        & 693
%        

%        

%        
        & 1371.0
%        

%        

%        
        & 964
%        

%        

%        
        & 0.041
%        

%        

%        
        & 0.617
%        

%        
        \\
        \hline

        

%        

%        
        -129.7
%        

%        

%        
        & 1080.9
%        

%        

%        
        & 1619.8
%        

%        

%        
        & 1077.4
%        

%        

%        
        & 689
%        

%        

%        
        & 1367.0
%        

%        

%        
        & 960
%        

%        

%        
        & 0.046
%        

%        

%        
        & 0.619
%        

%        
        \\
        \hline

        

%        

%        
        -121.7
%        

%        

%        
        & 1081.6
%        

%        

%        
        & 1617.1
%        

%        

%        
        & 1077.4
%        

%        

%        
        & 685
%        

%        

%        
        & 1362.0
%        

%        

%        
        & 954
%        

%        

%        
        & 0.053
%        

%        

%        
        & 0.624
%        

%        
        \\
        \hline

        

%        

%        
        -113.6
%        

%        

%        
        & 1082.6
%        

%        

%        
        & 1614.3
%        

%        

%        
        & 1077.4
%        

%        

%        
        & 681
%        

%        

%        
        & 1355.0
%        

%        

%        
        & 913
%        

%        

%        
        & 0.063
%        

%        

%        
        & 0.677
%        

%        
        \\
        \hline

        

%        

%        
        -105.6
%        

%        

%        
        & 433.7
%        

%        

%        
        & 1612.7
%        

%        

%        
        & 431.0
%        

%        

%        
        & 678
%        

%        

%        
        & 1344.0
%        

%        

%        
        & 829
%        

%        

%        
        & 0.078
%        

%        

%        
        & 0.792
%        

%        
        \\
        \hline

        

%        

%        
        -97.6
%        

%        

%        
        & 435.0
%        

%        

%        
        & 1611.3
%        

%        

%        
        & 431.0
%        

%        

%        
        & 676
%        

%        

%        
        & 1324.0
%        

%        

%        
        & 813
%        

%        

%        
        & 0.105
%        

%        

%        
        & 0.81
%        

%        
        \\
        \hline

        

%        

%        
        -89.6
%        

%        

%        
        & 438.8
%        

%        

%        
        & 1609.9
%        

%        

%        
        & 431.0
%        

%        

%        
        & 674
%        

%        

%        
        & 1271.0
%        

%        

%        
        & 797
%        

%        

%        
        & 0.178
%        

%        

%        
        & 0.83
%        

%        
        \\
        \hline

        

%        

%        
        -81.5
%        

%        

%        
        & 535.5
%        

%        

%        
        & 1608.9
%        

%        

%        
        & 431.0
%        

%        

%        
        & 673
%        

%        

%        
        & 728.0
%        

%        

%        
        & 773
%        

%        

%        
        & 0.923
%        

%        

%        
        & 0.862
%        

%        
        \\
        \hline

        

%        

%        
        -73.5
%        

%        

%        
        & 433.0
%        

%        

%        
        & 1663.1
%        

%        

%        
        & 431.0
%        

%        

%        
        & 671
%        

%        

%        
        & 1369.0
%        

%        

%        
        & 810
%        

%        

%        
        & 0.043
%        

%        

%        
        & 0.809
%        

%        
        \\
        \hline

        

%        

%        
        -65.5
%        

%        

%        
        & 433.6
%        

%        

%        
        & 1661.5
%        

%        

%        
        & 431.0
%        

%        

%        
        & 668
%        

%        

%        
        & 1360.0
%        

%        

%        
        & 811
%        

%        

%        
        & 0.055
%        

%        

%        
        & 0.805
%        

%        
        \\
        \hline

        

%        

%        
        -57.5
%        

%        

%        
        & 434.9
%        

%        

%        
        & 1660.0
%        

%        

%        
        & 431.0
%        

%        

%        
        & 666
%        

%        

%        
        & 1342.0
%        

%        

%        
        & 805
%        

%        

%        
        & 0.078
%        

%        

%        
        & 0.81
%        

%        
        \\
        \hline

        

%        

%        
        -49.4
%        

%        

%        
        & 438.4
%        

%        

%        
        & 1658.6
%        

%        

%        
        & 431.0
%        

%        

%        
        & 664
%        

%        

%        
        & 1296.0
%        

%        

%        
        & 792
%        

%        

%        
        & 0.142
%        

%        

%        
        & 0.827
%        

%        
        \\
        \hline

        

%        

%        
        -41.4
%        

%        

%        
        & 500.2
%        

%        

%        
        & 1657.5
%        

%        

%        
        & 431.0
%        

%        

%        
        & 662
%        

%        

%        
        & 805.0
%        

%        

%        
        & 763
%        

%        

%        
        & 0.806
%        

%        

%        
        & 0.864
%        

%        
        \\
        \hline

        

%        

%        
        -33.4
%        

%        

%        
        & 431.4
%        

%        

%        
        & 1703.7
%        

%        

%        
        & 431.0
%        

%        

%        
        & 660
%        

%        

%        
        & 1396.0
%        

%        

%        
        & 804
%        

%        

%        
        & 0.005
%        

%        

%        
        & 0.806
%        

%        
        \\
        \hline

        

%        

%        
        -25.4
%        

%        

%        
        & 431.6
%        

%        

%        
        & 1702.1
%        

%        

%        
        & 431.0
%        

%        

%        
        & 658
%        

%        

%        
        & 1395.0
%        

%        

%        
        & 807
%        

%        

%        
        & 0.007
%        

%        

%        
        & 0.8
%        

%        
        \\
        \hline

        

%        

%        
        -17.3
%        

%        

%        
        & 431.9
%        

%        

%        
        & 1700.5
%        

%        

%        
        & 431.0
%        

%        

%        
        & 656
%        

%        

%        
        & 1393.0
%        

%        

%        
        & 806
%        

%        

%        
        & 0.01
%        

%        

%        
        & 0.798
%        

%        
        \\
        \hline

        

%        

%        
        -9.3
%        

%        

%        
        & 432.6
%        

%        

%        
        & 1698.8
%        

%        

%        
        & 431.0
%        

%        

%        
        & 654
%        

%        

%        
        & 1387.0
%        

%        

%        
        & 818
%        

%        

%        
        & 0.017
%        

%        

%        
        & 0.78
%        

%        
        \\
        \hline

        

%        

%        
        -1.3
%        

%        

%        
        & 1454.0
%        

%        

%        
        & 1696.6
%        

%        

%        
        & 1414.7
%        

%        

%        
        & 650
%        

%        

%        
        & 1315.0
%        

%        

%        
        & 909
%        

%        

%        
        & 0.113
%        

%        

%        
        & 0.655
%        

%        
        \\
        \hline

        

%        

%        
        6.7
%        

%        

%        
        & 722.1
%        

%        

%        
        & 1698.7
%        

%        

%        
        & 718.3
%        

%        

%        
        & 653
%        

%        

%        
        & 1382.0
%        

%        

%        
        & 894
%        

%        

%        
        & 0.024
%        

%        

%        
        & 0.678
%        

%        
        \\
        \hline

        

%        

%        
        14.8
%        

%        

%        
        & 720.0
%        

%        

%        
        & 1701.0
%        

%        

%        
        & 718.3
%        

%        

%        
        & 657
%        

%        

%        
        & 1392.0
%        

%        

%        
        & 876
%        

%        

%        
        & 0.011
%        

%        

%        
        & 0.704
%        

%        
        \\
        \hline

        

%        

%        
        22.8
%        

%        

%        
        & 719.4
%        

%        

%        
        & 1703.4
%        

%        

%        
        & 718.3
%        

%        

%        
        & 660
%        

%        

%        
        & 1395.0
%        

%        

%        
        & 873
%        

%        

%        
        & 0.007
%        

%        

%        
        & 0.712
%        

%        
        \\
        \hline

        

%        

%        
        30.8
%        

%        

%        
        & 880.6
%        

%        

%        
        & 1685.2
%        

%        

%        
        & 718.3
%        

%        

%        
        & 663
%        

%        

%        
        & 785.0
%        

%        

%        
        & 820
%        

%        

%        
        & 0.834
%        

%        

%        
        & 0.787
%        

%        
        \\
        \hline

        

%        

%        
        38.8
%        

%        

%        
        & 729.1
%        

%        

%        
        & 1687.0
%        

%        

%        
        & 718.3
%        

%        

%        
        & 666
%        

%        

%        
        & 1325.0
%        

%        

%        
        & 859
%        

%        

%        
        & 0.102
%        

%        

%        
        & 0.736
%        

%        
        \\
        \hline

        

%        

%        
        46.9
%        

%        

%        
        & 723.9
%        

%        

%        
        & 1689.1
%        

%        

%        
        & 718.3
%        

%        

%        
        & 669
%        

%        

%        
        & 1359.0
%        

%        

%        
        & 875
%        

%        

%        
        & 0.056
%        

%        

%        
        & 0.718
%        

%        
        \\
        \hline

        

%        

%        
        54.9
%        

%        

%        
        & 722.0
%        

%        

%        
        & 1691.3
%        

%        

%        
        & 718.3
%        

%        

%        
        & 672
%        

%        

%        
        & 1372.0
%        

%        

%        
        & 881
%        

%        

%        
        & 0.039
%        

%        

%        
        & 0.713
%        

%        
        \\
        \hline

        

%        

%        
        62.9
%        

%        

%        
        & 721.1
%        

%        

%        
        & 1693.6
%        

%        

%        
        & 718.3
%        

%        

%        
        & 675
%        

%        

%        
        & 1378.0
%        

%        

%        
        & 885
%        

%        

%        
        & 0.03
%        

%        

%        
        & 0.711
%        

%        
        \\
        \hline

        

%        

%        
        70.9
%        

%        

%        
        & 720.6
%        

%        

%        
        & 1695.8
%        

%        

%        
        & 718.3
%        

%        

%        
        & 678
%        

%        

%        
        & 1382.0
%        

%        

%        
        & 888
%        

%        

%        
        & 0.025
%        

%        

%        
        & 0.709
%        

%        
        \\
        \hline

        

%        

%        
        78.9
%        

%        

%        
        & 720.2
%        

%        

%        
        & 1698.0
%        

%        

%        
        & 718.3
%        

%        

%        
        & 681
%        

%        

%        
        & 1385.0
%        

%        

%        
        & 890
%        

%        

%        
        & 0.021
%        

%        

%        
        & 0.71
%        

%        
        \\
        \hline

        

%        

%        
        87.0
%        

%        

%        
        & 719.9
%        

%        

%        
        & 1700.2
%        

%        

%        
        & 718.3
%        

%        

%        
        & 685
%        

%        

%        
        & 1387.0
%        

%        

%        
        & 876
%        

%        

%        
        & 0.018
%        

%        

%        
        & 0.732
%        

%        
        \\
        \hline

        

%        

%        
        95.0
%        

%        

%        
        & 743.5
%        

%        

%        
        & 1659.4
%        

%        

%        
        & 718.3
%        

%        

%        
        & 687
%        

%        

%        
        & 1196.0
%        

%        

%        
        & 842
%        

%        

%        
        & 0.285
%        

%        

%        
        & 0.782
%        

%        
        \\
        \hline

        

%        

%        
        103.0
%        

%        

%        
        & 727.1
%        

%        

%        
        & 1661.3
%        

%        

%        
        & 718.3
%        

%        

%        
        & 690
%        

%        

%        
        & 1319.0
%        

%        

%        
        & 878
%        

%        

%        
        & 0.114
%        

%        

%        
        & 0.735
%        

%        
        \\
        \hline

        

%        

%        
        111.0
%        

%        

%        
        & 723.6
%        

%        

%        
        & 1663.3
%        

%        

%        
        & 718.3
%        

%        

%        
        & 693
%        

%        

%        
        & 1347.0
%        

%        

%        
        & 890
%        

%        

%        
        & 0.075
%        

%        

%        
        & 0.721
%        

%        
        \\
        \hline

        

%        

%        
        119.1
%        

%        

%        
        & 722.1
%        

%        

%        
        & 1665.3
%        

%        

%        
        & 718.3
%        

%        

%        
        & 696
%        

%        

%        
        & 1360.0
%        

%        

%        
        & 896
%        

%        

%        
        & 0.057
%        

%        

%        
        & 0.715
%        

%        
        \\
        \hline

        

%        

%        
        127.1
%        

%        

%        
        & 721.3
%        

%        

%        
        & 1667.4
%        

%        

%        
        & 718.3
%        

%        

%        
        & 699
%        

%        

%        
        & 1367.0
%        

%        

%        
        & 901
%        

%        

%        
        & 0.047
%        

%        

%        
        & 0.712
%        

%        
        \\
        \hline

        

%        

%        
        135.1
%        

%        

%        
        & 720.7
%        

%        

%        
        & 1669.4
%        

%        

%        
        & 718.3
%        

%        

%        
        & 702
%        

%        

%        
        & 1372.0
%        

%        

%        
        & 904
%        

%        

%        
        & 0.04
%        

%        

%        
        & 0.711
%        

%        
        \\
        \hline

        

%        

%        
        143.1
%        

%        

%        
        & 720.4
%        

%        

%        
        & 1671.5
%        

%        

%        
        & 718.3
%        

%        

%        
        & 705
%        

%        

%        
        & 1376.0
%        

%        

%        
        & 907
%        

%        

%        
        & 0.035
%        

%        

%        
        & 0.71
%        

%        
        \\
        \hline

        

%        

%        
        151.2
%        

%        

%        
        & 720.1
%        

%        

%        
        & 1673.5
%        

%        

%        
        & 718.3
%        

%        

%        
        & 708
%        

%        

%        
        & 1378.0
%        

%        

%        
        & 909
%        

%        

%        
        & 0.031
%        

%        

%        
        & 0.71
%        

%        
        \\
        \hline

        
    \end{longtable}
%    

    \subsection{Интегральные параметры охлаждения.}

    
    \begin{enumerate}

        \item Температура охлаждающего воздуха на входе в канал:
            $ T_{в0}^{*} = 650\ К$.
        \item Полное давление охлаждающего воздуха на входе в канаЛ:
            $ p_{в0}^* = 0.5445\ МПа $.
        \item Суммарный расход воздуха на одну лопатку: $G_{в0} = 0.2\ кг/с$.
        \item Относительный расход воздуха на охлаждение СА:
            $g_{охл\ са} = 0.096$.
        \item Толщина стенок: $\Delta = 1.0\ мм$.
        \item Ширина канала: $\delta = 1.0\ мм$.
        \item Теплопроводность покрытия: $\lambda_п = 2\ Вт/(м \cdot К)$.
        \item Толщина покрытия: $\Delta_п = 0\ мм$.
        \item Средняя эффективность охлаждения: $\theta_{охл\ ср} = 0.735$.
    \end{enumerate}
    

\end{document}