\documentclass[a4paper,10pt]{article}
\usepackage{mathtext}
\usepackage[T2A]{fontenc}
\usepackage[utf8]{inputenc}
\usepackage[russian]{babel}
\usepackage{amsmath}
\usepackage{amsfonts}
\usepackage{amssymb}
\usepackage{graphicx}
\usepackage[left=2cm,right=2cm,
    top=2cm,bottom=2cm,bindingoffset=0cm]{geometry}
\usepackage{color}
\usepackage{gensymb}

\usepackage{enumitem}
\setlist[enumerate]{label*=\arabic*.}

\usepackage{indentfirst}

\usepackage{titlesec}

% Document
\begin{document}

    \begin{enumerate}

        </ macro stage_core(stgd, sttype='heat_drop') />
        <# sttype can be equal to 'heat_drop', or 'pressure', or 'work' #>

        </ set degr = 180 / 3.141592653589793 />

        \item Определим теплоперепад на сопловом аппарате:

        \[
            H_с = \left( 1 - \rho \right) H_0 =
	        \left( 1 - << stgd.rho | round(3) >> \right) \cdot << (stgd.H0 / 10**6) | round(4) >> \cdot 10^6 =
            << (stgd.H_s / 10**6) | round(4) >> \cdot 10^6 \/\ Дж/кг
        \]

        \item Окружная скорость на диаметре $ D_1 $:

        \[
            u_1 = \frac{\pi D_1 n }{60} =
                \frac{\pi \cdot << stgd.D1 | round(4) >> \cdot << stgd.n >>}{60} =
            << stgd.u1 | round(2) >>\ м/с
        \]

        \item Определим действительную скорость истечения из СА:

	    \[
            c_1 = \phi \sqrt{2 H_с} =
	        << stgd.phi | round(3) >> \cdot\sqrt{2 \cdot << (stgd.H0 / 10**6) | round(4) >> \cdot 10^6}  =
            << stgd.c1 | round(2) >> \/\ м/с
        \]

        \item Определим температуру на выходе из СА:

	    \[
            T_1 = T_0^* - \frac{ H_с \phi^2 }{ c_{pг} } =
	        << stgd.T0_stag | round(2) >> -
            \frac{
                << (stgd.H_s / 10**6) | round(4) >> \cdot 10^6 \cdot {<< stgd.phi | round(3) >>}^2
            }{
                2 \cdot << stgd.c_p_gas | round(2) >>
            } = << stgd.T1 | round(2) >> \/\ К
        \]

	    \item Определим температуру конца адиабатного расширения:

	    \[
            T_1^\prime = T_0^* - \frac{ H_c }{ c_{pг} } =
	        << stgd.T0_stag | round(2) >> -
            \frac{
                << (stgd.H_s / 10**6) | round(4) >> \cdot 10^6
            }{
                << stgd.c_p_gas | round(2) >>
            }
            = << stgd.T1_ad | round(2) >>  \/\ К
        \]

        \item Определим давление на выходе из СА:

	    \[
            p_1 = p_0^* \left(
                                \frac{ T_1^\prime }{ T_0^* }
                        \right)^
                    \frac{ k_г }{ k_г - 1 } =
            << (stgd.p0_stag / 10**6) | round(4) >> \cdot 10^6 \cdot
                \left(
                        \frac{ << stgd.T1_ad | round(2) >> }{ << stgd.T0_stag | round(2) >> }
                \right)^
                \frac{ << stgd.k_gas | round(4) >> }{ << stgd.k_gas | round(4) >> - 1 } =
            << (stgd.p1 / 10**6) | round(4) >> \cdot 10^6 \/\ МПа
        \]

        \item Определим площадь на выходе из СА:

	    \[
            A_{1a} = \pi l_1 D_1 =
	        \pi \cdot << stgd.l1 | round(4) >> \cdot << stgd.D1 | round(4) >> =
            << stgd.A1_a | round(5) >> \/\ м^2
        \]

        \item Определим плотность газа на выходе из СА:

	    \[
            \rho_1 = \frac{p_1}{R_г T_1} =
	        \frac{
                << (stgd.p1 / 10**6) | round(4) >> \cdot 10^6
            }{
                << stgd.work_fluid.R >> \cdot << stgd.T1 | round(2) >>
            } =
            << stgd.rho1 | round(3) >> \/\ кг/м^3
        \]

        \item Осевая составляющая абсолютной скорости на выходе из СА:

        \[
            c_{1a} = \frac{G_{вх} }{ \rho_1 A_{1a} } =
                \frac{
                    << stgd.G_stage_in | round(2) >>
                }{
                    << stgd.rho1 | round(3) >> \cdot << stgd.A1_a | round(5) >>
                } =
            << stgd.c1_a | round(2) >>\ м/с
        \]

        \item Угол потока в абсолютном движении после СА:

        \[
            \alpha_1 = \arcsin{ \frac{ c_{1a} }{ c_1 } } =
            \arcsin{ \frac{ << stgd.c1_a | round(2) >> }{ << stgd.c1 | round(2) >> } } =
            = << (stgd.alpha1 * degr) | round(3) >> \degree
        \]

        \item Окружная составляющая абсолютной скорости на входе:

        \[
            c_{1u} = c_1 \cos{\alpha_1} = << stgd.c1 | round(2) >> \cdot \cos{<< (stgd.alpha1 * degr) | round(3) >> \degree} =
            << stgd.c1_u | round(2) >>\ м/с
        \]

        \item Определим относительную скорость на входе в РК:

	    \begin{gather*}
	        w_1 = \sqrt{c_1^2 + u_1^2 - 2 c_1 u_1 \cos \alpha_1} =\\
	        = \sqrt{
            << stgd.c1 | round(2) >> ^ 2 +
            << stgd.u1 | round(2) >> ^ 2 -
            2 \cdot << stgd.c1 | round(2) >> \cdot << stgd.u1 | round(2) >> \cdot \cos 16.<< (stgd.alpha1 * degr) | round(3) >> \degree
            }
            = << stgd.w1 | round(2) >> \/\ м/с\\
	    \end{gather*}

        \item Угол потока в относительном движении:

        </ if stgd.c1_u - stgd.u1 >= 0 />
        \[
            \beta_1 = \arctan{ \frac{c_{1a}}{c_{1u} - u_1} } =
                    \arctan{ \frac{ << stgd.c1_a | round(2) >> }{<< stgd.c1_u | round(2) >> - << stgd.u1 | round(2) >>} } =
            << (stgd.beta1 * degr) | round(3) >> \degree
        \]
        </ else />
        \[
            \beta_1 = \pi + \arctan{ \frac{c_{1a}}{c_{1u} - u_1} } =
                    \pi + \arctan{ \frac{ << stgd.c1_a | round(2) >> }{<< stgd.c1_u | round(2) >> - << stgd.u1 | round(2) >>} } =
            << (stgd.beta1 * degr) | round(3) >> \degree
        \]
        </ endif />

        \item Осевая составляющая относительной скорости:

        \[
            w_{1a} = w_1 \sin{\beta_1} = << stgd.w1 | round(2) >> \cdot  \sin{<< (stgd.beta1 * degr) | round(3) >> \degree} =
            << stgd.w1_a | round(2) >>\ м/с
        \]

        \item Окружная составляющая относительной скорости:

        \[
            w_{1u} = w_1 \cos{\beta_1} = << stgd.w1 | round(2) >> \cdot  \cos{<< (stgd.beta1 * degr) | round(3) >> \degree} =
            << stgd.w1_u | round(2) >>\ м/с
        \]

         \item Определим теплоперепад на РК:

	    \[
            H_л = H_0 \rho \frac{T_1}{T_1^\prime} =
	        << (stgd.H0 / 10**6) | round(4) >> \cdot 10^6 \cdot << stgd.rho | round(4) >> \cdot
            \frac{ << stgd.T1 | round(2) >> }{ << stgd.T1_ad | round(2) >> } =
            << (stgd.H_l / 10**6) | round(4) >> \cdot 10^6 \/\ Дж/кг
        \]

        \item Окружная скорость на диаметре:

        \[
            u_2 = \frac{ \pi D_2 n }{ 60 } =
                    \frac{ \pi \cdot << stgd.D2 | round(4) >> \cdot << stgd.n | round(2) >> }{ 60 } =
            << stgd.u2 | round(2) >>\ м/с
        \]

        \item Температура торможения в относительном движении после СА:

        \[
            T_{1w}^* = T_1 + \frac{ w_1^2 }{ 2 \cdot c_{pг}} =
                << stgd.T1 | round(2) >> + \frac{ << stgd.w1 | round(2) >> ^ 2 }{ 2 \cdot << stgd.c_p_gas | round(2) >>}
        \]

        \item Определим относительную скорость истечения газа из РК:

	    \begin{gather*}
	        w_2 = \psi \sqrt{w_1^2 + 2H_л +\left( u_2^2 - u_1^2 \right)} =\\
	        = << stgd.psi | round(3) >> \cdot
            \sqrt{
                << stgd.w1 | round(2) >> ^ 2 +
                2 \cdot << (stgd.H_l / 10**6) | round(4) >> \cdot 10^6 +
                \left( << stgd.u2 | round(2) >> ^ 2 - << stgd.u1 | round(2) >> ^ 2 \right)
            } =
            << stgd.w2 | round(2) >> \/\ м/с\\
	    \end{gather*}

        \item Определим статическую температуру на выходе из РК:

	    \begin{gather*}
	        T_2 = T_1 + \frac{
	 	        \left( w_1^2  - w_2^2 \right) + \left( u_2^2 - u_1^2 \right)
            }{
                2 c_{pг}
            } =\\
	        = << stgd.T1 | round(2) >> + \frac{
	 	        \left( << stgd.w1 | round(2) >> ^ 2  - << stgd.w2 | round(2) >> ^ 2 \right) +
                \left( << stgd.u2 | round(2) >> ^ 2 - << stgd.u1 | round(2) >> ^ 2 \right)
	        }{
            2 \cdot << stgd.c_p_gas | round(2) >>
            }
            = << stgd.T2 | round(2) >> \/\ К\\
	    \end{gather*}

        \item Определим статическую температуру при адиабатическом процессе в РК:

	    \[
            T_2^\prime = T_1 - \frac{
	 	        H_л
	        }{ c_{p г}} =
	        << stgd.T1 | round(2) >> - \frac{
	 	        << (stgd.H_l / 10**6) | round(4) >> \cdot 10^6
	        }{
                << stgd.c_p_gas | round(2) >>
            }
            = << stgd.T2_ad | round(2) >> \/\ К
        \]

        \item Определим давление на выходе из РК:

	    \[
            p_2 = p_1 \left( \frac{T_2^\prime}{T_1} \right)^{\frac{k_г}{k_г - 1}} =
               << (stgd.p1 / 10**6) | round(4) >> \cdot 10^6 \cdot
               \left(
               \frac{ << stgd.T2_ad | round(2) >> }{ << stgd.T1 | round(2) >> }
               \right) ^
               {\frac{
               << stgd.k_gas | round(3) >>
               }{
               << stgd.k_gas | round(3) >> - 1
               }}
            = << (stgd.p2 / 10**6) | round(4) >> \cdot 10^6 \/\ Па
        \]

        \item Определим плотность газа на выходе из РК:
	    \[
            \rho_2 = \frac{p_2}{R T_2} =
                \frac{
                    << (stgd.p2 / 10**6) | round(4) >> \cdot 10^6
                }{
                    << stgd.work_fluid.R >> \cdot << stgd.T2 | round(2) >>
                }
            = << stgd.rho2 | round(3) >>\ кг/м^3
        \]

        \item Определим площадь на выходе из РК:
        \[
            A_{2a} = \pi D_2 l_2 = \pi \cdot << stgd.D2 | round(4) >> \cdot << stgd.l2 | round(4) >> =
            << stgd.A2_a | round(4) >>\ м^2
        \]

        \item Осевая составляющая абсолютной скорости на выходе из РК:
        \[
            c_{2a} = \frac{ G_{вх} }{ A_{2a} \rho_2 } =
            \frac{ << stgd.G_stage_in | round(2) >> }{ << stgd.A2_a | round(4) >> \cdot << stgd.rho2 | round(3) >> }
            = << stgd.c2_a | round(2) >>\ м/с
        \]

        \item Угол потока в относительном движении на выходе из РК:
        \[
            \beta_2 = \arcsin{ \frac{ c_{2a} }{ w_2 } } =
                    \arcsin{ \frac{ << stgd.c2_a | round(2) >> }{ << stgd.w2 | round(2) >> } }
            = << (stgd.beta2 * degr) | round(3) >> \degree
        \]

        \item Осевая составляющая относительной скорости потока на выходе из РК:
        \[
            w_{2a} = w_2 \cdot \sin{\beta_2} =
                    << stgd.w2 | round(2) >> \cdot \sin{<< (stgd.beta2 * degr) | round(3) >> \degree}
            = << stgd.w2_a | round(2) >>\ м/с
        \]

        \item Окружная составляющая относительной скорости потока на выходе из РК:
        \[
            w_{2u} = w_2 \cdot \cos{\beta_2} =
                    << stgd.w2 | round(2) >> \cdot \cos{<< (stgd.beta2 * degr) | round(3) >> \degree}
            = << stgd.w2_u | round(2) >>\ м/с
        \]

        \item Определим окружную составляющую скорости на выходе из РК:
	    \[
            c_{2u} = w_{2u} - u_2 =
	        << stgd.w2_u | round(2) >> - << stgd.u2 | round(2) >> = << stgd.c2_u | round(2) >> \/\ м/с
        \]

        \item Опеределим угол потока на выходе из РК:
        </ if stgd.c2_u >= 0 />
        \[
            \alpha_2 = \arctan{ \frac{ c_{2a} }{ c_{2u} } } =
                    \arctan{ \frac{ << stgd.c2_a | round(2) >> }{ << stgd.c2_u | round(2) >> } } =
            << (stgd.alpha2 * degr) | round(3) >> \degree
        \]
        </ else />
        \[
            \alpha_2 = \pi + \arctan{ \frac{ c_{2a} }{ c_{2u} } } =
                    \pi + \arctan{ \frac{ << stgd.c2_a | round(2) >> }{ << stgd.c2_u | round(2) >> } } =
            << (stgd.alpha2 * degr) | round(3) >> \degree
        \]
        </ endif />

        \item Определим скорость потока на выходе из РК:
	    \[
            c_2 = \sqrt{c_{2u}^2 + c_{2a}^2} =
                \sqrt{<< stgd.c2_u | round(2) >> ^ 2 + << stgd.c2_a | round(2) >> ^ 2} =
            << stgd.c2 | round(2) >> \/\ м/с
        \]

        \item Определим работу на окружности колеса:
	    \[
            L_u = c_{1u} u_1 + c_{2u} u_2 =
                    << stgd.c1_u | round(2) >> \cdot << stgd.u1 | round(2) >> +
                    << stgd.c2_u | round(2) >> \cdot << stgd.u2 | round(2) >> =
            << (stgd.L_u / 10**6) | round(4) >> \cdot 10^6 \/\ Дж/кг
        \]

        \item Определим КПД на окружности колеса:
	    \[
            \eta_u = \frac{L_u}{H_0} =
                \frac{ << (stgd.L_u / 10**6) | round(4) >> \cdot 10^6 }{ << (stgd.H0 / 10**6) | round(4) >> \cdot 10^6 }
            = << stgd.eta_u | round(4) >>
        \]

        \item Определим удельные потери в СА:
	    \[
            h_с = \left(
                        \frac{ 1 }{ \phi^2 } - 1
                \right)
                \frac{ c_1^2 }{ 2 } =
	        \left(
                \frac{ 1 }{ << stgd.phi | round(3) >> ^ 2} - 1
            \right) \cdot
            \frac{ << stgd.c1 | round(2) >> ^ 2 }{ 2 } = << (stgd.h_s / 10**3) | round(4) >> \cdot 10^3 \/\ Дж/кг
        \]

        \item Удельные потери в СА с учетом их использования в рабочих лопатках:
        \[
            h_с^\prime = h_с \frac{ T_2^\prime }{ T_1 } =
                << (stgd.h_s / 10**3) | round(4) >> \cdot 10^3 \cdot
                \frac{ << stgd.T2_ad | round(2) >> }{ << stgd.T1 | round(2) >> } =
            << (stgd.h_s_touch / 10**3) | round(4) >> \cdot 10^3 \/\ Дж/кг
        \]

        \item Относительные потери в СА:
        \[
            \zeta_с = \frac{ h_с }{ H_0 } =
                \frac{ << (stgd.h_s / 10**3) | round(4) >> \cdot 10^3 }{ << (stgd.H0 / 10**6) | round(4) >> \cdot 10^6 } =
            << stgd.zeta_s | round(4) >>
        \]

        \item Относительные потери в СА с учетом их использования в рабочих лопатках:
        \[
            \zeta_с^\prime = \frac{ h_с^\prime }{ H_0 } =
                \frac{ << (stgd.h_s_touch / 10**3) | round(4) >> \cdot 10^3 }{ << (stgd.H0 / 10**6) | round(4) >> \cdot 10^6 } =
            << stgd.zeta_s_touch | round(4) >>
        \]

        \item Удельные потери в рабочих лопатках:
        \[
            h_л = \left(
                    \frac{ 1 }{ \psi^2 } - 1
                \right)) \cdot
                \frac{ w_2^2 }{ 2 } =
            \left(
                \frac{ 1 }{ << stgd.psi | round(2) >> ^ 2 } - 1
            \right) \cdot
            \frac{ << stgd.w2 | round(2) >> ^ 2} {2}
            = << (stgd.h_l / 10**3) | round(4) >> \cdot 10^3 \/\ Дж/кг
        \]

        \item Относительные потери в рабочих лопатках:
        \[
            \zeta_л = \frac{ h_л }{ H_0 } =
                \frac{ << (stgd.h_l / 10**3) | round(4) >> \cdot 10^3 }{ << (stgd.H0 / 10**6) | round(4) >> \cdot 10^6 } =
            << stgd.zeta_l | round(4) >>
        \]

        \item Определим удельные потери с выходной скоростью:
        \[
            h_{вых} = \frac{ c_2 ^ 2 }{ 2} =
                    \frac{ << stgd.c2 | round(2) >> ^ 2 }{ 2 } =  << (stgd.h_v / 10**3) | round(4) >> \cdot 10^3 \/\ Дж/кг
        \]

        \item Относительные потери с выходной скоростью:
        \[
            \zeta_{вых} = \frac{ h_{вых} }{ H_0 } =
                \frac{ << (stgd.h_v / 10**3) | round(4) >> \cdot 10^3 }{ << (stgd.H0 / 10**6) | round(4) >> \cdot 10^6 } =
            << stgd.zeta_v | round(4) >>
        \]

        \item Проверка КПД на окружности колеса:
        \[
            \eta_u = 1 - \zeta_с^\prime - \zeta_л - \zeta_{вых} = 1 - << stgd.zeta_s_touch | round(4) >> -
                    << stgd.zeta_l | round(4) >> - << stgd.zeta_v | round(4) >> = << stgd.eta_u_check | round(4) >>
        \]

        \item Средний диаметр:
        \[
            D_{ср} = 0.5 \cdot (D_1 + D_2) =
                    0.5 \cdot (<< stgd.D1 | round(4) >> + << stgd.D2 | round(4) >>) =
            << stgd.D_av | round(4) >>\ м
        \]

        \item Определим удельные потери в радиальном зазоре:

	    \begin{gather*}
	        h_з = 1.37 \cdot
                \left(
                    1 + 1.6 \rho
                \right)
                \left(
                    1 + \frac{l_2}{D_{ср}}
                \right)
            \frac{ \delta_r }{ l_2 } \cdot L_u =\\
	        = 1.37 \cdot
            \left(
                1 + 1.6 \cdot << stgd.rho | round(2) >>
            \right)
            \left(
                1 + \frac{ << stgd.l2 | round(4) >> }{ << stgd.D_av | round(4) >> }
            \right)
            \frac{ << stgd.delta_r_rk | round(5) >> }{ << stgd.l2 | round(4) >> } \cdot
            << (stgd.L_u / 10**6) | round(4) >> \cdot 10^6 =
	        << (stgd.h_z / 10**3) | round(4) >> \cdot 10^3 \/\ Дж/кг\\
	    \end{gather*}

        \item Относительные удельные потери в радиальном зазоре:
        \[
            \zeta_з = \frac{ h_з }{ H_0 } =
                \frac{ << (stgd.h_z / 10**3) | round(4) >> \cdot 10^3 }{ << (stgd.H0 / 10**6) | round(4) >> \cdot 10^6 } =
            << stgd.zeta_z | round(4) >>
        \]

        \item Удельная работа ступени с учетом потери в радиальном зазоре:
        \[
            L_{uз} = L_u - h_з = << (stgd.L_u / 10**6) | round(4) >> \cdot 10^6 -
                << (stgd.h_z / 10**3) | round(4) >> \cdot 10^3 =
            << (stgd.L_uz / 10**6) | round(4) >> \cdot 10^6 \ Дж/кг
        \]

        \item Мощностной КПД ступени:
        \[
            \eta_т^\prime = \eta_u - \zeta_з =
                << stgd.eta_u | round(4) >> - << stgd.zeta_z | round(4) >> = << stgd.eta_t_touch | round(4) >>
        \]

        \item Лопаточный КПД ступени:
        \[
            \eta_л^\prime = \eta_т^\prime + \zeta_{вых} =
                 << stgd.eta_t_touch | round(4) >> +  << stgd.zeta_v | round(4) >> =
            << stgd.eta_l_touch | round(4) >>
        \]

        \item Средняя длина лопатки:
        \[
            l_{ср} = 0.5 \cdot (l_1 + l_2) =
                0.5 \cdot (<< stgd.l1 | round(4) >> + << stgd.l2 | round(4) >>) =
            << stgd.l_av | round(4) >>\ м
        \]

        \item Средняя окружная скоротсь:
        \[
            u_{ср} = 0.5 \cdot (u_1 + u_2) =
                0.5 \cdot (<< stgd.u1 | round(2) >> + << stgd.u2 | round(2) >>) =
            << stgd.u_av | round(2) >>\ м/с
        \]

        \item Затраты мощности на трение и вентиляцию:
        \begin{gather*}
            N_{т.в} = \left[
                    1.07 \cdot D_{av}^2 + 61 \cdot (1 - \varepsilon) \cdot D_{av} l_{av}
            \right] \cdot
            \left(
                \frac{ u_{av} }{ 100 }
            \right) ^ 3 \cdot
            \rho =\\
            = \left[
                1.07 \cdot << stgd.D_av | round(4) >>^2 +
                61 \cdot (1 - << stgd.epsilon | round(3) >>) \cdot
                << stgd.D_av | round(4) >> \cdot << stgd.l_av | round(4) >>
            \right] \cdot
            \left(
                \frac{ << stgd.u_av | round(2) >> }{ 100 }
            \right) ^ 3 \cdot
            << stgd.rho | round(4) >>=\\
            = << (stgd.N_tv / 10**3) | round(4) >> \cdot 10^3 \ Вт \\
        \end{gather*}

        \item Удельные потери на трение и вентиляцию:
        \[
            h_{т.в} = \frac{ N_{т.в} }{ G_{вх} } =
                \frac{
                    << (stgd.N_tv / 10**3) | round(4) >> \cdot 10^3
                }{
                    << stgd.G_stage_in | round(2) >>
                }
            = << (stgd.h_tv / 10**3) | round(4) >> \cdot 10^3 \ Дж/кг
        \]

        \item Относительные потери на трение и вентиляцию:
        \[
            \zeta_{т.в} = \frac{ h_{т.в} }{ H_0 } =
                \frac{ << (stgd.h_tv / 10**3) | round(4) >> \cdot 10^3 }{ << (stgd.H0 / 10**6) | round(4) >> \cdot 10^6 } =
            << stgd.zeta_tv | round(4) >>
        \]

        \item Мощностной КПД с учетом потерь на трению и вентиляцию:
        \[
            \eta_т = \eta_т^\prime - \zeta_{т.в} =
                << stgd.eta_t_touch |round(4) >> - << stgd.zeta_tv | round(4) >> =
            << stgd.eta_t | round(4) >>
        \]

        \item Лопаточный КПД с учетом потерь на трению и вентиляцию:
        \[
            \eta_л = \eta_л^\prime - \zeta_{т.в} =
                << stgd.eta_l_touch |round(4) >> - << stgd.zeta_tv | round(4) >> =
            << stgd.eta_l | round(4) >>
        \]

        </ if sttype != 'work' />
        \item Определим удельную работу ступени:
        \[
            L_т = H_0 \eta_т = << (stgd.H0 / 10**6) | round(4) >> \cdot 10^6 \cdot \eta_т =
            = << (stgd.L_t / 10**6) | round(4) >> \cdot 10^6 \ Дж/кг
        \]
        </ endif />

        \item Удельная работа ступени, отнесенная к расходу на в СА первой ступени:
        \[
            L_т^\prime = L_т \frac{ G_{вх} }{ G_т }  =
                << (stgd.L_t / 10**6) | round(4) >> \cdot 10^6 \cdot
                \frac{ << stgd.G_stage_in | round(2) >> }{ << stgd.G_turbine | round(2) >> } =
            << (stgd.L_t_prime / 10**6) | round(4) >> \cdot 10^6 \ Дж/кг
        \]

        \item Статическая температура за ступенью:
        \[
            T_{ст} = T_2 + \frac{ h_з }{ c_{pг} } + \frac{ h_{т.в} }{ c_{pг} } =
                << stgd.T2 | round(2) >> +
                \frac{<< (stgd.h_z / 10**3) | round(4) >> \cdot 10^3 }{ << stgd.c_p_gas | round(2) >> } +
                \frac{ << (stgd.h_tv / 10**3) | round(4) >> \cdot 10^3 }{ << stgd.c_p_gas | round(2) >> } =
            << stgd.T_st | round(2) >> \ К
        \]

        \item Температура торможения за турбиной:
        \[
            T_{ст}^* = T_{ст} + \frac{ h_{вых} }{ c_{pг} } =
                << stgd.T_st | round(2) >> +
                \frac{ << (stgd.h_v / 10**3) | round(4) >> \cdot 10^3 }{ << stgd.c_p_gas | round(2) >> } =
            << stgd.T_st_stag | round(2) >> \ К
        \]

        \item Средняя теплоемкость газа в интервале температур от 273 К до $T_0^*$:
        \[
            c_{pг\ ср} (T_0^*, \alpha_{вх}) =
            << stgd.work_fluid.c_p_av_func(stgd.T0_stag, alpha=stgd.alpha_air_in) | round(2) >> \ Дж/(кг \cdot К)
        \]

        \item Средняя теплоемкость газа в интервале температур от 273 К до $T_{ст}^*$:
        \[
            c_{pг\ ср} (T_{ст}^*, \alpha_{вх}) =
            << stgd.work_fluid.c_p_av_func(stgd.T_st_stag, alpha=stgd.alpha_air_in) | round(2) >> \ Дж/(кг \cdot К)
        \]

        \item Средняя теплоемкость газа в интервале температур от $T_0^*$ до $T_{ст}^*$:
        \begin{gather*}
            c_{pг}^\prime = \frac{
		        c_{pг\ ср} (T_0^*, \alpha_{вх}) (T_0^* - T_0) - c_{pг\ ср} (T_{ст}^*, \alpha_{вх})(T_{ст}^* - T_0)
		    }{
		        T_0^* - T_{ст}^*} =\\
            =\frac{
		        << stgd.work_fluid.c_p_av_func(stgd.T0_stag, alpha=stgd.alpha_air_in) | round(2) >> \cdot
                (<< stgd.T0_stag | round(2) >> - 273) -
		        << stgd.work_fluid.c_p_av_func(stgd.T_st_stag, alpha=stgd.alpha_air_in) | round(2) >> \cdot
                (<< stgd.T_st_stag | round(2) >> - 273)
		    }{
		        << stgd.T0_stag | round(2) >> - << stgd.T_st_stag | round(2) >>} =
		    << stgd.work_fluid.c_p_av_int_func(stgd.T_st_stag, stgd.T0_stag, alpha=stgd.alpha_air_in) | round(2) >> \ Дж / (кг \cdot К)\\
        \end{gather*}

        \item Новое значение показателя адиабаты:
        \[
            k_г^\prime = \frac{c_{pг}^\prime}{c_{pг}^\prime - R_г} =
                \frac{
                    << stgd.work_fluid.c_p_av_int | round(2) >>
                }{
                    << stgd.work_fluid.c_p_av_int | round(2) >> - << stgd.work_fluid.R | round(2) >>
                }
            = << stgd.work_fluid.k_av_int | round(4) >>
        \]

        \item Невязка по коэффициенту адиабаты:
        \[
            \delta = \frac{ \left| k_г - k_г^\prime \right| }{ k_г } \cdot 100 \%=
                \frac{
                    \left| << stgd.k_gas | round(4) >> - << stgd.work_fluid.k_av_int | round(4) >> \right|
                }{
                    << stgd.k_gas | round(4) >>
                } \cdot 100 \% =
            << (stgd.k_res * 100) | round(4) >> \%
        \]

        \item Давление торможения на выходе из ступени:
        \[
            p_2^* = p_2 \left(
                            \frac{ T_{ст}^* }{ T_{ст} }
                    \right) ^ \frac{ k_г }{ k_г - 1 } =
                 << (stgd.p2 / 10**6) | round(4) >> \cdot 10^6 \cdot \left(
                            \frac{ << stgd.T_st_stag | round(2) >> }{ << stgd.T_st | round(2) >> }
                    \right) ^
                \frac{ << stgd.k_gas | round(4) >> }{ << stgd.k_gas | round(4) >> - 1 } =
            << (stgd.p2_stag / 10**6) | round(4) >> \cdot 10^6 \ Па
        \]

        \item Теплоперепад по параметрам торможения:
        \begin{gather*}
            H_0^* = c_{pг} T_0^* \left[
                        1 - \left(
                                \frac{p_2^*}{p_0^*}
                            \right) ^
                        \frac{k_г - 1}{k_г}
                    \right] =\\
            << stgd.c_p_gas | round(2) >> \cdot << stgd.T0_stag | round(2) >> \cdot
                    \left[
                        1 - \left(
                                \frac{
                                    << (stgd.p2_stag / 10**6) | round(4) >> \cdot 10^6
                                }{
                                    << (stgd.p0_stag / 10**6) | round(4) >> \cdot 10^6
                                }
                            \right) ^
                        \frac{<< stgd.k_gas | round(4) >> - 1}{<< stgd.k_gas | round(4) >>}
                    \right]
            = << (stgd.H0_stag / 10**6) | round(4)  >> \cdot 10^6 \ Дж/кг\\
        \end{gather*}

        \item КПД по параметрам торможения:
        \[
            \eta_т^* = \frac{ L_т }{ H_0^* } =
                \frac{
                    << (stgd.L_t / 10**6) | round(4) >> \cdot 10^6
                }{
                    << (stgd.H0_stag / 10**6) | round(4)  >> \cdot 10^6 } =
            << stgd.eta_t_stag | round(4) >>
        \]

        \item Расход на выходе из ступени:
        \[
            G_{вых} = G_{вх} + G_т g_{охл} =
                << stgd.G_stage_in | round(2) >> + << stgd.G_turbine | round(2) >> \cdot
                << stgd.g_cool | round(4) >> =
            << stgd.G_stage_out | round(2) >> \ кг/с
        \]

        \item Относительный расход топлива на выходе из ступени:
        \[
            g_{топл.вых} = \frac{ G_{топл} }{ G_{вых} - G_{топл} } =
                 \frac{ << stgd.G_fuel |round(3) >> }{ << stgd.G_stage_out | round(2) >> - << stgd.G_fuel |round(3) >> } =
            << stgd.g_fuel_out | round(4) >>
        \]

        \item Коэффициент избытка воздуха на выходе из ступени:
        \[
            \alpha_{вых} = \frac{ 1 }{ l_0 g_{топл.вых} } =
                \frac{ 1 }{ << stgd.work_fluid.l0 >> \cdot << stgd.g_fuel_out | round(4) >> } =
            << stgd.alpha_air_out | round(3) >>
        \]

        \item Абсолютный расход охлаждающего воздуха:
        \[
            G_{охл} = G_т g_{охл} = << stgd.G_turbine | round(2) >> \cdot << stgd.g_cool | round(4) >> =
            << stgd.G_cool | round(3) >>
        \]

        \item Определим температуру торможения на выходе из ступени после подмешивания охлаждающего воздуха.
        \begin{enumerate}

            \item Истинная теплоемкость охлаждающего воздуха при температуре $T_{охл} = << stgd.T_cool | round(2) >>\ К $:
            \[
                c_{pв} (T_{охл}) = << stgd.cool_fluid.c_p_real_func(stgd.T_cool) | round(2) >>\ Дж/ (кг \cdot К)
            \]

            \item Истинная теплоемкость газа при температуре $T_{ст}^* = << stgd.T_st_stag | round(2) >> \ К $:
            \[
                c_{pг} (T_{ст}^*, \alpha_{вх}) =
                << stgd.work_fluid.c_p_real_func(stgd.T_st_stag, alpha=stgd.alpha_air_in) | round(2) >>\ Дж/ (кг \cdot К)
            \]

            \item Значение температуры смеси с предпоследней итерации $T_{см}^{*} = << stgd.T_mix_stag | round(2) >>\ К$.

            \item Истинная теплоемкость смеси:
            \[
                c_{pг} (T_{см}^{*}, \alpha_{вых}) =
                << stgd.work_fluid.c_p_real_func(stgd.T_mix_stag, alpha=stgd.alpha_air_out) | round(2) >>\ Дж/ (кг \cdot К)
            \]

            \item Новое значение температуры смеси:
            \begin{gather*}
                T_{см}^*\prime = \frac{
                        c_{pг} (T_{ст}^*, \alpha_{вх}) T_{ст}^* G_{вх} + c_{pв} (T_{охл}) T_{охл} G_{охл}
                    }{
                        c_{pг} (T_{см}^{*}, \alpha_{вых}) G_{вых}
                    } =\\
                = \frac{
                    << stgd.work_fluid.c_p_real_func(stgd.T_st_stag, alpha=stgd.alpha_air_in) | round(2) >>
                    \cdot << stgd.T_st_stag | round(2) >> \cdot << stgd.G_stage_in | round(2) >> +
                    << stgd.cool_fluid.c_p_real_func(stgd.T_cool) | round(2) >>
                    \cdot << stgd.T_cool | round(2) >> \cdot << stgd.G_cool | round(3) >>
                }{
                    << stgd.work_fluid.c_p_real_func(stgd.T_mix_stag, alpha=stgd.alpha_air_out) | round(2) >>
                    \cdot  << stgd.G_stage_out | round(2) >>
                } =
                << stgd.T_mix_stag_new | round(2) >>\ К\\
            \end{gather*}

            \item Значение невязки:
            \[
                \delta = \frac{ \left| T_{см}^{*} - T_{см}^*\prime \right| }{T_{см}^{*}} \cdot 100 \% =
                    \frac{
                        \left| << stgd.T_mix_stag | round(2) >> - << stgd.T_mix_stag_new | round(2) >> \right|
                    }{
                        << stgd.T_mix_stag | round(2) >>
                    } \cdot 100 \% =
                << ( stgd.T_mix_stag_res * 100) | round(3) >> \%
            \]

        \end{enumerate}

        </ endmacro />

    \end{enumerate}


    </ macro heat_drop_stage(stgd) />
    \begin{enumerate}

        \item Относительный расход топлива на входе в ступень:
        \[
            g_{топл.вх} = \frac{ G_{топл} }{ G_{вх} - G_{топл} } =
                \frac{ << stgd.G_fuel |round(3) >> }{ << stgd.G_stage_in | round(2) >> - << stgd.G_fuel |round(3) >> } =
            << stgd.g_fuel_in | round(4) >>
        \]

        \item Коэффициент избытка воздуха на входе:
        \[
            \alpha_{вх} = \frac{ 1 }{ l_0 g_{топл.вх} } =
                \frac{ 1 }{ << stgd.work_fluid.l0 >> \cdot << stgd.g_fuel_in | round(4) >> } =
            << stgd.alpha_air_in | round(3) >>
        \]

        \item Средний в ступени коэффициент адиабаты из предпоследней итерации:
        \[
            k_г = << stgd.k_gas | round(4) >>
        \]

        \item Средняя в ступени теплоемкость газа из предпоследней итерации:
        \[
            c_{pг} = << stgd.c_p_gas | round(2) >> \ Дж/(кг \cdot К)
        \]

        << stage_core(stgd, 'heat_drop') >>

    \end{enumerate}
    </ endmacro />


    </ macro pressure_stage(stgd) />

    \begin{enumerate}

        \item Относительный расход топлива на входе в ступень:
        \[
            g_{топл.вх} = \frac{ G_{топл} }{ G_{вх} - G_{топл} } =
                \frac{ << stgd.G_fuel |round(3) >> }{ << stgd.G_stage_in | round(2) >> - << stgd.G_fuel |round(3) >> } =
            << stgd.g_fuel_in | round(4) >>
        \]

        \item Коэффициент избытка воздуха на входе:
        \[
            \alpha_{вх} = \frac{ 1 }{ l_0 g_{топл.вх} } =
                \frac{ 1 }{ << stgd.work_fluid.l0 >> \cdot << stgd.g_fuel_in | round(4) >> } =
            << stgd.alpha_air_in | round(3) >>
        \]

        \item Средний в ступени коэффициент адиабаты из предпоследней итерации:
        \[
            k_г = << stgd.k_gas | round(4) >>
        \]

        \item Средняя в ступени теплоемкость газа из предпоследней итерации:
        \[
            c_{pг} = << stgd.c_p_gas | round(2) >> \ Дж/(кг \cdot К)
        \]

        \item Статический теплоперепад на ступени:
        \[
            H_0 = c_{pг} T_0^* \left[
                        1 - \left(
                            \frac{p_0^*}{p_2} ^ \frac{1 - k_г}{k_г}
                        \right)
                    \right] =
            << stgd.c_p_gas | round(2) >> \cdot << stgd.T0_stag | round(2) >> \cdot \left[
                        1 - \left(
                            \frac{
                                << (stgd.p0_stag / 10**6) | round(4) >> \cdot 10^6
                            }{
                                << (stgd.p2 / 10**6) | round(4) >> \cdot 10^6 } ^
                            \frac{1 - << stgd.k_gas | round(4) >>}{<< stgd.k_gas | round(4) >>}
                        \right)
                    \right] =
            << (stgd.H0 / 10**6) | round(4) >> \cdot 10^6 \ Дж/кг
        \]

        << stage_core(stgd, 'pressure') >>

    \end{enumerate}
    </ endmacro />


    </ macro work_stage(stgd) />

    \begin{enumerate}

        \item Относительный расход топлива на входе в ступень:
        \[
            g_{топл.вх} = \frac{ G_{топл} }{ G_{вх} - G_{топл} } =
                \frac{ << stgd.G_fuel |round(3) >> }{ << stgd.G_stage_in | round(2) >> - << stgd.G_fuel |round(3) >> } =
            << stgd.g_fuel_in | round(4) >>
        \]

        \item Коэффициент избытка воздуха на входе:
        \[
            \alpha_{вх} = \frac{ 1 }{ l_0 g_{топл.вх} } =
                \frac{ 1 }{ << stgd.work_fluid.l0 >> \cdot << stgd.g_fuel_in | round(4) >> } =
            << stgd.alpha_air_in | round(3) >>
        \]

        \item Мощностной КПД из предпоследней итерации:
        \[
            \eta_{т0} = << stgd.eta_t_old | round(4) >>
        \]

        \item Статический теплоперепад на ступени:
        \[
            H_0 = \frac{L_т}{\eta_{т0}} =
                \frac{ << (stgd.L_t / 10**6) | round(4) >> \cdot 10^6 }{ << stgd.eta_t_old | round(4) >> } =
            << (stgd.H0 / 10**6) | round(4) >> \cdot 10^6 \ Дж/кг
        \]

        \item Средний в ступени коэффициент адиабаты из предпоследней итерации:
        \[
            k_г = << stgd.k_gas | round(4) >>
        \]

        \item Средняя в ступени теплоемкость газа из предпоследней итерации:
        \[
            c_{pг} = << stgd.c_p_gas | round(2) >> \ Дж/(кг \cdot К)
        \]

        << stage_core(stgd, 'work') >>

        \item Невязка по мощностному КПД:
        \[
            \delta_\eta = \frac{ \left| \eta_т - \eta_{т0} \right| }{ \eta_{т0} } \cdot 100 \% =
                \frac{
                    \left| << stgd.eta_t | round(4) >> - << stgd.eta_t_old | round(4) >> \right|
                }{
                    << stgd.eta_t_old | round(4) >> } \cdot 100 \% =
            << (stgd.eta_t_res * 100) | round(3) >> \%
        \]

    \end{enumerate}
     </ endmacro />


    </ macro stage_init_data(stgd, sttype) />
    \begin{enumerate}

        \item Температура торможения на входе в ступень: $T_0^* = << stgd.T0_stag | round(2) >>\ К $.
        \item Давление торможения на входе в ступень: $p_0^* = << (stgd.p0_stag / 10**6) | round(4) >> \cdot 10^6 \ Па$.
        \item Расход газа на входе в ступень: $G_{вх} = << stgd.G_stage_in | round(2) >>\ кг/с$.
        \item Расход газа на входе в СА первой ступени: $ G_т = << stgd.G_turbine | round(2) >>\ кг/с $.
        \item Расход топлива на входе в турбину: $ G_{топл} = << stgd.G_fuel | round(3) >>\ кг/с $.
        \item Степень реактивности: $ \rho = << stgd.rho | round(3) >> $.
        \item Коэффициент скорости в СА: $ \phi = << stgd.phi | round(3) >> $.
        \item Коэффициент скорости в РК: $ \psi = << stgd.psi | round(3) >> $.
        \item Длина лопатки на входе в РК: $ l_1 = << stgd.l1 | round(4) >>\ м $.
        \item Длина лопатки на выходе из РК: $ l_2 = << stgd.l2 | round(4) >>\ м $.
        \item Средний диаметр на входе в РК: $ D_1 = << stgd.D1 | round(4) >>\ м $.
        \item Средний диаметр на выходе в РК: $ D_2 = << stgd.D2 | round(4) >>\ м $.
        \item Радиальный зазор: $ \delta_r = << stgd.delta_r_rk | round(5) >>\ м $.
        \item Частота вращения ротора: $ n = << stgd.n | round(1) >>\ об/мин $
        \item Степень парциальности: $ \varepsilon = << stgd.epsilon | round(2) >> $.
        \item Расход охлаждающего воздуха, отнесенный к расходу на входе в турбину: $ g_{охл} = << stgd.g_cool | round(4) >> $.
        \item Температура торможения охлаждающего воздуха: $ T_{охл} = << stgd.T_cool | round(2) >>\ К $.

        </ if sttype=='heat_drop' />
        \item Статический теплоперепад на ступени: $ H_0 = << (stgd.H0 / 10**6) | round(4) >> \cdot 10^6 \ Дж/кг $.

        </ elif sttype=='pressure' />
        \item Статическое давление на выходе: $ p_2 = << (stgd.p2 / 10**6) | round(4) >> \cdot 10^6 \ Па $.

        </ elif sttype=='work' />
        \item Удельная работы турбины: $ L_т = << (stgd.L_t / 10**6) | round(4)  >> \cdot 10^6 \ Дж/кг $.

        </ endif />

    \end{enumerate}
    </ endmacro />


    </ macro integrate_turbine_param(turb, L_t_sum_exp) />
    \begin{enumerate}

        \item Суммарная работа всех ступеней:
        \[
            L_{т\Sigma} = << L_t_sum_exp >> = << (turb.L_t_sum / 10**6) | round(4) >> \cdot 10^6 \ Дж/кг
        \]

        \item Средняя теплоемкость газа в интервале температур от 273 К до $T_г^*$:
        \[
            c_{pг\ ср} (T_г^*, \alpha_{вх}) =
            << turb.work_fluid.c_p_av_func(turb.T_g_stag, alpha=turb.first.alpha_air_in) | round(2) >> \ Дж/(кг \cdot К)
        \]

        \item Средняя теплоемкость газа в интервале температур от 273 К до $T_т$:
        \[
            c_{pг\ ср} (T_т, \alpha_{вх}) =
            << turb.work_fluid.c_p_av_func(turb.last.T_st, alpha=turb.first.alpha_air_in) | round(2) >> \ Дж/(кг \cdot К)
        \]

        \item Средняя теплоемкость газа в интервале температур от $T_0^*$ до $T_т$:
        \begin{gather*}
            c_{pг} = \frac{
		         c_{pг\ ср} (T_г^*, \alpha_{вх}) (T_г^* - T_0) - c_{pг\ ср} (T_{т}, \alpha_{вх})(T_т - T_0)
		    }{
		        T_г^* - T_т} =\\
            =\frac{
                << turb.work_fluid.c_p_av_func(turb.T_g_stag, alpha=turb.first.alpha_air_in) | round(2) >> \cdot
                (<< turb.T_g_stag | round(2) >> - 273) -
		        << turb.work_fluid.c_p_av_func(turb.last.T_st, alpha=turb.first.alpha_air_in) | round(2) >> \cdot
                (<< turb.last.T_st | round(2) >> - 273)
		    }{
		        << turb.T_g_stag | round(2) >> - << turb.last.T_st | round(2) >>} =
		    << turb.c_p_gas | round(2) >> \ Дж / (кг \cdot К)\\
        \end{gather*}

        \item Средний показателя адиабаты:
        \[
            k_г = \frac{c_{pг}}{c_{pг} - R_г} =
                \frac{
                    << turb.c_p_gas | round(2) >>
                }{
                    << turb.c_p_gas | round(2) >> - << turb.work_fluid.R | round(2) >>
                }
            = << turb.k_gas | round(4) >>
        \]

        \item Статический теплоперепад на турбине:
        \begin{gather*}
            H_т = c_{pг} T_г^* \left[
                        1 - \left(
                                \frac{p_г^*}{p_т} ^
                                \frac{1 - k_г}{k_г}
                    \right)
                \right] =\\
            = << turb.c_p_gas | round(2) >> \cdot << turb.T_g_stag | round(2) >>
                \left[
                    1 - \left(
                            \frac{
                                << (turb.p_g_stag / 10**6) | round(4) >> \cdot 10^6
                            }{
                                << (turb.last.p2 / 10**6) | round(4) >> \cdot 10^6 } ^
                            \frac{ 1 - << turb.k_gas | round(4) >> }{ << turb.k_gas | round(4) >> }
                    \right)
            \right] =
            << (turb.H_t / 10**6) | round(4) >> \cdot 10^6 \ Дж/кг\\
        \end{gather*}

        \item Средняя теплоемкость газа в интервале температур от 273 К до $T_т^*$:
        \[
            c_{pг\ ср} (T_т^*, \alpha_{вх}) =
            << turb.work_fluid.c_p_av_func(turb.last.T_mix_stag, alpha=turb.first.alpha_air_in) | round(2) >> \ Дж/(кг \cdot К)
        \]

        \item Средняя теплоемкость газа в интервале температур от $T_0^*$ до $T_т^*$:
        \begin{gather*}
            c_{pг}^* = \frac{
		         c_{pг\ ср} (T_г^*, \alpha_{вх}) (T_г^* - T_0) - c_{pг\ ср} (T_т^*, \alpha_{вх})(T_т^* - T_0)
		    }{
		        T_г^* - T_т^*} =\\
            =\frac{
                << turb.work_fluid.c_p_av_func(turb.T_g_stag, alpha=turb.first.alpha_air_in) | round(2) >> \cdot
                (<< turb.T_g_stag | round(2) >> - 273) -
		        << turb.work_fluid.c_p_av_func(turb.last.T_mix_stag, alpha=turb.first.alpha_air_in) | round(2) >> \cdot
                (<< turb.last.T_mix_stag | round(2) >> - 273)
		    }{
		        << turb.T_g_stag | round(2) >> - << turb.last.T_mix_stag | round(2) >>} =\\
		     = << turb.c_p_gas_stag | round(2) >> \ Дж / (кг \cdot К)\\
        \end{gather*}

        \item Средний показателя адиабаты по параметрам торможения:
        \[
            k_г^* = \frac{ c_{pг}^* }{ c_{pг}^* - R_г } =
                \frac{
                    << turb.c_p_gas_stag | round(2) >>
                }{
                    << turb.c_p_gas_stag | round(2) >> - << turb.work_fluid.R | round(2) >>
                }
            = << turb.k_gas_stag | round(4) >>
        \]

        \item Теплоерепад на турбине оп параметрам торможения:
        \begin{gather*}
            H_т^* = c_{pг}^* T_г^* \left[
                        1 - \left(
                                \frac{p_г^*}{p_т^*} ^
                                \frac{1 - k_г^*}{k_г^*}
                    \right)
                \right] =\\
            =<< turb.c_p_gas_stag | round(2) >> \cdot << turb.T_g_stag | round(2) >>
                \left[
                    1 - \left(
                            \frac{
                                << (turb.p_g_stag / 10**6) | round(4) >> \cdot 10^6
                            }{
                                << (turb.last.p2_stag / 10**6) | round(4) >> \cdot 10^6 } ^
                            \frac{ 1 - << turb.k_gas_stag | round(4) >> }{ << turb.k_gas_stag | round(4) >> }
                    \right)
            \right] =
            << (turb.H_t_stag / 10**6) | round(4) >> \cdot 10^6 \ Дж/кг\\
        \end{gather*}

        \item Мощностной КПД турбины:
        \[
            \eta_т = \frac{ L_{т\Sigma} }{ H_т } =
                \frac{ << (turb.L_t_sum / 10**6) | round(4) >> \cdot 10^6 }{ << (turb.H_t / 10**6) | round(4) >> \cdot 10^6 } =
            << turb.eta_t | round(4) >>
        \]

        \item Лопаточный КПД турбины:
        \[
            \eta_л = \frac{
                        L_{т\Sigma} + 0.5 \cdot c_{вых}^2
                    }{ H_т } =
            \frac{
                << (turb.L_t_sum / 10**6) | round(4) >> \cdot 10^6 + 0.5 \cdot << turb.last.c2 | round(2) >> ^ 2
            }{ << (turb.H_t / 10**6) | round(4) >> \cdot 10^6 } =
            << turb.eta_l | round(4) >>
        \]

        \item КПД турбины по параметрам торможения:
        \[
            \eta_т^* = \frac{ L_{т\Sigma} }{ H_т^* } =
                \frac{ << (turb.L_t_sum / 10**6) | round(4) >> \cdot 10^6 }{ << (turb.H_t_stag / 10**6) | round(4) >> \cdot 10^6 } =
            << turb.eta_t_stag | round(4) >>
        \]

        \item Степень понижения давления по статическим параметрам:
        \[
            \pi_{т} = \frac{p_{01}^*}{p_{2 << turb.stage_number >>}} =
            \frac{<< (turb.first.p0_stag / 10**6) | round(4) >>}{<< (turb.last.p2 / 10**6) | round(4) >>} =
            << (turb.pi_t) | round(3) >>
        \]

        \item Степень понижения давления по параметрам торможения:
        \[
            \pi_{т}^* = \frac{p_{01}^*}{p_{2 << turb.stage_number >>}^*} =
            \frac{<< (turb.first.p0_stag / 10**6) | round(4) >>}{<< (turb.last.p2_stag / 10**6) | round(4) >>} =
            << (turb.pi_t_stag) | round(3) >>
        \]

    \end{enumerate}
    </ endmacro />

\end{document}